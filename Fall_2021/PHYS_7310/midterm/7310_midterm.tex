\documentclass[12pt]{article}

%%%%%%%%%%%%%%%%%%%%%%%%%%%%%%%%%%%%%%%%%%%%%%%%%%%%%%%%%%%%%%%%%%%%%%%%%%%%%%%%
%                           Package preset for homework
%%%%%%%%%%%%%%%%%%%%%%%%%%%%%%%%%%%%%%%%%%%%%%%%%%%%%%%%%%%%%%%%%%%%%%%%%%%%%%%%
% Miscellaneous
\usepackage[margin=1in]{geometry}
\usepackage[utf8]{inputenc}
\usepackage{indentfirst}
\usepackage{blindtext}
\usepackage{graphicx}
\usepackage{xr-hyper}
\usepackage{hyperref}
\usepackage{enumitem}
\usepackage{color}
\usepackage{float}
% Math
\usepackage{latexsym}
\usepackage{amsfonts}
\usepackage{amssymb}
\usepackage{amsmath}
\usepackage{commath}
\usepackage{amsthm}
\usepackage{bbold}
\usepackage{bm}
% Physics
\usepackage{physics}
\usepackage{siunitx}
% Code typesetting
\usepackage{listings}
% Citation
\usepackage[authoryear]{natbib}
\usepackage{appendix}
\usepackage[capitalize]{cleveref}
% Title & name
\title{Homework}
\author{Tien Vo}
\date{\today}


%%%%%%%%%%%%%%%%%%%%%%%%%%%%%%%%%%%%%%%%%%%%%%%%%%%%%%%%%%%%%%%%%%%%%%%%%%%%%%%%
%                   User-defined commands and environments
%%%%%%%%%%%%%%%%%%%%%%%%%%%%%%%%%%%%%%%%%%%%%%%%%%%%%%%%%%%%%%%%%%%%%%%%%%%%%%%%
%%% Misc
\sisetup{load-configurations=abbreviations}
\newcommand{\due}[1]{\date{Due: #1}}
\newcommand{\hint}{\textit{Hint}}
\let\oldt\t
\renewcommand{\t}[1]{\text{#1}}

%%% Bold sets & abbrv
\newcommand{\N}{\mathbb{N}}
\newcommand{\Z}{\mathbb{Z}}
\newcommand{\R}{\mathbb{R}}
\newcommand{\Q}{\mathbb{Q}}
\let\oldP\P
\renewcommand{\P}{\mathbb{P}}
\newcommand{\LL}{\mathcal{L}}
\newcommand{\FF}{\mathcal{F}}
\newcommand{\HH}{\mathcal{H}}
\newcommand{\NN}{\mathcal{N}}
\newcommand{\ZZ}{\mathcal{Z}}
\newcommand{\RN}[1]{\textup{\uppercase\expandafter{\romannumeral#1}}}
\newcommand{\ua}{\uparrow}
\newcommand{\da}{\downarrow}

%%% Unit vectors
\newcommand{\xhat}{\vb{\hat{x}}}
\newcommand{\yhat}{\vb{\hat{y}}}
\newcommand{\zhat}{\vb{\hat{z}}}
\newcommand{\nhat}{\vb{\hat{n}}}
\newcommand{\rhat}{\vb{\hat{r}}}
\newcommand{\phihat}{\bm{\hat{\phi}}}
\newcommand{\thetahat}{\bm{\hat{\theta}}}

%%% Other math stuff
\providecommand{\units}[1]{\,\ensuremath{\mathrm{#1}}\xspace}
% Set new style for problem
\newtheoremstyle{problemstyle}  % <name>
        {10pt}                   % <space above>
        {10pt}                   % <space below>
        {\normalfont}           % <body font>
        {}                      % <indent amount}
        {\bfseries\itshape}     % <theorem head font>
        {\normalfont\bfseries:} % <punctuation after theorem head>
        {.5em}                  % <space after theorem head>
        {}                      % <theorem head spec (can be left empty, 
                                % meaning `normal')>

% Set problem environment
\theoremstyle{problemstyle}
\newtheorem{problemenv}{Problem}[section]
\newenvironment{problem}[1]{%
  \renewcommand\theproblemenv{#1}%
  \problemenv
}{\endproblemenv}
% Set lemma environment
\newenvironment{lemma}[2][Lemma]{\begin{trivlist}
\item[\hskip \labelsep {\bfseries #1}\hskip \labelsep {\bfseries #2.}]}{\end{trivlist}}
% Set solution environment
\newenvironment{solution}{
    \begin{proof}[Solution]$ $\par\nobreak\ignorespaces
}{\end{proof}}
\numberwithin{equation}{problemenv}

%%% Page format
\setlength{\parindent}{0.5cm}
\setlength{\oddsidemargin}{0in}
\setlength{\textwidth}{6.5in}
\setlength{\textheight}{8.8in}
\setlength{\topmargin}{0in}
\setlength{\headheight}{18pt}

%%% Code environments
\definecolor{dkgreen}{rgb}{0,0.6,0}
\definecolor{gray}{rgb}{0.5,0.5,0.5}
\definecolor{mauve}{rgb}{0.58,0,0.82}
\lstset{frame=tb,
  language=Python,
  aboveskip=3mm,
  belowskip=3mm,
  showstringspaces=false,
  columns=flexible,
  basicstyle={\small\ttfamily},
  numbers=none,
  numberstyle=\tiny\color{gray},
  keywordstyle=\color{blue},
  commentstyle=\color{dkgreen},
  stringstyle=\color{mauve},
  breaklines=true,
  breakatwhitespace=true,
  tabsize=4
}
\lstset{
  language=Mathematica,
  numbers=left,
  numberstyle=\tiny\color{gray},
  numbersep=5pt,
  breaklines=true,
  captionpos={t},
  frame={lines},
  rulecolor=\color{black},
  framerule=0.5pt,
  columns=flexible,
  tabsize=2
}


\title{Midterm: Phys 7310 (Fall 2021)}

\begin{document}
\maketitle
%%%%%%%%%%%%%%%%%%%%%%%%%%%%%%%%%%%%%%%%%%%%%%%%%%%%%%%%%%%%%%%%%%%%%%%%%%%%%%%%
\begin{problem}{M.1}[Charge density of an electron]
An electron in an atom has the charge density
\begin{equation}\label{p1:rho}
    \rho(\vb{x})=B\cos^2\theta e^{-r/a} 
\end{equation}
where $a$ and $B$ are constants.

(a) Calculate all the nonzero multipole moments $q_{lm}$ for this charge
distribution.

(b) From the $q_{lm}$ obtain the total charge $q$, dipole moment $\vb{p}$, and
quadrupole tensor $Q_{ij}$. Use these to directly write down the scalar
potential $\Phi(\vb{x})$ generated by the charge density.
\begin{solution}
(a) From (4.3, Jackson), the multipole moments $q_{lm}$ are
\begin{align}\label{p1a:q_lm1}
    q_{lm}
    &=\int Y_{lm}^\ast(\Omega')r'^l\rho(\vb{x}')d^3x'\notag\\
    &=B\sqrt{\frac{2l+1}{4\pi}\frac{(l-m)!}{(l+m)!}}
    \int_0^\infty dr'r'^{l+2}e^{-r'/a}
    \int_{-1}^1d(\cos\theta')P_l^m(\cos\theta')\cos^2\theta'
    \int_0^{2\pi}e^{-im\phi'}
\end{align}
Note that for $m\neq 0$, the last integration is proportional to
$\eval{\exp{-im\phi'}}_{\phi'=0}^{\phi'=2\pi}=0$. Thus, $q_{lm}$ are only
non-zero for $m=0$. This is reflected by the azimuthal symmetry in the charge
distribution \eqref{p1:rho}. Then, \eqref{p1a:q_lm1} becomes
\begin{align}
    q_{lm}
    &=2\pi B\sqrt{\frac{2l+1}{4\pi}}\int_0^{\infty}dr'r'^{l+2}e^{-r'/a}
    \int_{-1}^1dxx^2P_l(x)\tag{$x=\cos\theta'$}\\
    &=2\pi a^{l+3}B\sqrt{\frac{2l+1}{4\pi}}\int_0^\infty du u^{(l+3)-1}e^{-u}
    \int_{-1}^1dx x^2P_l(x)\tag{$u=r'/a$}\\
    &=2\pi a^{l+3}B\sqrt{\frac{2l+1}{4\pi}}\Gamma(l+3)\begin{cases}
        2/3 & l=0\\
        4/15 & l=2
    \end{cases}
\end{align}
where we have used (3.32, Jackson) for the Legendre polynomial integration. The
only two non-zero moments are then
\begin{equation}
    q_{00}=\frac43\sqrt\pi a^3B
    \qquad\text{and}\qquad
    q_{20}=32\sqrt{\frac\pi5}a^5B
\end{equation}

(b) From (4.4, Jackson), we can calculate the charge
\begin{equation}
    q=\sqrt{4\pi}q_{00}=\frac{8}{3}\pi a^3B
\end{equation}
From (4.5, Jackson), $\vb{p}=\bm{0}$ since $q_{lm}=0$ for $l=1$. From (4.6,
Jackson), 
\begin{equation}
    Q_{33}=2\sqrt{\frac{4\pi}{5}}q_{20}=\frac{128\pi}{5}a^5B
\end{equation}
Also, because $q_{21}=-q_{2,-1}^\ast=0$, $Q_{13}=Q_{23}=0$. Similarly, because
$q_{22}=q_{2,-2}^\ast=0$, $Q_{11}-Q_{22}=Q_{12}=0$. Finally, requiring that
$Q_{ij}$ is traceless, we can calculate
\begin{equation}
    Q_{11}=Q_{22}=-\frac12Q_{33}=-\frac{64\pi}{5}a^5B 
\end{equation}
Then from (4.10), the potential is
\begin{align}
    \Phi
    &=\frac1{4\pi\epsilon_0}\qty[\frac{q}{r}+\frac12\qty(
    Q_{11}\frac{x^2}{r^5}+Q_{22}\frac{y^2}{r^5}+Q_{33}\frac{z^2}{r^5}
    )]\notag\\
    &=\frac1{4\pi\epsilon_0}\qty(\frac{q}{r}
    +\frac{Q_{11}}{2}\frac{x^2+y^2-2z^2}{r^5})\notag\\
    &=\frac{a^3B}{\epsilon_0}\qty(\frac{2}{3r}
    -\frac{8a^2}{5}\frac{x^2+y^2-2z^2}{r^5})
\end{align}
\end{solution}
\end{problem}
%%%%%%%%%%%%%%%%%%%%%%%%%%%%%%%%%%%%%%%%%%%%%%%%%%%%%%%%%%%%%%%%%%%%%%%%%%%%%%%%    
%%%%%%%%%%%%%%%%%%%%%%%%%%%%%%%%%%%%%%%%%%%%%%%%%%%%%%%%%%%%%%%%%%%%%%%%%%%%%%%%
\newpage
\begin{problem}{M.2}[Charged cylinder]
A hollow cylindrical tube of radius $R$ is centered on the negative $z$-axis,
extending from $z=0$ to $z=-\infty$. It is open on the ends, with no end caps.
Constant surface charge densty $\sigma$ is attached to the surface of the tube.

(a) Find and evaluate an integral for the electrostatic potential $\Phi(z)$
everywhere along the $z$-axis. In evaluating the integral, you will find a term
that is infinite, but independent of $z$; drop this term, and explain briefly
why it is physically reasonable to do so.

(b) Use the result of part (a) to find an expression for $\Phi(r,\theta)$ for
any $\theta$ with $r\ll R$ in terms of Legendre polynomials, keeping the first
two terms in the expansion in $r /R$.
\begin{solution}
\end{solution}
\end{problem}
%%%%%%%%%%%%%%%%%%%%%%%%%%%%%%%%%%%%%%%%%%%%%%%%%%%%%%%%%%%%%%%%%%%%%%%%%%%%%%%%
%%%%%%%%%%%%%%%%%%%%%%%%%%%%%%%%%%%%%%%%%%%%%%%%%%%%%%%%%%%%%%%%%%%%%%%%%%%%%%%%
\newpage
\begin{problem}{M.3}[Dielectric sphere]
A dielectric sphere with radius $a$ and permittivity $\epsilon$ has a fixed
density of free charge attached to its surface:
\begin{equation}
    \sigma_{\text{free}}=\sigma_0\cos\theta, 
\end{equation}
with $\sigma_0$ a constant. The sphere sits centered at the origin in otherwise
empty space.

(a) Find the potential $\Phi(\vb{x})$ everywhere in space.

(b) For the region inside the sphere, find the electric field $\vb{E}$, electric
displacement $\vb{D}$ and polarization $\vb{P}$. (Hint: it is useful to start by
passing to Cartesian coordinates.)

(c) Find the polarization surface charge density $\sigma_{\text{pol}}$ on the
surface of the sphere. Is there bulk polarization charge density
$\rho_{\text{pol}}$ within the sphere? Explain.
\begin{solution}
(a) This charge density has azimuthal symmetry. So we can use the general
solution (3.33, Jackson) to the Laplace equation
\begin{equation}
    \Phi=\sum_{l=0}^\infty\qty[A_lr^l+B_lr^{-(l+1)}]P_l(\cos\theta)
\end{equation}
for the potential. Inside the sphere ($r<a$), the radial term with negative
power has to vanish because it contains the origin. Outside the sphere $(r\geq
a$), the radial term with positive power has to vanish because the potential
tends to zero at infinity.
\begin{subequations}\label{p3a:Phi}
    \begin{align}
        \Phi_{\text{in}}&=\sum_{l=0}^\infty A_lr^lP_l(\cos\theta)\\
        \Phi_{\text{out}}&=\sum_{l=0}^\infty B_lr^{-(l+1)}P_l(\cos\theta)
    \end{align} 
\end{subequations}
At $r=a$, the potential is continuous
\begin{equation}
    \eval{\Phi_{\text{in}}}_{r=a}
    =\eval{\Phi_{\text{out}}}_{r=a}\Rightarrow
    \sum_{l=0}^\infty A_la^lP_l(\cos\theta)=\sum_{l=0}^\infty
    B_la^{-(l+1)}P_l(\cos\theta)
\end{equation}
Because the Legendre polynomials $P_l$ are orthogonal, the coefficients must be
equal and we can write
\begin{equation}\label{p3a:B_l}
    B_l=A_la^{2l+1} 
\end{equation}
Also, at $r=a$, the electric displacement follows the boundary condition
\begin{equation}
    D_{\text{out}}-D_{\text{in}}=\eval{\qty[-\epsilon_0\frac{\partial\Phi_{\text{out}}}{\partial
    r}+\epsilon\frac{\partial\Phi_{\text{in}}}{\partial
    r}]}_{r=a}=\sigma_{\text{free}} 
\end{equation}
From \eqref{p3a:Phi} and \eqref{p3a:B_l}, we then have
\begin{equation}
    \sum_{l=0}^\infty\qty[\epsilon l+\epsilon_0(l+1)]a^{l-1}A_lP_l(\cos\theta)
    =\sigma_0\cos\theta=\sigma_0P_1(\cos\theta)
\end{equation}
The only non-trivial term must then be $l=1$ where
\begin{equation}
    A_l=\frac{\sigma_0}{\epsilon+2\epsilon_0}\Rightarrow
    B_l=\frac{\sigma_0}{\epsilon+2\epsilon_0}a^{2l+1}
\end{equation}
Then the potential inside and outside the sphere is
\begin{subequations}
    \begin{align}
        \Phi_{\text{in}}&=
        A_1r\cos\theta=\frac{\sigma_0}{\epsilon+2\epsilon_0}r\cos\theta
        \label{p3a:Phi_in}\\
        \Phi_{\text{out}}
        &=\frac{B_1}{r^2}\cos\theta
        =\frac{\sigma_0}{\epsilon+2\epsilon_0}\frac{a^3}{r^2}\cos\theta
    \end{align} 
\end{subequations}

(b) In Cartesian coordinates, the potential \eqref{p3a:Phi_in} inside the 
sphere is
\begin{equation}
    \Phi_{\text{in}}(x,y,z)=\frac{\sigma_0}{\epsilon+2\epsilon_0}z 
\end{equation}
By definition, the electric field is thus
\begin{equation}
    \vb{E}_{\text{in}}=-\grad\Phi_{\text{in}}=-\frac{\sigma_0}{\epsilon+2\epsilon_0}\zhat 
\end{equation}
Then from (4.37, Jackson), the electric displacement is
\begin{equation}
    \vb{D}=\epsilon\vb{E}=-\frac{\epsilon}{\epsilon+2\epsilon_0}\sigma_0\zhat 
\end{equation}
and from (4.36, Jackson), the polarization is
\begin{equation}
    \vb{P}=\qty(\epsilon-\epsilon_0)\vb{E}
    =-\frac{\epsilon-\epsilon_0}{\epsilon+2\epsilon_0}\sigma_0\zhat
    =-\frac{\epsilon-\epsilon_0}{\epsilon+2\epsilon_0}\sigma_0
    \qty(\cos\theta\rhat-\sin\theta\thetahat)
\end{equation}

(c) From (4.46, Jackson), the polarization surface charge density at $r=a$ is
\begin{equation}
    \sigma_{\text{pol}}=\eval{\vb{P}\vdot\rhat}_{r=a}
    =-\frac{\epsilon-\epsilon_0}{\epsilon+2\epsilon_0}\sigma_0\cos\theta
\end{equation}
There is no polarization volume charge density inside the sphere because the
polarization $\vb{P}$ is uniform. We can also show this explicitly by
calculating
\begin{equation}
    \div{\vb{P}}=-\frac{\epsilon-\epsilon_0}{\epsilon+2\epsilon_0}\sigma_0
    \qty[\frac{\cos\theta}{r^2}\frac{\partial(r^2)}{\partial
    r}-\frac1{r\sin\theta}\frac{\partial(\sin^2\theta)}{\partial\theta}]
    \sim\frac{2\cos\theta}{r}-\frac{2\sin\theta\cos\theta}{r\sin\theta}
    =0
\end{equation}
\end{solution}
\end{problem}
%%%%%%%%%%%%%%%%%%%%%%%%%%%%%%%%%%%%%%%%%%%%%%%%%%%%%%%%%%%%%%%%%%%%%%%%%%%%%%%%
%%%%%%%%%%%%%%%%%%%%%%%%%%%%%%%%%%%%%%%%%%%%%%%%%%%%%%%%%%%%%%%%%%%%%%%%%%%%%%%%
\newpage
\begin{problem}{M.4}[Potential in a rectangular tube]

(a) Consider a semi-infinite rectangular tube region defined by $0<x<a,0<y<a$
and $0<z<\infty$. Construct a Dirichlet Green's function for this region (with
$z\to\infty$ treated as part of the boundary) of the form
\begin{equation}\label{p4:G}
    G(\vb{x},\vb{x}')=\qty(\frac{2}{a})^2\sum_{m,n=1}^\infty\sin\qty(\frac{m\pi
    x}{a})\sin\qty(\frac{m\pi x'}{a})\sin\qty(\frac{n\pi
y}{a})\sin\qty(\frac{n\pi y'}{a})g_{m,n}(z,z')
\end{equation}
by deriving an equation for $g_{m,n}(z,z')$ and solving it with suitable
boundary conditions (first at $z=0$ and $z\to\infty$, then at $z=z'$) as well as
requiring symmetry between $z$ and $z'$. For simplicity, let
$k^2\equiv\pi^2(m^2+n^2) /a^2$.

(b) The potential on the $z=0$ square is held at a constant value $V$, while the
rest of the walls of the tube are grounded, and the potential vanishes at
infinity. Find an expression for $\Phi(\vb{x})$ everywhere in the region, and
find the limiting behavior at large $z$. To do this you may either use the
Green's function you constructed in part (a), or use separation of variables and
impose the boundary conditions.
\begin{solution}
(a) From (1.39, Jackson), the Green function \eqref{p4:G} follows
$\grad'^2G(\vb{x},\vb{x}')=-4\pi\delta(\vb{x}-\vb{x}')$. So we can write
\begin{align}\label{p4a:LHS1}
    \qty(\frac2a)^2\sum_{m,n=1}^\infty
    &\sin\qty(\frac{m\pi x}{a})\sin\qty(\frac{m\pi x'}{a})
    \sin\qty(\frac{n\pi y}{a})\sin\qty(\frac{n\pi y'}{a})
    \qty[\frac{\partial^2}{\partial z'^2}
    -\frac{\pi^2(m^2+n^2)}{a^2}]g_{m,n}(z,z')\notag\\
    &=-4\pi\delta(x-x')\delta(y-y')\delta(z-z')
\end{align}
By completeness, the delta functions can be expanded into
\begin{equation}\label{p4a:RHS1}
    \text{RHS}=-4\pi\qty(\frac2a)^2\sum_{m,n=1}^\infty 
    \sin\qty(\frac{m\pi x}{a})\sin\qty(\frac{m\pi x'}{a})
    \sin\qty(\frac{n\pi y}{a})\sin\qty(\frac{n\pi y'}{a})
    \delta(z-z')
\end{equation}
Recall from Homework 3, the orthogonality condition is
\begin{equation}
    \delta_{m,n}=\braket{\sqrt{\frac2a}\sin\qty(\frac{m\pi x}{a})}{\sqrt{\frac2a}\sin\qty(\frac{m\pi x}{a})}
    =\frac2a\int_0^a\sin\qty(\frac{m\pi x}{a})\sin\qty(\frac{n\pi x}{a})dx
\end{equation}
Then, we can take the inner product of both the LHS of \eqref{p4a:LHS1} and the
RHS of \eqref{p4a:RHS1} with $\sqrt{2 /a}\sin(m'\pi x /a), \sqrt{2 /a}\sin(m'\pi
x' /a), \sqrt{2 /a}\sin(n'\pi y /a),$ and $\sqrt{2 /a}\sin(n'\pi y' /a)$ to
yield
\begin{equation}
    \sum_{m,n=1}^\infty\delta_{m,m'}\delta_{m,m'}\delta_{n,n'}\delta_{n,n'}\qty(\frac{\partial^2}{\partial
    z'^2}-k^2)g_{m,n}(z,z')=-4\pi\sum_{m,n=1}^\infty\delta_{m,m'}\delta_{m,m'}\delta_{n,n'}\delta_{n,n'}\delta(z-z')
\end{equation}
where we have defined $k^2=\pi^2(m^2+n^2) /a^2$. It follows that
\begin{equation}\label{p4a:DE}
    \qty(\frac{\partial^2}{\partial z'^2}-k^2)g_{m,n}(z,z')=-4\pi\delta(z-z') 
\end{equation}
where we have dropped the primes on $m$ and $n$. Recall from Problem 3.4 in
Homework 3, the general solution to this differential equation is
\begin{equation}
    g_{m,n}(z,z')=\begin{cases}
        A_{m,n}e^{kz'}+B_{m,n}e^{-kz'} & z'<z\\
        C_{m,n}e^{kz'}+D_{m,n}e^{-kz'} & z'\geq z
    \end{cases}
\end{equation}
Note that we have rewritten the $\sinh$ and $\cosh$ basis into exponentials for
their apparent behavior at $\infty$. Given a finite $z\in\R$, $g_{m,n}$ has to
vanish at $z'\to\infty$. So $C_{m,n}$ must be zero. Also, imposing the Dirichlet
boundary condition at $z'=0$, $g_{m,n}(z,0)=A_{m,n}+B_{m,n}=0$. So we now have
\begin{equation}
    g_{m,n}(z,z')=\begin{cases}
        2A_{m,n}\sinh(kz') & z'<z\\
        D_{m,n}e^{-kz'} & z'\geq z
    \end{cases}
\end{equation}
At $z'=z$, $g_{m,n}$ has to be continuous. So we can write
\begin{equation}\label{p4a:EQ1}
    D_{m,n}=2A_{m,n}\sinh(kz)e^{kz} 
\end{equation}
Also, we proved in Homework 3.4 from \eqref{p4a:DE} that $g_{m,n}$ has to 
satisfy the jump condition
\begin{equation}\label{p4a:EQ2}
    \lim_{\epsilon\to0}\qty(
    \eval{\frac{\partial g_{m,n}}{\partial z'}}_{z'=z+\epsilon}
    -\eval{\frac{\partial g_{m,n}}{\partial z'}}_{z'=z-\epsilon}
    )=-4\pi
    \Rightarrow D_{m,n}e^{-kz}+2A_{m,n}\cosh(kz)=\frac{4\pi}{k}
\end{equation}
From \eqref{p4a:EQ1} and \eqref{p4a:EQ2}, we can solve for
\begin{equation}
    A_{m,n}=\frac{2\pi}{k}e^{-kz}
    \qquad\text{and}\qquad
    D_{m,n}=\frac{4\pi}{k}\sinh(kz)
\end{equation}
Then the final solution for $g_{m,n}$ is
\begin{equation}\label{p4a:g_1}
    g_{m,n}(z,z')=\begin{cases}
        \frac{4\pi}{k}\sinh(kz')e^{-kz} & z'<z\\
        \frac{4\pi}{k}\sinh(kz)e^{-kz'} & z'\geq z
    \end{cases}
\end{equation}
or we can write more succinctly as
\begin{equation}
    g_{m,n}(z,z')=\frac{4\pi}{k}\sinh(kz_<)e^{-kz_>} 
\end{equation}
where $z_< =\min(z,z')$ and $z_> =\max(z,z')$. From \eqref{p4a:g_1}, it can be
verified that $g_{m,n}$ is symmetric between $z$ and $z'$
\begin{equation}
    g_{m,n}(z',z)=\begin{cases}
        \frac{4\pi}{k}\sinh(kz)e^{-kz'} & z<z'\\
        \frac{4\pi}{k}\sinh(kz')e^{-kz} & z\geq z'
    \end{cases}
    \Rightarrow g_{m,n}(z,z')=g_{m,n}(z',z)
\end{equation}
The full Green function can then be written as
\begin{align}
    G(\vb{x},\vb{x}')
    &=4\pi\qty(\frac2a)^2\sum_{m,n=1}^\infty
    \frac{\sinh(kz_<)e^{-kz_>}}{k}
    \sin\qty(\frac{m\pi x}{a})\sin\qty(\frac{m\pi x'}{a})
    \sin\qty(\frac{n\pi y}{a})\sin\qty(\frac{n\pi y'}{a})\notag\\
    &=\frac{16}{a}\sum_{m,n=1}^\infty\frac{\sinh(kz_<)e^{-kz_>}}{\sqrt{m^2+n^2}}
    \sin\qty(\frac{m\pi x}{a})\sin\qty(\frac{m\pi x'}{a})
    \sin\qty(\frac{n\pi y}{a})\sin\qty(\frac{n\pi y'}{a})
\end{align}

(b) From (1.44, Jackson), with $\Phi(\vb{x}')=V$ on the surface $S$ at $z'=0$
with the normal vector $\nhat=-\zhat$, $z_< =z'$, $z_> =z$ and the potential is
\begin{align}
    \Phi(\vb{x})
    &=\frac1{4\pi}\oint_S\Phi(\vb{x}')
    \eval{\frac{\partial G}{\partial z'}}_{z'=0}da'\notag\\
    &=V\qty(\frac2a)^2\sum_{m,n=1}^\infty e^{-kz}
    \sin\qty(\frac{m\pi x}{a})\sin\qty(\frac{n\pi y}{a})
    \qty[\int_0^a\sin\qty(\frac{m\pi x'}{a})dx']
    \qty[\int_0^a\sin\qty(\frac{n\pi y'}{a})dy']
    \notag\\
    &=\frac{4V}{\pi^2}\sum_{m,n=1}^\infty
    \frac{\qty[1-(-1)^m][1-(-1)^n]}{mn}\sin\qty(\frac{m\pi x}{a})
    \sin\qty(\frac{n\pi y}{a})\exp\qty(-\frac{\pi\sqrt{m^2+n^2}z}{a})
\end{align}
where we have used Mathematica to evaluate the integrals. At large $z$, the
potential decays exponentially and tends to zero, as expected from the boundary
condition.
\end{solution}
\end{problem}
%%%%%%%%%%%%%%%%%%%%%%%%%%%%%%%%%%%%%%%%%%%%%%%%%%%%%%%%%%%%%%%%%%%%%%%%%%%%%%%%
\end{document}
