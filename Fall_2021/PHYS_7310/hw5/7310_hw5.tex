\documentclass[12pt]{article}

%%%%%%%%%%%%%%%%%%%%%%%%%%%%%%%%%%%%%%%%%%%%%%%%%%%%%%%%%%%%%%%%%%%%%%%%%%%%%%%%
%                           Package preset for homework
%%%%%%%%%%%%%%%%%%%%%%%%%%%%%%%%%%%%%%%%%%%%%%%%%%%%%%%%%%%%%%%%%%%%%%%%%%%%%%%%
% Miscellaneous
\usepackage[margin=1in]{geometry}
\usepackage[utf8]{inputenc}
\usepackage{indentfirst}
\usepackage{blindtext}
\usepackage{graphicx}
\usepackage{xr-hyper}
\usepackage{hyperref}
\usepackage{enumitem}
\usepackage{color}
\usepackage{float}
% Math
\usepackage{latexsym}
\usepackage{amsfonts}
\usepackage{amssymb}
\usepackage{amsmath}
\usepackage{commath}
\usepackage{amsthm}
\usepackage{bbold}
\usepackage{bm}
% Physics
\usepackage{physics}
\usepackage{siunitx}
% Code typesetting
\usepackage{listings}
% Citation
\usepackage[authoryear]{natbib}
\usepackage{appendix}
\usepackage[capitalize]{cleveref}
% Title & name
\title{Homework}
\author{Tien Vo}
\date{\today}


%%%%%%%%%%%%%%%%%%%%%%%%%%%%%%%%%%%%%%%%%%%%%%%%%%%%%%%%%%%%%%%%%%%%%%%%%%%%%%%%
%                   User-defined commands and environments
%%%%%%%%%%%%%%%%%%%%%%%%%%%%%%%%%%%%%%%%%%%%%%%%%%%%%%%%%%%%%%%%%%%%%%%%%%%%%%%%
%%% Misc
\sisetup{load-configurations=abbreviations}
\newcommand{\due}[1]{\date{Due: #1}}
\newcommand{\hint}{\textit{Hint}}
\let\oldt\t
\renewcommand{\t}[1]{\text{#1}}

%%% Bold sets & abbrv
\newcommand{\N}{\mathbb{N}}
\newcommand{\Z}{\mathbb{Z}}
\newcommand{\R}{\mathbb{R}}
\newcommand{\Q}{\mathbb{Q}}
\let\oldP\P
\renewcommand{\P}{\mathbb{P}}
\newcommand{\LL}{\mathcal{L}}
\newcommand{\FF}{\mathcal{F}}
\newcommand{\HH}{\mathcal{H}}
\newcommand{\NN}{\mathcal{N}}
\newcommand{\ZZ}{\mathcal{Z}}
\newcommand{\RN}[1]{\textup{\uppercase\expandafter{\romannumeral#1}}}
\newcommand{\ua}{\uparrow}
\newcommand{\da}{\downarrow}

%%% Unit vectors
\newcommand{\xhat}{\vb{\hat{x}}}
\newcommand{\yhat}{\vb{\hat{y}}}
\newcommand{\zhat}{\vb{\hat{z}}}
\newcommand{\nhat}{\vb{\hat{n}}}
\newcommand{\rhat}{\vb{\hat{r}}}
\newcommand{\phihat}{\bm{\hat{\phi}}}
\newcommand{\thetahat}{\bm{\hat{\theta}}}

%%% Other math stuff
\providecommand{\units}[1]{\,\ensuremath{\mathrm{#1}}\xspace}
% Set new style for problem
\newtheoremstyle{problemstyle}  % <name>
        {10pt}                   % <space above>
        {10pt}                   % <space below>
        {\normalfont}           % <body font>
        {}                      % <indent amount}
        {\bfseries\itshape}     % <theorem head font>
        {\normalfont\bfseries:} % <punctuation after theorem head>
        {.5em}                  % <space after theorem head>
        {}                      % <theorem head spec (can be left empty, 
                                % meaning `normal')>

% Set problem environment
\theoremstyle{problemstyle}
\newtheorem{problemenv}{Problem}[section]
\newenvironment{problem}[1]{%
  \renewcommand\theproblemenv{#1}%
  \problemenv
}{\endproblemenv}
% Set lemma environment
\newenvironment{lemma}[2][Lemma]{\begin{trivlist}
\item[\hskip \labelsep {\bfseries #1}\hskip \labelsep {\bfseries #2.}]}{\end{trivlist}}
% Set solution environment
\newenvironment{solution}{
    \begin{proof}[Solution]$ $\par\nobreak\ignorespaces
}{\end{proof}}
\numberwithin{equation}{problemenv}

%%% Page format
\setlength{\parindent}{0.5cm}
\setlength{\oddsidemargin}{0in}
\setlength{\textwidth}{6.5in}
\setlength{\textheight}{8.8in}
\setlength{\topmargin}{0in}
\setlength{\headheight}{18pt}

%%% Code environments
\definecolor{dkgreen}{rgb}{0,0.6,0}
\definecolor{gray}{rgb}{0.5,0.5,0.5}
\definecolor{mauve}{rgb}{0.58,0,0.82}
\lstset{frame=tb,
  language=Python,
  aboveskip=3mm,
  belowskip=3mm,
  showstringspaces=false,
  columns=flexible,
  basicstyle={\small\ttfamily},
  numbers=none,
  numberstyle=\tiny\color{gray},
  keywordstyle=\color{blue},
  commentstyle=\color{dkgreen},
  stringstyle=\color{mauve},
  breaklines=true,
  breakatwhitespace=true,
  tabsize=4
}
\lstset{
  language=Mathematica,
  numbers=left,
  numberstyle=\tiny\color{gray},
  numbersep=5pt,
  breaklines=true,
  captionpos={t},
  frame={lines},
  rulecolor=\color{black},
  framerule=0.5pt,
  columns=flexible,
  tabsize=2
}


\title{Homework 5: Phys 7310 (Fall 2021)}

\begin{document}
\maketitle 

%%%%%%%%%%%%%%%%%%%%%%%%%%%%%%%%%%%%%%%%%%%%%%%%%%%%%%%%%%%%%%%%%%%%%%%%%%%%%%%%
\begin{problem}{5.1}[Spherical harmonic addition theorem]
(a) A unit vector in spherical coordinates may be written as
\begin{equation}
    \nhat\equiv\qty(n_x,n_y,n_z)=\qty(\sin\theta\cos\phi,\sin\theta\sin\phi,\cos\theta) 
\end{equation}
For vectors $\vb{x}$ and $\vb{x}'$ with associated unit vectors $\nhat$ and
$\nhat'$, show that
\begin{equation}
    \cos\gamma=\cos\theta\cos\theta'+\sin\theta\sin\theta'\cos(\phi-\phi') 
\end{equation}
where $\gamma$ is the angle between $\vb{x}$ and $\vb{x}'$. Now show that the
$l=1$ case of the spherical harmonic addition theorem (Jackson 3.62) reduces to
the same relation.

(b) Using the explicit forms of the spherical harmonics given on Jackson page
109, verify the ``sum rule" (Jackson 3.69) that is a special case of the addition
theorem, for $l=2$ and $l=3$.
\begin{solution}
    (a) Given $\nhat=\vb{x}/\abs{\vb{x}}$ and $\nhat'=\vb{x}'/\abs{\vb{x}'}$, 
    we can calculate
    \begin{align}\label{p1a:cos}
    \cos\gamma
    &=\nhat\vdot\nhat'\notag\\
    &=\sin\theta\cos\varphi\sin\theta'\cos\varphi'
    +\sin\theta\sin\varphi\sin\theta'\sin\varphi'
    +\cos\theta\cos\varphi'\notag\\
    &=\sin\theta\sin\theta'\qty(\cos\varphi\cos\varphi'+\sin\varphi\sin\varphi')
    +\cos\theta\cos\theta'\notag\\
    &=\sin\theta\sin\theta'\cos(\varphi-\varphi')+\cos\theta\cos\theta'
\end{align}
Now, from (Jackson 3.62), $P_1(\cos\gamma)=\cos\gamma$ and
\begin{align}
    \cos\gamma
    &=\frac{4\pi}{3}\sum_{m=-1}^1\frac{3}{4\pi}
    \frac{(1-m)!}{(1+m)!}P_1^m(\cos\theta)P_1^m(\cos\theta')e^{-im\varphi'}e^{im\varphi}\notag\\
    &=\frac12e^{-i(\varphi-\varphi)'}\sin\theta\sin\theta'
    +\cos\theta\cos\theta'+\frac12e^{i(\varphi-\varphi')}\sin\theta\sin\theta'\notag\\
    &=\cos\theta\cos\theta'+\sin\theta\sin\theta'\cos(\varphi-\varphi')
\end{align}
We get the same result as \eqref{p1a:cos}.

(b) For $l=2$, the sum rule is
\begin{align}
    \frac{5}{4\pi}
    &=\sum_{m=-2}^2\abs{Y_{2m}(\Omega)}^2\notag\\
    &=\abs{Y_{20}(\Omega)}^2+2\sum_{m=1}^2\abs{Y_{2m}(\Omega)}^2
        \tag{by symmetry}\\
    &=\frac5{4\pi}\qty(\frac32\cos^2\theta-\frac12)^2
    +\frac{15}{4\pi}\sin^2\theta\cos^2\theta+\frac{15}{16\pi}\sin^4\theta\notag\\
    &=\frac{5}{4\pi}\frac{9\cos^4\theta-6\cos^2\theta+1+12\sin^2\theta\cos^2\theta+3\sin^4\theta}{4}\notag\\
    &=\frac{5}{4\pi}
\end{align}
where we have used Mathematica to simplify in the last equality. For $l=3$,
\begin{align}
    \frac{7}{4\pi}
    &=\abs{Y_{30}(\Omega)}^2+2\sum_{m=1}^3\abs{Y_{3m}(\Omega)}^2\notag\\
    &=\frac{7}{4\pi}\qty(\frac52\cos^3\theta-\frac32\cos\theta)^2
    +\frac{21}{32\pi}\sin^2\theta(5\cos^2\theta-1)^2
    +\frac{105}{16\pi}\sin^4\theta\cos^2\theta+\frac{35}{32\pi}\sin^6\theta\notag\\
    &=\frac{7}{4\pi}\Bigg[\frac{25}{4}\cos^9\theta
        -\frac{15}{2}\cos^4\theta+\frac94\cos^2\theta+\frac{75}{8}\sin^2\theta\cos^4\theta-\frac{15}{4}\sin^2\theta\cos^2\theta\notag\\
    &\qquad+\frac38\sin^2\theta+\frac{15}{4}\sin^4\theta\cos^2\theta+\frac58\sin^6\theta\Bigg]\notag\\
    &=\frac{7}{4\pi}
\end{align}
\end{solution}
\end{problem}
%%%%%%%%%%%%%%%%%%%%%%%%%%%%%%%%%%%%%%%%%%%%%%%%%%%%%%%%%%%%%%%%%%%%%%%%%%%%%%%%
%%%%%%%%%%%%%%%%%%%%%%%%%%%%%%%%%%%%%%%%%%%%%%%%%%%%%%%%%%%%%%%%%%%%%%%%%%%%%%%%
\begin{problem}{5.2}[A dipole in spherical harmonics]
Two point charges $q$ and $-q$ are located on the $z$ axis at $z=a$ and $z=-a$,
respectively.

(a) Find the electrostatic potential as an expansion in spherical harmonics and
powers of $r$ for both $r>a$ and $r<a$.

(b) Keeping the product $qa\equiv p /2$ constant, take the limit of $a\to 0$ and
find the potential for $r\neq 0$. This is by definition a dipole along the $z$
axis and its potential.

(c) Suppose now that the dipole of part (b) is surrounded by a \textit{grounded}
spherical shell of radius $b$ concentric with the origin. By linear
superposition find the potential everywhere inside the shell.
\begin{solution}
(a) The potential due to the two point charges is
\begin{align}\label{p2a:Phi1}
    \Phi&=\frac{q}{4\pi\epsilon_0}\qty[\frac1{\abs{\vb{x}-a\zhat}}-\frac1{\abs{\vb{x}+a\zhat}}]\notag\\
    &=\frac{q}{\epsilon_0}\sum_{l=0}^\infty\sum_{m=-l}^l
    \frac1{2l+1}\frac{r_<^l}{r_>^{l+1}}\qty[
    Y_{lm}^\star(\theta'=0,\phi')-Y_{lm}^\star(\theta'=\pi,\phi')
    ]Y_{lm}(\Omega)
\end{align}
Now, note that by definition,
\begin{align}
    Y_{lm}^\star(\theta'=0,\phi')-Y_{lm}^\star(\theta'=\pi,\phi')
    &=\sqrt{\frac{2l+1}{4\pi}\frac{(l-m)!}{(l+m)!}}\qty[1-(-1)^{l+m}]P_l^m(1)e^{-im\phi'}
\end{align}
But $P_l^m(1)$ is only non-zero for $m=1$. Thus, the finite sum over $m$ in
\eqref{p2a:Phi1} reduces to only one term with
\begin{equation}
    Y_{l0}^\star(\theta'=0,\phi')-Y_{l0}^\star(\theta'=\pi,\phi') 
    =\sqrt{\frac{2l+1}{4\pi}}\qty[1-(-1)^l]P_l(1)
\end{equation}
Then we can write \eqref{p2a:Phi1} as
\begin{align}\label{p2a:Phi2}
    \Phi&=\frac{q}{4\pi\epsilon_0}\sum_{l=0}^\infty
        \frac{r_<^l}{r_>^{l+1}}\qty[1-(-1)^l]P_l(\cos\theta)\notag\\
    &=\begin{cases}
    (q/4\pi\epsilon_0)(1/a)\sum_{l=0}^\infty\qty(r/a)^l
    \qty[1-(-1)^l]P_l(\cos\theta) & r<a\\
    (q/4\pi\epsilon_0)(1/r)\sum_{l=0}^\infty\qty(a/r)^l
    \qty[1-(-1)^l]P_l(\cos\theta) & r>a
    \end{cases}
\end{align}
(b) From \eqref{p2a:Phi2} for $r>a$, we can write
\begin{align}
    \Phi_{r>a}&=\frac{p}{8\pi\epsilon_0}\sum_{l=0}^\infty 
    \frac{a^{l-1}}{r^{l+1}}\qty[1-(-1)^l]P_l(\cos\theta)\tag{$p=2qa$}\\
    &=\frac{p}{8\pi\epsilon_0}\qty[
        \frac{2}{r^2}P_1(\cos\theta)+\frac{2a^3}{r^4}P_3(\cos\theta)+\hdots]\notag\\
    &\approx\frac{p\cos\theta}{4\pi\epsilon_0r^2}\notag\\
    &=\frac{\vb{p}\vdot\rhat}{4\pi\epsilon_0r^2}
\end{align}
(c) The dipole induces some charge on the shell. Thus, the total potential is
\begin{equation}
    \Phi=\frac{p\cos\theta}{4\pi\epsilon_0r^2}+\sum_{l=0}^\infty
    A_lr^lP_l(\cos\theta) 
\end{equation}
where we have set $B_l=0$ in the general solution in spherical coordinates
because $r^{-(l+1)}\to\infty$ as $r\to 0$. At $r=b$, $\Phi=0$, so we can write
\begin{equation}
    \sum_{l=0}^\infty
    A_lb^lP_l(\cos\theta)=-\frac{p\cos\theta}{4\pi\epsilon_0r^2} 
\end{equation}
By orthogonality, the coefficients $A_l$ are 
\begin{equation}
    A_l=-\frac{2l+1}{2}b^{-l}\frac{p}{4\pi\epsilon_0b^2}
    \int_{-1}^1d(\cos\theta)P_l(\cos\theta)P_1(\cos\theta)
    =-\frac{p}{4\pi\epsilon_0 b^{2+l}}\delta_{1l}
\end{equation}
The only non-trivial coefficient is thus $A_1=-p /4\pi\epsilon_0b^3$. Then the
potential can be written as
\begin{equation}
    \Phi=\frac{p\cos\theta}{4\pi\epsilon_0b^2}\qty[\qty(\frac{b}{r})^2-\frac{r}{b}] 
\end{equation}
\end{solution}
\end{problem}
%%%%%%%%%%%%%%%%%%%%%%%%%%%%%%%%%%%%%%%%%%%%%%%%%%%%%%%%%%%%%%%%%%%%%%%%%%%%%%%%
%%%%%%%%%%%%%%%%%%%%%%%%%%%%%%%%%%%%%%%%%%%%%%%%%%%%%%%%%%%%%%%%%%%%%%%%%%%%%%%%
\begin{problem}{5.3}[A disk inside a sphere]
Consider a disk of radius $R$ centered on the origin, with uniform surface
charge density $\sigma$. The disk sits inside a grounded, conducting sphere of
radius $b$, also centered on the origin. Use a suitable limit of the Green's
function in Jackson 3.125 to find the potential everywhere inside the sphere, as
an expansion in Legendre polynomials. (This is similar to the example in class
and Jackson 3.10, except with a disk instead of a ring. Fill in the steps of how
the spherical harmonics reduce to Legendre polynomials, and treat the cases
$r<R$ and $R<r<b$ separately when you do the radial integral. You can use the
results for $P_l(0)$ found above Jackson 3.131).
\begin{solution}
First, note that the total charge can be written as
\begin{equation}
    Q=\int\sigma r'dr'd\varphi'=\int \rho
    r'^2dr'd(\cos\theta')d\varphi' 
\end{equation}
Thus, we can write the volume charge density as
\begin{equation}
    \rho(\vb{x}')=\sigma\frac{\delta(\cos\theta')}{r'} 
\end{equation}
Now, from (3.125), the Green's function inside the sphere with $a\to 0$ is
\begin{equation}
    G(\vb{x},\vb{x}')=4\pi\sum_{l=0}^\infty\sum_{m=-l}^l\frac{Y_{lm}^\star(\Omega')Y_{lm}(\Omega)}{2l+1}r_<^l\qty[\frac1{r_>^{l+1}}-\frac{r_>^l}{b^{2l+1}}]
\end{equation}
Then the potential can be written as
\begin{equation}\label{p3:Phi1}
    \Phi=\frac{\sigma}{\epsilon_0}\sum_{l=0}^\infty\sum_{m=-l}^l
    \frac1{2l+1}\int_0^{2\pi}d\varphi'Y_{lm}^\star(\theta'=\pi/2,\varphi')Y_{lm}(\theta,\varphi)\int_0^Rdr'r'r_<^l\qty[\frac1{r_>^{l+1}}-\frac{r_>^l}{b^{2l+1}}]
\end{equation}
Note that the domain for $r'$ is $[0,R]$ because $\sigma=0$ for $r'>R$. Now, for
$m\neq 0$, the integration over $\varphi'$ is
\begin{equation}
    \int_0^{2\pi}d\varphi' e^{-im\varphi'}=-\frac{i}{m}\qty(1-e^{-2im\pi})=0
\end{equation}
and for $m=0$, $e^{-im\varphi'}=1$ and the integration is
\begin{equation}\label{p3:int_phi}
    \int_0^{2\pi}d\varphi'=2\pi 
\end{equation}
Thus, the only non-trivial term in the summation $\sum_{m=-l}^l$ is $m=0$. The
function $Y_{l0}^\star(\Omega')Y_{l0}(\Omega)$ reduces to
\begin{equation}\label{p3:Y}
    Y_{l0}^\star(\theta'=\pi/2,\varphi')Y_{l0}(\theta,\varphi)
    =\frac{2l+1}{4\pi}P_l(0)P_l(\cos\theta)
\end{equation}
Thus, the potential \eqref{p3:Phi1} can be written with \eqref{p3:int_phi} and
\eqref{p3:Y} as
\begin{align}
    \Phi
    &=\frac{\sigma}{2\epsilon_0}\sum_{l=0}^\infty P_l(0)P_{l}(\cos\theta)
    \int_0^Rdr' r'r_<^l\qty[\frac1{r_>^{l+1}}-\frac{r_>^l}{b^{2l+1}}]\notag\\
    &=\frac{\sigma R}{2\epsilon_0}\sum_{l=0}^\infty P_{2l}(0)P_{2l}(\cos\theta)
    \int_0^Rdr'\frac{r'}{R}r_<^l\qty[\frac1{r_>^{l+1}}-\frac{r_>^l}{b^{2l+1}}]
\end{align}
where the odd terms $P_{2l+1}(0)$ all vanish. Now, we consider two cases:

\qquad \textbf{Case 1}: $r>R\Rightarrow r_< =\min(r,r')=r'$ and $r_>
=\max(r,r')=r$

The integration over $r'$ is thus
\begin{equation}
    I=\frac1R\qty[\frac1{r^{2l+1}}-\frac{r^{2l}}{b^{4l+1}}]\int_0^Rdr' r'^{2l+1}
    =\frac1{2l+2}\qty(\frac{R}{r})^{2l+1}\qty[1-\qty(\frac{r}{b})^{4l+1}]
\end{equation}
Then we can write the potential as
\begin{align}\label{p3:Phi_out}
    \Phi_{r>R}&=\frac{\sigma
    R}{2\epsilon_0}\sum_{l=0}^\infty\frac{P_{2l}(0)P_{2l}(\cos\theta)}{2l+2}\qty(\frac{R}{r})^{2l+1}\qty[1-\qty(\frac{r}{b})^{4l+1}]\notag\\
    &=\frac{\sigma R}{2\epsilon_0}\sum_{l=0}^\infty
        \frac{(-1)^l(2l-1)!!}{(2l+2)2^ll!}\qty(\frac{R}{r})^{2l+1}
        \qty[1-\qty(\frac{r}{b})^{4l+1}]P_{2l}(\cos\theta)
\end{align}

\qquad \textbf{Case 2}: $r<R$.

We have to split the integral $\int_0^R=\int_0^r+\int_r^R$ where $r_< =r'$ and
$r_> =r$ in the former and $r_< =r$ and $r_> =r'$ in the latter.
\begin{align}
    I
    &=\frac1R\qty[\frac1{r^{2l+1}}-\frac{r^{2l}}{b^{4l+1}}]\int_0^rdr' r'^{2l+1}
    +\frac{r^{2l}}{R}\int_r^R
    dr'\qty[\frac1{r'^{2l}}-\frac{r'^{2l+1}}{b^{4l+1}}]\notag\\
    &=\frac1{(2l+2)}\frac{r}{R}\qty[1-\qty(\frac{r}{b})^{4l+1}]
    -\frac{r^{2l}}{(2l-1)R}\qty[\frac1{R^{2l-1}}-\frac1{r^{2l-1}}]
    -\frac{r^{2l}}{(2l+2)Rb^{4l+1}}\qty(R^{2l+2}-r^{2l+2})\notag\\
    &=\frac{1}{(2l+2)}\frac{r}{R}\qty[1-\frac{r^{2l-1}R^{2l+2}}{b^{4l+1}}]
    +\frac1{2l-1}\frac{r}{R}\qty[1-\qty(\frac{r}{R})^{2l-1}]
\end{align}
Then the potential is
\begin{align}\label{p3:Phi_in}
    \Phi_{r<R}&=\frac{\sigma R}{2\epsilon_0}\sum_{l=0}^\infty
        \frac{P_{2l}(0)P_{2l}(\cos\theta)}{2l+2}\qty(\frac{r}{R})
            \qty[1-\frac{r^{2l-1}R^{2l+2}}{b^{4l+1}}+\frac{2l+2}{2l-1}\qty[1-\qty(\frac{r}{R})^{2l-1}]]\notag\\
    &=\frac{\sigma R}{2\epsilon_0}\sum_{l=0}^\infty
    \frac{(-1)^l(2l-1)!!}{(2l+2)2^ll!}\qty(\frac{r}{R})
    \qty[1-\frac{r^{2l-1}R^{2l+2}}{b^{4l+1}}+\frac{2l+2}{2l-1}\qty[1-\qty(\frac{r}{R})^{2l-1}]]P_{2l}(\cos\theta)
\end{align}

Note that \eqref{p3:Phi_out} vanishes at $r=b$ and at $r=R$, the potential
$\Phi$ is continuous
\begin{equation}
    \eval{\Phi_{r>R}}_{r=R}=\eval{\Phi_{r<R}}_{r=R}
    =\frac{\sigma R}{2\epsilon_0}\sum_{l=0}^\infty
    \frac{(-1)^l(2l-1)!!}{(2l+2)2^ll!}\qty[1-\qty(\frac{R}{b})^{4l+1}]
\end{equation}
Thus, this solution satisfies all boundary and continuity conditions.
\end{solution}
\end{problem}
%%%%%%%%%%%%%%%%%%%%%%%%%%%%%%%%%%%%%%%%%%%%%%%%%%%%%%%%%%%%%%%%%%%%%%%%%%%%%%%%
\end{document}
