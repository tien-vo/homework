\documentclass[12pt]{article}

%%%%%%%%%%%%%%%%%%%%%%%%%%%%%%%%%%%%%%%%%%%%%%%%%%%%%%%%%%%%%%%%%%%%%%%%%%%%%%%%
%                           Package preset for homework
%%%%%%%%%%%%%%%%%%%%%%%%%%%%%%%%%%%%%%%%%%%%%%%%%%%%%%%%%%%%%%%%%%%%%%%%%%%%%%%%
% Miscellaneous
\usepackage[margin=1in]{geometry}
\usepackage[utf8]{inputenc}
\usepackage{indentfirst}
\usepackage{blindtext}
\usepackage{graphicx}
\usepackage{xr-hyper}
\usepackage{hyperref}
\usepackage{enumitem}
\usepackage{color}
\usepackage{float}
% Math
\usepackage{latexsym}
\usepackage{amsfonts}
\usepackage{amssymb}
\usepackage{amsmath}
\usepackage{commath}
\usepackage{amsthm}
\usepackage{bbold}
\usepackage{bm}
% Physics
\usepackage{physics}
\usepackage{siunitx}
% Code typesetting
\usepackage{listings}
% Citation
\usepackage[authoryear]{natbib}
\usepackage{appendix}
\usepackage[capitalize]{cleveref}
% Title & name
\title{Homework}
\author{Tien Vo}
\date{\today}


%%%%%%%%%%%%%%%%%%%%%%%%%%%%%%%%%%%%%%%%%%%%%%%%%%%%%%%%%%%%%%%%%%%%%%%%%%%%%%%%
%                   User-defined commands and environments
%%%%%%%%%%%%%%%%%%%%%%%%%%%%%%%%%%%%%%%%%%%%%%%%%%%%%%%%%%%%%%%%%%%%%%%%%%%%%%%%
%%% Misc
\sisetup{load-configurations=abbreviations}
\newcommand{\due}[1]{\date{Due: #1}}
\newcommand{\hint}{\textit{Hint}}
\let\oldt\t
\renewcommand{\t}[1]{\text{#1}}

%%% Bold sets & abbrv
\newcommand{\N}{\mathbb{N}}
\newcommand{\Z}{\mathbb{Z}}
\newcommand{\R}{\mathbb{R}}
\newcommand{\Q}{\mathbb{Q}}
\let\oldP\P
\renewcommand{\P}{\mathbb{P}}
\newcommand{\LL}{\mathcal{L}}
\newcommand{\FF}{\mathcal{F}}
\newcommand{\HH}{\mathcal{H}}
\newcommand{\NN}{\mathcal{N}}
\newcommand{\ZZ}{\mathcal{Z}}
\newcommand{\RN}[1]{\textup{\uppercase\expandafter{\romannumeral#1}}}
\newcommand{\ua}{\uparrow}
\newcommand{\da}{\downarrow}

%%% Unit vectors
\newcommand{\xhat}{\vb{\hat{x}}}
\newcommand{\yhat}{\vb{\hat{y}}}
\newcommand{\zhat}{\vb{\hat{z}}}
\newcommand{\nhat}{\vb{\hat{n}}}
\newcommand{\rhat}{\vb{\hat{r}}}
\newcommand{\phihat}{\bm{\hat{\phi}}}
\newcommand{\thetahat}{\bm{\hat{\theta}}}

%%% Other math stuff
\providecommand{\units}[1]{\,\ensuremath{\mathrm{#1}}\xspace}
% Set new style for problem
\newtheoremstyle{problemstyle}  % <name>
        {10pt}                   % <space above>
        {10pt}                   % <space below>
        {\normalfont}           % <body font>
        {}                      % <indent amount}
        {\bfseries\itshape}     % <theorem head font>
        {\normalfont\bfseries:} % <punctuation after theorem head>
        {.5em}                  % <space after theorem head>
        {}                      % <theorem head spec (can be left empty, 
                                % meaning `normal')>

% Set problem environment
\theoremstyle{problemstyle}
\newtheorem{problemenv}{Problem}[section]
\newenvironment{problem}[1]{%
  \renewcommand\theproblemenv{#1}%
  \problemenv
}{\endproblemenv}
% Set lemma environment
\newenvironment{lemma}[2][Lemma]{\begin{trivlist}
\item[\hskip \labelsep {\bfseries #1}\hskip \labelsep {\bfseries #2.}]}{\end{trivlist}}
% Set solution environment
\newenvironment{solution}{
    \begin{proof}[Solution]$ $\par\nobreak\ignorespaces
}{\end{proof}}
\numberwithin{equation}{problemenv}

%%% Page format
\setlength{\parindent}{0.5cm}
\setlength{\oddsidemargin}{0in}
\setlength{\textwidth}{6.5in}
\setlength{\textheight}{8.8in}
\setlength{\topmargin}{0in}
\setlength{\headheight}{18pt}

%%% Code environments
\definecolor{dkgreen}{rgb}{0,0.6,0}
\definecolor{gray}{rgb}{0.5,0.5,0.5}
\definecolor{mauve}{rgb}{0.58,0,0.82}
\lstset{frame=tb,
  language=Python,
  aboveskip=3mm,
  belowskip=3mm,
  showstringspaces=false,
  columns=flexible,
  basicstyle={\small\ttfamily},
  numbers=none,
  numberstyle=\tiny\color{gray},
  keywordstyle=\color{blue},
  commentstyle=\color{dkgreen},
  stringstyle=\color{mauve},
  breaklines=true,
  breakatwhitespace=true,
  tabsize=4
}
\lstset{
  language=Mathematica,
  numbers=left,
  numberstyle=\tiny\color{gray},
  numbersep=5pt,
  breaklines=true,
  captionpos={t},
  frame={lines},
  rulecolor=\color{black},
  framerule=0.5pt,
  columns=flexible,
  tabsize=2
}

\newcommand{\ret}[1]{\qty[#1]_\text{ret}}
\newcommand{\Rhat}{\bm{\hat{R}}}

\title{Homework 10: Phys 7310 (Fall 2021)}

\begin{document}
\maketitle
%%%%%%%%%%%%%%%%%%%%%%%%%%%%%%%%%%%%%%%%%%%%%%%%%%%%%%%%%%%%%%%%%%%%%%%%%%%%%%%%
\begin{problem}{10.1}[Solutions for $\vb{E}$ and $\vb{B}$ in electrodynamics]
The potentials $\Phi$ and $\vb{A}$ are often easier to deal with, but it is
possible to work with the electric and magnetic fields $\vb{E}$ and $\vb{B}$
directly. In particular, they also obey wave equations with velocity parameter
$c$.

(a) Starting from Maxwell's equations in vacuum,
\begin{align}
    \div{\vb{E}}&=\frac{\rho}{\epsilon_0}, &
    \div{\vb{B}}&=0,\notag\\
    \curl{\vb{E}}+\frac{\partial\vb{B}}{\partial t}&=0 &
    \curl{\vb{B}}-\frac1{c^2}\frac{\partial\vb{E}}{\partial t}&=\mu_0\vb{J}
    +\frac1{c^2}\frac{\partial\vb{E}}{\partial t}
\end{align}
show that $\vb{E}$ and $\vb{B}$ obey wave equations with sources as given in
Jackon (6.49)--(6.50), and therefore have the solutions (6.51)--(6.52). (Thiscan
also be obtained starting with the solutions for the potentials (6.48), but I
want you to derive it without the potentials.)

(b) The solutions (6.51)--(6.52) are correct, but it is useful to recast them in
a form that more obviously reduces to the static forms. Fill in the steps to
derive (6.53) and (6.54), where we recall $\qty[f(\vb{x}',t')]_\text{ret}\equiv
f(\vb{x}',t-R /c)$.

(c) Use the results of part (b) to derive the solutions (6.55) and (6.56) for
$\vb{E}$ and $\vb{B}$. These are general solutions for the electric and magnetic
fields produced by fixed sources $\rho(\vb{x}',t')$ and $\vb{J}(\vb{x}',t')$ in
the fully dynamic case. Show that in the static limit these reduce to Coulomb's
Law (1.5) and the Biot-Savart Law (5.14).

(d) Comment on the $R$-dependence of the new terms that only appear in the
time-dependent case; do these fall off faster or more slowly than the static
fields? These are \textit{radiation fields}.
\begin{solution}
(a) Taking the curl of Ampere's Law and using the identity
$\curl{\vb{\curl\vb{E}}}=\grad\qty(\div{\vb{E}})-\laplacian\vb{E}$, we can
write
\begin{align}
    &&\frac1{\epsilon_0}\grad\rho-\laplacian\vb{E}
    &=-\frac{\partial}{\partial
    t}\qty(\mu_0\vb{J}+\frac1{c^2}\frac{\partial\vb{E}}{\partial t})
    =-\mu_0\frac{\partial\vb{J}}{\partial
    t}-\frac1{c^2}\frac{\partial^2\vb{E}}{\partial t^2}\notag\\
    &\Rightarrow&
    \laplacian\vb{E}-\frac1{c^2}\frac{\partial^2\vb{E}}{\partial t^2}
    &=-\frac1{\epsilon_0}\qty(-\grad\rho-\frac1{c^2}\frac{\partial\vb{J}}{\partial
    t})
\end{align}
Similarly, taking the curl of Faraday's Law, we get
$\curl{\curl{\vb{B}}}=-\laplacian\vb{B}$ because $\div{\vb{B}}=0$. Thus
\begin{equation}
    -\laplacian\vb{B}=\mu_0\curl\vb{J}-\frac1{c^2}\frac{\partial^2\vb{B}}{\partial
    t^2}\Rightarrow
    \laplacian\vb{B}-\frac1{c^2}\frac{\partial^2\vb{B}}{\partial t^2}
    =-\mu_0\curl{\vb{J}}
\end{equation}

(b) First, when taking the gradient of $\ret\rho=\ret\rho(\vb{x}',t-R /c)$, we
have to use the chain rule
\begin{align}
    \grad'\ret\rho
    &=\frac{\partial\ret\rho}{\partial\vb{x}'}+\frac{\partial\ret\rho}{\partial
    t'}\frac{\partial t'}{\partial \vb{x}'}\notag\\
    &=\ret{\grad'\rho}+\ret{\frac{\partial\rho}{\partial
    t'}}\grad'\qty(t-R/c)\notag\\
    &=\ret{\grad'\rho}-\frac1c\ret{\frac{\partial\rho}{\partial
    t'}}\grad'R\notag\\
    &=\ret{\grad'\rho}+\frac{\Rhat}{c}\ret{\frac{\partial\rho}{\partial t'}}
\end{align}
Thus, we arrive at (6.53, Jackson)
\begin{equation}
    \ret{\grad'\rho}=\grad'\ret{\rho}-\frac{\Rhat}{c}\ret{\frac{\partial\rho}{\partial
    t'}} 
\end{equation}

Now, to derive (6.54, Jackson), we write the cross product in index notation
\begin{align}
    \grad'\times\ret{\vb{J}}
    &=\epsilon_{ijk}\vb{\hat{e}}_i\partial_j J_k(\vb{x}',t-R/c)\notag\\
    &=\epsilon_{ijk}\vb{\hat{e}}_i\qty[\partial_jJ_k(\vb{x}',t')+\frac{\partial
    J_k}{\partial t'}\partial_j(t-R/c)]\notag\\
    &=\ret{\grad'\vb{J}}+\epsilon_{ijk}\vb{\hat{e}}_i\partial_jt'\frac{\partial
    J_k}{\partial t'}\notag\\
    &=\ret{\grad'\vb{J}}+\grad'(t-R/c)\times\ret{\frac{\partial\vb{J}}{\partial
    t'}}\notag\\
    &=\ret{\grad'\times\vb{J}}+\frac{\Rhat}{c}\times\ret{\frac{\partial\vb{J}}{\partial
    t'}}
\end{align}
Thus,
\begin{equation}
    \ret{\grad'\times\vb{J}}=\grad'\times\ret{\vb{J}}
    +\frac1c\ret{\frac{\partial\vb{J}}{\partial t'}}\times\Rhat
\end{equation}

(c) From (6.51, Jackson),
\begin{align}
    \vb{E}
    &=\frac1{4\pi\epsilon_0}\int
    d^3x'\frac1R\ret{-\grad'\rho-\frac1{c^2}\frac{\partial\vb{J}}{\partial
    t'}}\notag\\
    &=\frac1{4\pi\epsilon_0}\int d^3x'\frac1R\qty{
        -\grad'\ret\rho+\frac{\Rhat}{c}\ret{\frac{\partial\rho
        }{\partial t'}}-\frac1{c^2}\ret{\frac{\partial\vb{J}}{\partial t'}}
    }\notag\\
    &=\frac1{4\pi\epsilon_0}\qty{
        -\int d^3x'\frac1R\grad'\ret\rho+\int d^3x'
        \qty{\frac{\Rhat}{cR}\ret{\frac{\partial\rho}{\partial
        t'}}-\frac1{c^2R}\ret{\frac{\partial\vb{J}}{\partial t'}}}
    }
\end{align}
Integrating the first term by parts, we get
\begin{equation}
    \int d^3x'\frac1R\grad'\ret\rho
    =\int d^3x'\qty[\grad'\qty(\frac{\ret\rho}{R})-\ret\rho\grad'\qty(\frac1R)]
    =\oint\frac{\ret\rho}{R}d\vb{a}'
    -\int d^3x'\ret\rho\frac{\Rhat}{R^2}
\end{equation}
The boundary term (surface integral) vanishes because $\ret\rho$ vanishes at
$\infty$. So we can write the electric field as
\begin{align}
    \vb{E}=\frac1{4\pi\epsilon_0}\int d^3x'\qty{
        \frac{\ret\rho}{R^2}\Rhat+\frac{\Rhat}{cR}\ret{\frac{\partial\rho}{\partial
        t'}}
        -\frac1{c^2R}\ret{\frac{\partial\vb{J}}{\partial t'}}
    } 
\end{align}

Similarly, we can find the magnetic field by integrating by parts. From (6.52,
Jackson),
\begin{align}
    \vb{B}
    &=\frac{\mu_0}{4\pi}\int d^3x'\frac1R\ret{\grad'\times\vb{J}}\notag\\
    &=\frac{\mu_0}{4\pi}\int d^3x'\frac1R\qty(
    \grad'\times\ret{\vb{J}}+\frac1c\ret{\frac{\partial\vb{J}}{\partial
    t'}}\times\Rhat
    )\notag\\
    &=\frac{\mu_0}{4\pi}\qty{
        \int
        d^3x'\qty[\grad'\times\qty(\frac{\ret{\vb{J}}}{R})-\grad'\qty(\frac1R)\times\ret{\vb{J}}]
        +\int d^3x'\ret{\frac{\partial\vb{J}}{\partial t'}}\times\frac{\Rhat}{cR}
    }\notag\\
    &=\frac{\mu_0}{4\pi}\int d^3x'\qty{-\frac{\Rhat}{R}\times\ret{\vb{J}}
    +\ret{\frac{\partial\vb{J}}{\partial t'}}\times\frac{\Rhat}{cR}}
\end{align}
\end{solution}
\end{problem}
%%%%%%%%%%%%%%%%%%%%%%%%%%%%%%%%%%%%%%%%%%%%%%%%%%%%%%%%%%%%%%%%%%%%%%%%%%%%%%%%    
%%%%%%%%%%%%%%%%%%%%%%%%%%%%%%%%%%%%%%%%%%%%%%%%%%%%%%%%%%%%%%%%%%%%%%%%%%%%%%%%
\begin{problem}{10.2}[Causality in Coulomb gauge]
In the Coulomb gauge, $\Phi(\vb{x},t)$ is instantaneous (that is, it is
determined by the behavior of $\rho(\vb{x},t)$ at the same time), but
$\vb{A}(\vb{x},t)$ is causal (it is the solution of the wave equation).
Causality of $\vb{B}$ follows from causality of $\vb{A}$ (you do not need to
show this). Starting from equations for $\Phi$ and $\vb{A}$ in Coulomb gauge,
show that $\vb{E}$ is also causal. What is the source of $\vb{E}$? (Hint: it
should be the same as what you got in the previous problem.)
\begin{solution}
Starting from the LHS of the wave equation for $\vb{A}$ (6.24, Jackson) in 
Coulomb gauge, we can use the definition of $\vb{E}$ and write
\begin{align}
    \laplacian{\vb{A}}
    -\frac1{c^2}\frac{\partial^2\vb{A}}{\partial t^2}
    &=\laplacian\qty(-\grad\Phi-\vb{E})
-\frac1{c^2}\frac{\partial^2}{\partial t^2}\qty(-\grad\Phi-\vb{E})\notag\\
    &=-\grad\qty(\frac{\rho}{\epsilon_0})-\laplacian\vb{E}
    +\frac1{c^2}\grad\frac{\partial^2\Phi}{\partial t^2}
    +\frac1{c^2}\frac{\partial^2\vb{E}}{\partial t^2}
\end{align}
Equating this to the RHS of (6.24, Jackson), we can write
\begin{equation}
    \laplacian\vb{E}-\frac1{c^2}\frac{\partial^2\vb{E}}{\partial t^2}
    =\frac1{\epsilon_0}\grad\rho+\mu_0\frac{\partial\vb{J}}{\partial t}
\end{equation}
So $\vb{E}$ also follows the wave equation of the general form (6.32, Jackson),
meaning it is causal. The RHS determines the source of $\vb{E}$ where the source
distribution $f(\vb{x},t)$ is
\begin{equation} 
    f(\vb{x},t)=-\frac1{4\pi\epsilon_0}\grad\rho-\frac{\mu_0}{4\pi}\frac{\partial\vb{J}}{\partial
    t}
\end{equation}
\end{solution}
\end{problem}
%%%%%%%%%%%%%%%%%%%%%%%%%%%%%%%%%%%%%%%%%%%%%%%%%%%%%%%%%%%%%%%%%%%%%%%%%%%%%%%%
%%%%%%%%%%%%%%%%%%%%%%%%%%%%%%%%%%%%%%%%%%%%%%%%%%%%%%%%%%%%%%%%%%%%%%%%%%%%%%%%
\begin{problem}{10.3}[A conducting shell and the stress tensor]
A (perfectly) conducting spherical shell of radius $a$ is placed in a uniform
electric field $\vb{E}_0$. Find the force tending to separate the two halves of
the sphere across a diametrical plane perpendicular to $\vb{E}_0$ in two ways:

(a) Using the stress tensor.

(b) By integrating the appropriate projection of $\sigma^2 /2\epsilon_0$ over a
hemisphere.

You may use old results for a conducting sphere in a uniform electric field.
\begin{solution}
(a) From (2.15, Jackson), the electric field strength on the surface of the 
conductor is $\abs{\vb{E}}=3E_0\cos\theta$. So by definition, the stress tensor
is
\begin{equation}
    T_{\alpha\beta}=\epsilon_0\qty[E_\alpha E_\beta-\frac12E^2\delta_{\alpha\beta}] 
    =\frac12\epsilon_0E^2\mqty(-1&0&0\\0&-1&0\\0&0&1)
    =\frac92\epsilon_0E_0^2\cos^2\theta\mqty(-1&0&0\\0&-1&0\\0&0&1)
\end{equation}
Then by definition, the force on one hemisphere of the shell is
\begin{align}
    \vb{F}
    &=\int \vb{T}\vdot\rhat da\notag\\
    &=\frac92\epsilon_0E_0^2\int\mqty(-1&0&0\\0&-1&0\\0&0&1)
    \mqty(\sin\theta\cos\phi\\\sin\theta\sin\phi\\\cos\theta)\cos^2\theta
    a^2\sin\theta d\theta d\phi\notag\\
    &=\frac92\epsilon_0E_0^2a^2\int_0^{\pi/2}\sin\theta\cos^3\theta
    d\theta\int_0^{2\pi}d\phi\zhat\notag\\
    &=\frac94\pi a^2\epsilon_0E_0^2\zhat
\end{align}

(b) By symmetry, the force has to be in the $z$ direction. So using (2.15,
Jackson),
\begin{align}
    F_z=\int\frac{\sigma^2}{2\epsilon_0}\rhat\vdot\zhat da 
    =\frac92\epsilon_0E_0^2a^2\int_0^{\pi/2}\sin\theta\cos^3\theta
    d\theta\int_0^{2\pi}d\phi=\frac94\pi a^2\epsilon_0E_0^2
\end{align}
This is the same result as (a).
\end{solution}
\end{problem}
%%%%%%%%%%%%%%%%%%%%%%%%%%%%%%%%%%%%%%%%%%%%%%%%%%%%%%%%%%%%%%%%%%%%%%%%%%%%%%%%
\end{document}
