\documentclass[12pt]{article}

%%%%%%%%%%%%%%%%%%%%%%%%%%%%%%%%%%%%%%%%%%%%%%%%%%%%%%%%%%%%%%%%%%%%%%%%%%%%%%%%
%                           Package preset for homework
%%%%%%%%%%%%%%%%%%%%%%%%%%%%%%%%%%%%%%%%%%%%%%%%%%%%%%%%%%%%%%%%%%%%%%%%%%%%%%%%
% Miscellaneous
\usepackage[margin=1in]{geometry}
\usepackage[utf8]{inputenc}
\usepackage{indentfirst}
\usepackage{blindtext}
\usepackage{graphicx}
\usepackage{xr-hyper}
\usepackage{hyperref}
\usepackage{color}
\usepackage{float}
% Math
\usepackage{latexsym}
\usepackage{amsfonts}
\usepackage{amssymb}
\usepackage{amsmath}
\usepackage{commath}
\usepackage{amsthm}
\usepackage{bbold}
\usepackage{bm}
% Physics
\usepackage{physics}
\usepackage{siunitx}
% Code typesetting
\usepackage{listings}
% Citation
\usepackage[authoryear]{natbib}
\usepackage{appendix}
\usepackage[capitalize]{cleveref}
% Title & name
\title{Homework}
\author{Tien Vo}
\date{\today}


%%%%%%%%%%%%%%%%%%%%%%%%%%%%%%%%%%%%%%%%%%%%%%%%%%%%%%%%%%%%%%%%%%%%%%%%%%%%%%%%
%                   User-defined commands and environments
%%%%%%%%%%%%%%%%%%%%%%%%%%%%%%%%%%%%%%%%%%%%%%%%%%%%%%%%%%%%%%%%%%%%%%%%%%%%%%%%
%%% Misc
\sisetup{load-configurations=abbreviations}
\newcommand{\due}[1]{\date{Due: #1}}
\newcommand{\hint}{\textit{Hint}}
\let\oldt\t
\renewcommand{\t}[1]{\text{#1}}

%%% Bold sets & abbrv
\newcommand{\N}{\mathbb{N}}
\newcommand{\Z}{\mathbb{Z}}
\newcommand{\R}{\mathbb{R}}
\newcommand{\Q}{\mathbb{Q}}
\let\oldP\P
\renewcommand{\P}{\mathbb{P}}
\newcommand{\LL}{\mathcal{L}}
\newcommand{\FF}{\mathcal{F}}
\newcommand{\HH}{\mathcal{H}}
\newcommand{\NN}{\mathcal{N}}
\newcommand{\ZZ}{\mathcal{Z}}
\newcommand{\RN}[1]{\textup{\uppercase\expandafter{\romannumeral#1}}}
\newcommand{\ua}{\uparrow}
\newcommand{\da}{\downarrow}

%%% Unit vectors
\newcommand{\xhat}{\vb{\hat{x}}}
\newcommand{\yhat}{\vb{\hat{y}}}
\newcommand{\zhat}{\vb{\hat{z}}}
\newcommand{\nhat}{\vb{\hat{n}}}
\newcommand{\rhat}{\vb{\hat{r}}}
\newcommand{\phihat}{\bm{\hat{\phi}}}
\newcommand{\thetahat}{\bm{\hat{\theta}}}

%%% Other math stuff
\providecommand{\units}[1]{\,\ensuremath{\mathrm{#1}}\xspace}
% Set new style for problem
\newtheoremstyle{problemstyle}  % <name>
        {10pt}                   % <space above>
        {10pt}                   % <space below>
        {\normalfont}           % <body font>
        {}                      % <indent amount}
        {\bfseries\itshape}     % <theorem head font>
        {\normalfont\bfseries:} % <punctuation after theorem head>
        {.5em}                  % <space after theorem head>
        {}                      % <theorem head spec (can be left empty, 
                                % meaning `normal')>

% Set problem environment
\theoremstyle{problemstyle}
\newtheorem{problemenv}{Problem}[section]
\newenvironment{problem}[1]{%
  \renewcommand\theproblemenv{#1}%
  \problemenv
}{\endproblemenv}
% Set lemma environment
\newenvironment{lemma}[2][Lemma]{\begin{trivlist}
\item[\hskip \labelsep {\bfseries #1}\hskip \labelsep {\bfseries #2.}]}{\end{trivlist}}
% Set solution environment
\newenvironment{solution}{
    \begin{proof}[Solution]$ $\par\nobreak\ignorespaces
}{\end{proof}}
\numberwithin{equation}{problemenv}

%%% Page format
\setlength{\parindent}{0.5cm}
\setlength{\oddsidemargin}{0in}
\setlength{\textwidth}{6.5in}
\setlength{\textheight}{8.8in}
\setlength{\topmargin}{0in}
\setlength{\headheight}{18pt}

%%% Code environments
\definecolor{dkgreen}{rgb}{0,0.6,0}
\definecolor{gray}{rgb}{0.5,0.5,0.5}
\definecolor{mauve}{rgb}{0.58,0,0.82}
\lstset{frame=tb,
  language=Python,
  aboveskip=3mm,
  belowskip=3mm,
  showstringspaces=false,
  columns=flexible,
  basicstyle={\small\ttfamily},
  numbers=none,
  numberstyle=\tiny\color{gray},
  keywordstyle=\color{blue},
  commentstyle=\color{dkgreen},
  stringstyle=\color{mauve},
  breaklines=true,
  breakatwhitespace=true,
  tabsize=4
}
\lstset{
  language=Mathematica,
  numbers=left,
  numberstyle=\tiny\color{gray},
  numbersep=5pt,
  breaklines=true,
  captionpos={t},
  frame={lines},
  rulecolor=\color{black},
  framerule=0.5pt,
  columns=flexible,
  tabsize=2
}


\title{Final: Phys 7310 (Fall 2021)}

\begin{document}
\maketitle
%%%%%%%%%%%%%%%%%%%%%%%%%%%%%%%%%%%%%%%%%%%%%%%%%%%%%%%%%%%%%%%%%%%%%%%%%%%%%%%
\begin{problem}{F.1}[Potentials and gauges]
Consider two vector potentials
\begin{equation}
    \vb{A}(\vb{x})=Bx\yhat,\qquad \vb{A}'(\vb{x})=\frac{B}2\qty(x\yhat-y\xhat)
\end{equation}
where $B$ is a constant. The scalar potential vanishes.

(a) Show that $\vb{A}$ and $\vb{A}'$ lead to the same magnetic field, a uniform
field in the $z$ direction. Find a gauge transformation $\Lambda(\vb{x})$ that
relates them.

(b) Which of $\vb{A}$ and $\vb{A}'$ (or both, or neither) are in the Coulomb
gauge? The answer places a constraint on $\Lambda$; show that $\Lambda$ indeed
satisfies this contraint.

(c) Starting with $\vb{A}$, consider a new gauge transformation by the function
\begin{equation}
    \Lambda(\vb{x},t)=xe^{2t} 
\end{equation}
Is the electric field now zero or nonzero? Demonstrate your answer explicitly.
\begin{solution}
(a) First, the magnetic field corresponding to $\vb{A}$ is
\begin{equation}
    \vb{B}=\curl{\vb{A}}=\frac{\partial A}{\partial x}\zhat=B\zhat 
\end{equation}
and that corresponding to $\vb{A}'$ is
\begin{equation}
    \vb{B}'=\curl{\vb{A}'}=\qty(\frac{\partial A_y}{\partial x}-\frac{\partial
    A_x}{\partial y})\zhat=\qty(\frac{B}2+\frac{B}2)\zhat=B\zhat
\end{equation}
Thus, $\vb{B}'=\vb{B}$. Now, we want to find $\Lambda$ such that
\begin{equation}
    \grad\Lambda=\vb{A}'-\vb{A}=-\frac{B}2\qty(x\yhat+y\xhat)
\end{equation}
Thus,
\begin{equation}
    \frac{\partial\Lambda}{\partial x}=-\frac{By}{2}\qquad\text{and}\qquad
    \frac{\partial\Lambda}{\partial y}=-\frac{Bx}{2}
\end{equation}
One solution is clearly $\Lambda=-(B/2)xy$.

(b) From (6.21, Jackson), Coulomb gauge is that in which $\div{\vb{A}}=0$.
Thus, considering that both $\div{\vb{A}}=\div{\vb{A}'}=0$, both of them are in
the Coulomb gauge. Then this places a contraint on $\Lambda$ through the Lorenz
condition (6.17, Jackson)
\begin{equation}
    \laplacian\Lambda=\frac1c^2\frac{\partial^2\Lambda}{\partial t^2}=0 
\end{equation}
Since $\Lambda$ is only up to the first order in $x$ and $y$, the Laplacian is
indeed
\begin{equation}
    \laplacian\Lambda=\frac{\partial^2\Lambda}{\partial
    x^2}+\frac{\partial^2\Lambda}{\partial y^2}
    \sim\frac{\partial y}{\partial x}+\frac{\partial x}{\partial y}=0
\end{equation}

(c) The transformed potentials are
\begin{equation}
    \vb{A}'=\vb{A}+\grad\Lambda=Bx\yhat+e^{2t}\xhat
    \qquad\text{and}\qquad
    \Phi'=-\frac{\partial\Lambda}{\partial t}=-2xe^{2t}
\end{equation}
Then the electric field is
\begin{equation}
    \vb{E}'=-\grad\Phi'-\frac{\partial\vb{A}'}{\partial
    t}=2e^{2t}\xhat-2e^{2t}\xhat=\vb{0}
\end{equation}
\end{solution}
\end{problem}
%%%%%%%%%%%%%%%%%%%%%%%%%%%%%%%%%%%%%%%%%%%%%%%%%%%%%%%%%%%%%%%%%%%%%%%%%%%%%%%    
%%%%%%%%%%%%%%%%%%%%%%%%%%%%%%%%%%%%%%%%%%%%%%%%%%%%%%%%%%%%%%%%%%%%%%%%%%%%%%%
\begin{problem}{F.2}[Fields in a circular capacitor]
A parallel plate capacitor is composed of two circular plates of radius $R$, a
distance $d$ apart. Starting at $t=0$, the capacitor is slowly charged by a
constant current $I$. The space between the plate is empty. You may assume $R\gg
d$ and ignore edge effects.

(a) Calculate the magnitude and direction of the Poynting vector at the boundary
of the region between the plates ($\rho=R,0<z<d$) and use this to find the total
power flowing into this region.

(b) Calculate the electrostatic energy stored in the capacitor by integrating
the electrostatic energy density. Determine its change with time and compare
this to the result for part (a).
\begin{solution}
(a) Ignoring edge effects, we assume the plates are charged with a uniform 
charge density $\sigma=Q/\pi R^2$. Then the electric field inside the capacitor
is uniform
\begin{equation}\label{p2a:E}
    \vb{E}=\frac{\sigma}{\epsilon_0}\nhat=\frac{Q}{\pi\epsilon_0R^2}\nhat
    =\frac{It}{\pi\epsilon_0R^2}\nhat
\end{equation}
where $\nhat=\zhat$ is a vector normal to the plates. Now, we can
draw a circular surface $S$ with radius $\rho$ inside the capacitor such that
$\vb{B}$ is constant on the boundary $\partial S$. Then by Ampere's Law,
\begin{equation}\label{p2a:B}
    \oint_{\partial S}\vb{B}\vdot d\vb{l}=B2\pi
    \rho=\frac1c^2\oint_S\frac{\partial\vb{E}}{\partial t}\nhat
    da=\frac{\mu_0I}{\pi R^2} \pi \rho^2\Rightarrow
    \vb{B}=\frac{\mu_0I}{2\pi R}\frac{\rho}{R}\phihat
\end{equation}
by symmetry. So the Poynting vector inside the capacitor is
\begin{equation}
    \vb{S}=\frac{\vb{E}\times\vb{B}}{\mu_0}=\frac{It}{\epsilon_0\pi
    R^2}\frac{Ir}{2\pi R^2}\zhat\times\phihat
    =-\frac{I^2}{2\pi^2\epsilon_0}\frac{rt}{R^4}\rhat
\end{equation}
This points radially inward. So choosing a normal vector $\nhat=-\rhat$, we can
calculate the total power flowing into the capacitor at $\rho=R$
\begin{equation}
    P_\text{in}=\oint\vb{S}\vdot\nhat da
    =\frac{I^2}{2\pi^2\epsilon_0}\frac{t}{R^3}(2\pi
    Rd)=\frac{I^2dt}{\pi\epsilon_0R^2}
\end{equation}

(b) Using the electric and magnetic fields \eqref{p2a:E}, \eqref{p2a:B}, the  
energy density is
\begin{equation}
    u=\frac12\qty(\epsilon_0E^2+\frac{B^2}{\mu_0}) 
    =\frac12\qty(\frac{I^2t^2}{\pi^2\epsilon_0R^4}+\frac{\mu_0I^2\rho^2}{4\pi^2R^4})
    =\frac{\mu_0I^2}{8\pi^2R^4}\qty(4c^2t^2+\rho^2)
\end{equation}
Then the total energy is obtained by integrating over the volume of the
capacitor
\begin{align}
    U&=\int
    ud^3x=\frac{\mu_0I^2}{8\pi^2R^4}\int_0^R\int_0^{2\pi}\int_0^d
    \qty(4c^2t^2+\rho^2)\rho d\rho d\phi dz\notag\\
     &=\frac{\mu_0I^2d}{4\pi R^4}\qty(\frac{R^4}{4}+2c^2R^2t^2)\tag{by
     Mathematica}\\
     &=\frac{\mu_0I^2d}{4\pi R^2}\qty(\frac{R^2}{4}+2c^2t^2)
\end{align}
So the total energy increases quadratically with time. The rate of charging is
\begin{equation}
    \frac{dU}{dt}=\frac{I^2dt}{\pi\epsilon_0 R^2} 
\end{equation}
which is the same as the power $P_\text{in}$ flowing into the capacitor in part 
(a).
\end{solution}
\end{problem}
%%%%%%%%%%%%%%%%%%%%%%%%%%%%%%%%%%%%%%%%%%%%%%%%%%%%%%%%%%%%%%%%%%%%%%%%%%%%%%%
%%%%%%%%%%%%%%%%%%%%%%%%%%%%%%%%%%%%%%%%%%%%%%%%%%%%%%%%%%%%%%%%%%%%%%%%%%%%%%%
\begin{problem}{F.3}[Hard ferromagnetic shell]
A spherical shell of hard ferromagnet is centered on the origin between $r=a$
and $r=b$. The shell has uniform magnetization of magnitude $M_0$ in the $z$
direction. Inside and outside the shell is empty space.

(a) Find an expression for the magnetic scalar potential $\Phi_M$ for all three
regions $r<a$, $a<r<b$, and $r>b$, and find the $r$ and $\theta$ components of
the vector fields $\vb{H}$ and $\vb{B}$ just for the region $a<r<b$.

(b) Show how the value of both components of $\vb{B}$ (not $\vb{H}$!) satisfy
the expected boundary conditions at $r=a$, relating the surface current that
appears to the magnetization.
\begin{solution}
(a) Since we already know the solution for a uniformly magnetized solid sphere 
of hard ferromagnet, we can use the principle of superposition here. From
Section 5.10 in Jackson, a sphere of radius $b$ and magnetization 
$\vb{M}_+=M_0\zhat$ has a potential
\begin{equation}
    \Phi_+(r,\theta)=\begin{cases}
        (1/3)M_0r\cos\theta & r<b\\
        (1/3)M_0(b^3/r^2)\cos\theta & r\geq b
    \end{cases}
\end{equation}
and a sphere of radius $a$ and magnetization $\vb{M}_-=-M_0\zhat$ has a 
potential
\begin{equation}
    \Phi_-(r,\theta)=\begin{cases}
        -(1/3)M_0r\cos\theta & r<a\\
        -(1/3)M_0(a^3/r^2)\cos\theta & r\geq a
    \end{cases}
\end{equation}
Then for $r<a$, the total magnetization is $\vb{M}=\vb{M}_++\vb{M}_-=\vb{0}$, as
expected of a hollow shell and the total potential is also
$\Phi_M=\Phi_++\Phi_-=0$. For $a<r<b$, the total magnetization is only
$\vb{M}=\vb{M}_+$ and the total potential is
\begin{equation}
    \Phi_M=\frac{M_0}{3}r\cos\theta-\frac{M_0}{3}\frac{a^3}{r^2}\cos\theta
    =\frac{M_0}{3}\qty(1-\frac{a^3}{r^3})r\cos\theta
\end{equation}
Similarly, for $r>b$, there is no magnetization and
\begin{equation}
    \Phi_M=\frac{M_0}{3}\qty(b^3-a^3)\frac{\cos\theta}{r^2} 
\end{equation}
In summary,
\begin{subequations}
    \begin{align}
        \Phi_M(r<a)&=0\\
        \Phi_M(a<r<b)&=\frac{M_0}{3}\qty(1-\frac{a^3}{r^3})r\cos\theta\\
        \Phi_M(r>b)&=\frac{M_0}{3}\frac{b^3-a^3}{r^2}\cos\theta
    \end{align} 
\end{subequations}
Then the field $\vb{H}$ for $a<r<b$ is
\begin{align}
    \vb{H}&=-\grad\Phi_M\notag\\
          &=-\frac{\partial\Phi_M}{\partial
          r}\rhat-\frac1r\frac{\partial\Phi_M}{\partial\theta}\thetahat\notag\\
          &=-\frac{M_0}{3}\qty(1+2\frac{a^3}{r^3})\cos\theta\rhat+\frac{M_0}{3}\qty(1-\frac{a^3}{r^3})\sin\theta\thetahat
\end{align}
and the magnetic field is
\begin{align} 
    \vb{B}&=\mu_0\qty(\vb{H}+\vb{M})\notag\\
          &=\frac{2\mu_0M_0}{3}\qty(1-\frac{a^3}{r^3})\cos\theta\rhat
          -\frac{\mu_0M_0}{3}\qty(2+\frac{a^3}{r^3})\sin\theta\thetahat
\end{align}
where we have used Mathematica to save the algebra steps.

(b) At $r=a$, the boundary condition is
\begin{equation}\label{p3b:bc}
    \vb{B}_{r\to a^+}-\vb{B}_{r\to a^-}=\mu_0\qty(\vb{K}_b\times\nhat)
    =\mu_0\qty(\vb{M}\times\rhat)\times\rhat
    =\mu_0M_0\sin\theta\thetahat
\end{equation}
where the normal vector points radially inward $\nhat=-\rhat$. There is no
magnetic field inside the shell because there's no current. So now we consider
the LHS
\begin{equation}
    -\vb{B}_{r\to a^-}=\mu_0M_0\sin\theta\thetahat
\end{equation}
which is the same as the RHS in \eqref{p3b:bc}. So this magnetic field (all
components) satisfies the boundary condition at $r=a$.
\end{solution}
\end{problem}
%%%%%%%%%%%%%%%%%%%%%%%%%%%%%%%%%%%%%%%%%%%%%%%%%%%%%%%%%%%%%%%%%%%%%%%%%%%%%%%
%%%%%%%%%%%%%%%%%%%%%%%%%%%%%%%%%%%%%%%%%%%%%%%%%%%%%%%%%%%%%%%%%%%%%%%%%%%%%%%
\begin{problem}{F.4}[Dielectric in a waveguide]
Consider a waveguide with square cross-section, of side length $a$. The
waveguide axis is in the $z$ direction and the sides of the square are oriented
along $x$ and $y$. Assume the walls are perfect conductors. For $z\geq 0$ the
waveguide is filled with dielectric with permittivity $\epsilon$, while for
$z<0$ the waveguide is empty. Take $\mu=\mu_0$ everywhere.

(a) Electromagnetic radiation in the lowest TE mode of frequency $\omega$ flows
down the waveguide in the positive $z$ direction. Use the boundary conditions
for the $\vb{E}$ and $\vb{H}$ fields at $z=0$ to obtain all the independent
constraints on the coefficients for the incident, reflected, and transmitted
waves.

(b) Calculate the reflection coefficient for this interface, with your answer
given in terms of $\omega,a,c$ and the index of refraction for the dielectric
$n$ (and pure numbers).
\begin{solution}
(a) Let medium 1 and 2 be the regions $z<0$ and $z\geq 0$, respectively.
By assumption, $n_1=1$ and $n_2=n$. Now, from (8.46, Jackson), the incident 
fields are
\begin{subequations}
    \begin{align}
        \vb{E}_I&=\frac{i\omega a\mu_0}{\pi}H_{0I}\sin\qty(\frac{\pi
        x}{a})e^{i(k_1z-\omega t)}\yhat\\
        \vb{H}_I&=\frac{k_1}{\mu_0\omega}\zhat\times\vb{E}_I+
        H_{Iz}\zhat=
        -\frac{ik_1a}{\pi}H_{0I}\sin\qty(\frac{\pi x}{a})e^{i(k_1z-\omega
        t)}\xhat+
        H_{0I}\cos\qty(\frac{\pi x}{a})e^{i(k_1z-\omega t)}\zhat
    \end{align} 
\end{subequations}
where $k_{1,2}=\sqrt{\mu_{1,2}\epsilon_{1,2}}\sqrt{\omega^2-\omega_{1,2}^2}$ as
defined in (8.39, Jackson), $\omega_{1,2}=\pi/(\sqrt{\mu_{1,2}\epsilon_{1,2}}a)$
are the lowest TE modes in medium 1 and 2. Similarly, the reflected fields are
\begin{subequations}
    \begin{align}
        \vb{E}_R&=\frac{i\omega a\mu_0}{\pi}H_{0R}\sin\qty(\frac{\pi
        x}{a})e^{i(-k_1z-\omega t)}\yhat\\
        \vb{H}_R&=-\frac{k_1}{\mu_0\omega}\zhat\times\vb{E}_R+
        H_{Rz}\zhat=
        \frac{ik_1a}{\pi}H_{0R}\sin\qty(\frac{\pi x}{a})e^{i(-k_1z-\omega
        t)}\xhat+
        H_{0R}\cos\qty(\frac{\pi x}{a})e^{i(-k_1z-\omega t)}\zhat
    \end{align} 
\end{subequations}
and the transmitted fields are
\begin{subequations}
    \begin{align}
        \vb{E}_T&=\frac{i\omega a\mu_0}{\pi}H_{0T}\sin\qty(\frac{\pi
        x}{a})e^{i(k_2z-\omega t)}\yhat\\
        \vb{H}_T&=\frac{k_2}{\mu_0\omega}\zhat\times\vb{E}_T+
        H_{Tz}\zhat=
        -\frac{ik_2a}{\pi}H_{0T}\sin\qty(\frac{\pi x}{a})e^{i(k_2z-\omega
        t)}\xhat+
        H_{0T}\cos\qty(\frac{\pi x}{a})e^{i(k_2z-\omega t)}\zhat
    \end{align} 
\end{subequations}

At $z=0$, we must require that the tangential $\vb{E}$ and $\vb{H}$ are
continuous. This means
\begin{equation}
    \frac{i\omega a\mu_0}{\pi}H_{0I}\sin\qty(\frac{\pi x}{a})e^{-i\omega t}
    +\frac{i\omega a\mu_0}{\pi}H_{0R}\sin\qty(\frac{\pi x}{a})e^{-i\omega t}
    =\frac{i\omega a}{\pi}H_{0T}\sin\qty(\frac{\pi x}{a})e^{-i\omega t}
\end{equation}
and
\begin{equation}
    -\frac{ik_1a}{\pi}H_{0I}\sin\qty(\frac{\pi x}{a})e^{-i\omega t}
    +\frac{ik_1a}{\pi}H_{0R}\sin\qty(\frac{\pi x}{a})e^{-i\omega t}
    =-\frac{ik_2a}{\pi}H_{0T}\sin\qty(\frac{\pi x}{a})e^{-i\omega t}
\end{equation}
Simplifying, we get $H_{0I}+H_{0R}=H_{0T}$ and
\begin{equation}
    H_{0I}-H_{0R}=\frac{k_2}{k_1}H_{0T}
    =\sqrt{\frac{\mu\epsilon}{\mu_0\epsilon_0}}\sqrt{\frac{\omega^2-\omega_2^2}{\omega^2-\omega_1^2}}H_{0T}
    =n\sqrt{\frac{\omega^2-c^2\pi^2/n^2a^2}{\omega^2-c^2\pi^2/a^2}}H_{0T}
    =n'H_{0T}
\end{equation}

(b) Using Mathematica to solve for the boundary conditions in part (a), we get
\begin{equation}
    \frac{H_{0R}}{H_{0I}}=\frac{1-n'}{1+n'}\qquad\text{and}\qquad
    \frac{H_{0T}}{H_{0I}}=\frac{2}{1+n'}
\end{equation}
Then the reflection coefficient is
\begin{align}
    R=\abs{\frac{E_{0R}}{E_{0I}}}^2= \abs{\frac{H_{0R}}{H_{0I}}}^2
    =\abs{\frac{1-n\sqrt{\frac{\omega^2-c^2\pi^2/n^2a^2}{\omega^2-c^2\pi^2/a^2}}}{1+n\sqrt{\frac{\omega^2-c^2\pi^2/n^2a^2}{\omega^2-c^2\pi^2/a^2}}}}^2
\end{align}
\end{solution}
\end{problem}
%%%%%%%%%%%%%%%%%%%%%%%%%%%%%%%%%%%%%%%%%%%%%%%%%%%%%%%%%%%%%%%%%%%%%%%%%%%%%%%
\end{document}
