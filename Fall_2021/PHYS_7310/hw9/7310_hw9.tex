\documentclass[12pt]{article}

%%%%%%%%%%%%%%%%%%%%%%%%%%%%%%%%%%%%%%%%%%%%%%%%%%%%%%%%%%%%%%%%%%%%%%%%%%%%%%%%
%                           Package preset for homework
%%%%%%%%%%%%%%%%%%%%%%%%%%%%%%%%%%%%%%%%%%%%%%%%%%%%%%%%%%%%%%%%%%%%%%%%%%%%%%%%
% Miscellaneous
\usepackage[margin=1in]{geometry}
\usepackage[utf8]{inputenc}
\usepackage{indentfirst}
\usepackage{blindtext}
\usepackage{graphicx}
\usepackage{xr-hyper}
\usepackage{hyperref}
\usepackage{enumitem}
\usepackage{color}
\usepackage{float}
% Math
\usepackage{latexsym}
\usepackage{amsfonts}
\usepackage{amssymb}
\usepackage{amsmath}
\usepackage{commath}
\usepackage{amsthm}
\usepackage{bbold}
\usepackage{bm}
% Physics
\usepackage{physics}
\usepackage{siunitx}
% Code typesetting
\usepackage{listings}
% Citation
\usepackage[authoryear]{natbib}
\usepackage{appendix}
\usepackage[capitalize]{cleveref}
% Title & name
\title{Homework}
\author{Tien Vo}
\date{\today}


%%%%%%%%%%%%%%%%%%%%%%%%%%%%%%%%%%%%%%%%%%%%%%%%%%%%%%%%%%%%%%%%%%%%%%%%%%%%%%%%
%                   User-defined commands and environments
%%%%%%%%%%%%%%%%%%%%%%%%%%%%%%%%%%%%%%%%%%%%%%%%%%%%%%%%%%%%%%%%%%%%%%%%%%%%%%%%
%%% Misc
\sisetup{load-configurations=abbreviations}
\newcommand{\due}[1]{\date{Due: #1}}
\newcommand{\hint}{\textit{Hint}}
\let\oldt\t
\renewcommand{\t}[1]{\text{#1}}

%%% Bold sets & abbrv
\newcommand{\N}{\mathbb{N}}
\newcommand{\Z}{\mathbb{Z}}
\newcommand{\R}{\mathbb{R}}
\newcommand{\Q}{\mathbb{Q}}
\let\oldP\P
\renewcommand{\P}{\mathbb{P}}
\newcommand{\LL}{\mathcal{L}}
\newcommand{\FF}{\mathcal{F}}
\newcommand{\HH}{\mathcal{H}}
\newcommand{\NN}{\mathcal{N}}
\newcommand{\ZZ}{\mathcal{Z}}
\newcommand{\RN}[1]{\textup{\uppercase\expandafter{\romannumeral#1}}}
\newcommand{\ua}{\uparrow}
\newcommand{\da}{\downarrow}

%%% Unit vectors
\newcommand{\xhat}{\vb{\hat{x}}}
\newcommand{\yhat}{\vb{\hat{y}}}
\newcommand{\zhat}{\vb{\hat{z}}}
\newcommand{\nhat}{\vb{\hat{n}}}
\newcommand{\rhat}{\vb{\hat{r}}}
\newcommand{\phihat}{\bm{\hat{\phi}}}
\newcommand{\thetahat}{\bm{\hat{\theta}}}

%%% Other math stuff
\providecommand{\units}[1]{\,\ensuremath{\mathrm{#1}}\xspace}
% Set new style for problem
\newtheoremstyle{problemstyle}  % <name>
        {10pt}                   % <space above>
        {10pt}                   % <space below>
        {\normalfont}           % <body font>
        {}                      % <indent amount}
        {\bfseries\itshape}     % <theorem head font>
        {\normalfont\bfseries:} % <punctuation after theorem head>
        {.5em}                  % <space after theorem head>
        {}                      % <theorem head spec (can be left empty, 
                                % meaning `normal')>

% Set problem environment
\theoremstyle{problemstyle}
\newtheorem{problemenv}{Problem}[section]
\newenvironment{problem}[1]{%
  \renewcommand\theproblemenv{#1}%
  \problemenv
}{\endproblemenv}
% Set lemma environment
\newenvironment{lemma}[2][Lemma]{\begin{trivlist}
\item[\hskip \labelsep {\bfseries #1}\hskip \labelsep {\bfseries #2.}]}{\end{trivlist}}
% Set solution environment
\newenvironment{solution}{
    \begin{proof}[Solution]$ $\par\nobreak\ignorespaces
}{\end{proof}}
\numberwithin{equation}{problemenv}

%%% Page format
\setlength{\parindent}{0.5cm}
\setlength{\oddsidemargin}{0in}
\setlength{\textwidth}{6.5in}
\setlength{\textheight}{8.8in}
\setlength{\topmargin}{0in}
\setlength{\headheight}{18pt}

%%% Code environments
\definecolor{dkgreen}{rgb}{0,0.6,0}
\definecolor{gray}{rgb}{0.5,0.5,0.5}
\definecolor{mauve}{rgb}{0.58,0,0.82}
\lstset{frame=tb,
  language=Python,
  aboveskip=3mm,
  belowskip=3mm,
  showstringspaces=false,
  columns=flexible,
  basicstyle={\small\ttfamily},
  numbers=none,
  numberstyle=\tiny\color{gray},
  keywordstyle=\color{blue},
  commentstyle=\color{dkgreen},
  stringstyle=\color{mauve},
  breaklines=true,
  breakatwhitespace=true,
  tabsize=4
}
\lstset{
  language=Mathematica,
  numbers=left,
  numberstyle=\tiny\color{gray},
  numbersep=5pt,
  breaklines=true,
  captionpos={t},
  frame={lines},
  rulecolor=\color{black},
  framerule=0.5pt,
  columns=flexible,
  tabsize=2
}


\title{Homework 9: Phys 7310 (Fall 2021)}

\begin{document}
\maketitle
%%%%%%%%%%%%%%%%%%%%%%%%%%%%%%%%%%%%%%%%%%%%%%%%%%%%%%%%%%%%%%%%%%%%%%%%%%%%%%%%
\begin{problem}{9.1}[Hard ferromagnetic cylinder]
A magnetically ``hard'' material is in the shape of a right circular cylinder of
length $L$ and radius $a$. The cylinder has a permanent magnetization $M_0$,
uniform throughout its volume and parallel to its axis. Determine the magnetic
field $\vb{H}$ and magnetic induction $\vb{B}$ at all points on the axis of the
cylinder, both inside and outside. Use the magnetic scalar potential $\Phi_M$.
In addition, find $H_z$ and $B_z$ at $z\sim 0$ where $0$ is the vertical middle
of the cylinder, for the limits $a\ll L$ and $a\gg L$.
\begin{solution}
From (5.100, Jackson),
\begin{equation}
    \Phi_M(z)=\frac1{4\pi}\oint_S\frac{\hat{\vb{n}}'\vdot\vb{M}(\vb{x}')da'}{\abs{z\zhat-\vb{x}'}}
\end{equation}
because $\grad'\vdot\vb{M}=0$ inside the cylinder. The surface integral is only
non-trivial on the caps $S_\pm$ at $z=\pm L /2$ because
$\hat{\vb{n}}'=\pm\zhat$. We can then evaluate for $\Phi_M$
\begin{align}
    \Phi_M(z)
    &=\frac{M_0}{4\pi}\qty[\int_0^a\frac{\rho'd\rho'}{\sqrt{\rho'^2+(z-L/2)^2}}-\int_0^a\frac{\rho'd\rho'}{\sqrt{\rho'^2+(z+L/2)^2}}]\int_0^{2\pi}
    d\phi\notag\\
    &=\frac{M_0}{2}\eval{\qty[\sqrt{\rho'^2+(z-L/2)^2}-\sqrt{\rho'^2+(z+L/2)^2}]}_0^a\notag\\
    &=\frac{M_0}{2}\qty[\sqrt{a^2+\qty(z-\frac{L}{2})^2}-\sqrt{a^2+\qty(z+\frac{L}{2})^2}+\abs{z+\frac{L}{2}}-\abs{z-\frac{L}{2}}]
\end{align}
Thus, the magnetic field $H_z$ is
\begin{equation}\label{p1:H}
    H_z=-\frac{\partial\Phi_m}{\partial z}
    =-\frac{M_0}{2}\qty[\frac{z-L/2}{\sqrt{a^2+(z-L/2)^2}}
    -\frac{z+L/2}{\sqrt{a^2+(z+L/2)^2}}+\frac{z+L/2}{\abs{z+L/2}}-\frac{z-L/2}{\abs{z-L/2}}]
\end{equation}
everywhere along the axis. The magnetic induction inside is
\begin{align}
    B_{\abs{z}\leq L/2}
    &=\mu_0H_z+\mu_0M_0\notag\\
    &=-\frac{\mu_0M_0}{2}\qty[\frac{z-L/2}{\sqrt{a^2+(z-L/2)^2}}
    -\frac{z+L/2}{\sqrt{a^2+(z+L/2)^2}}+2]+\mu_0M_0\notag\\
    &=-\frac{\mu_0M_0}{2}\qty[\frac{z-L/2}{\sqrt{a^2+(z-L/2)^2}}-\frac{z+L/2}{\sqrt{a^2+(z+L/2)^2}}]
\end{align}
and the magnetic induction outside is
\begin{align}\label{p1:B}
    B_{\abs{z}>L/2}
    &=\mu_0H_z\notag\\
    &=-\frac{\mu_0M_0}{2}\qty[\frac{z-L/2}{\sqrt{a^2+(z-L/2)^2}}-\frac{z+L/2}{\sqrt{a^2+(z+L/2)^2}}]
\end{align}
So the magnetic induction also has the same form everywhere along $z$.

From \eqref{p1:H} and \eqref{p1:B}, at $z=0$, the magnetic field and magnetic
induction are
\begin{subequations}
    \begin{align}
        B_z&=\mu_0M_0\frac{L}{\sqrt{4a^2+L^2}}\\
        H_z&=M_0\qty[\frac{L}{\sqrt{4a^2+L^2}}-1]
    \end{align} 
\end{subequations}
Then for $a\ll L$,
\begin{subequations}
    \begin{align}
        B_z&=\mu_0M_0\frac{1}{\sqrt{1+4(a/L)^2}}\approx\mu_0M_0\qty[1-2\frac{a^2}{L^2}]\\
        H_z&=M_0\qty[\frac{1}{\sqrt{1+4(a/L)^2}}-1]
        \approx-2M_0\frac{a^2}{L^2}
    \end{align} 
\end{subequations}
Similarly, for $L\ll a$,
\begin{subequations}
    \begin{align}
        B_z&=\mu_0M_0\frac{L/a}{\sqrt{4+(L/a)^2}}\approx\frac{\mu_0M_0}{2}\frac{L}{a}\\
        H_z&=M_0\qty[\frac{L/a}{\sqrt{4+(L/a)^2}}-1]
        \approx -M_0\qty[1-\frac{L}{2a}]
    \end{align} 
\end{subequations}
\end{solution}
\end{problem}
%%%%%%%%%%%%%%%%%%%%%%%%%%%%%%%%%%%%%%%%%%%%%%%%%%%%%%%%%%%%%%%%%%%%%%%%%%%%%%%%    
%%%%%%%%%%%%%%%%%%%%%%%%%%%%%%%%%%%%%%%%%%%%%%%%%%%%%%%%%%%%%%%%%%%%%%%%%%%%%%%%
\begin{problem}{9.2}[Self-inductance]
A circuit consists of a long thin conducting shell of radius $a$ and a parallel
return wire of radius $b$ on axis inside. If the current is assumed distributed
uniformly throughout the cross section of the wire, calculate the
self-inductance per unit length. What is the self-inductance if the inner
conductor is a thin hollow tube?
\begin{solution}
Draw an Amperian circular loop perpendicular to the axis of the wire with a
radius $r<b$. Then the enclosed current is
\begin{equation}
    I_\text{enc}=\oint_S\vb{J}\vdot d\vb{a}=
    4\pi r^2J=I\frac{r^2}{b^2}
\end{equation}
where $I$ is the total current running through the wire. Then by Ampere's Law,
the magnetic field is
\begin{equation}
    \vb{H}_{r<b}=\frac{I_\text{enc}}{2\pi r}\phihat
    =\frac{I}{2\pi b}\frac{r}{b}\phihat
\end{equation}
The magnetic induction is thus
\begin{equation}
    \vb{B}_{r<b}=\mu\vb{H}=\frac{\mu I}{2\pi b}\frac{r}{b}\phihat
\end{equation}
For $b\leq r\leq a$, the enclosed current is $I$ and
\begin{equation}
    \vb{H}_{b\leq r\leq a}=\frac{I}{2\pi r}\phihat
    \qquad\text{and}\qquad
    \vb{B}_{b\leq r\leq a}=\frac{\mu I}{2\pi r}\phihat
\end{equation}
By definition, the self-inductance is
\begin{align}
    L
    =\frac1{I^2}\int\frac{B^2}{\mu}d^3x
    =\frac{\mu}{2\pi}\qty[\frac1{b^2}\int_0^brdr+\int_b^a\frac{dr}{r}]
    =\frac{\mu}{2\pi}\qty[\frac12+\ln\frac{a}{b}]
\end{align}

If the wire is hollow, then there is no magnetic field inside the wire and the
self-inductance is just
\begin{equation}
    L_\text{hollow}=\frac{\mu}{2\pi}\ln\frac{a}{b} 
\end{equation}
\end{solution}
\end{problem}
%%%%%%%%%%%%%%%%%%%%%%%%%%%%%%%%%%%%%%%%%%%%%%%%%%%%%%%%%%%%%%%%%%%%%%%%%%%%%%%%
\end{document}
