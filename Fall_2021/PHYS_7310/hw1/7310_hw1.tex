\documentclass[12pt]{article}


%%%%%%%%%%%%%%%%%%%%%%%%%%%%%%%%%%%%%%%%%%%%%%%%%%%%%%%%%%%%%%%%%%%%%%%%%%%%%%%%
%                           Package preset for homework
%%%%%%%%%%%%%%%%%%%%%%%%%%%%%%%%%%%%%%%%%%%%%%%%%%%%%%%%%%%%%%%%%%%%%%%%%%%%%%%%
% Miscellaneous
\usepackage[margin=1in]{geometry}
\usepackage[utf8]{inputenc}
\usepackage{indentfirst}
\usepackage{blindtext}
\usepackage{graphicx}
\usepackage{xr-hyper}
\usepackage{hyperref}
\usepackage{enumitem}
\usepackage{color}
\usepackage{float}
% Math
\usepackage{latexsym}
\usepackage{amsfonts}
\usepackage{amssymb}
\usepackage{amsmath}
\usepackage{commath}
\usepackage{amsthm}
\usepackage{bbold}
\usepackage{bm}
% Physics
\usepackage{physics}
\usepackage{siunitx}
% Code typesetting
\usepackage{listings}
% Citation
\usepackage[authoryear]{natbib}
\usepackage{appendix}
\usepackage[capitalize]{cleveref}
% Title & name
\title{Homework}
\author{Tien Vo}
\date{\today}


%%%%%%%%%%%%%%%%%%%%%%%%%%%%%%%%%%%%%%%%%%%%%%%%%%%%%%%%%%%%%%%%%%%%%%%%%%%%%%%%
%                   User-defined commands and environments
%%%%%%%%%%%%%%%%%%%%%%%%%%%%%%%%%%%%%%%%%%%%%%%%%%%%%%%%%%%%%%%%%%%%%%%%%%%%%%%%
%%% Misc
\sisetup{load-configurations=abbreviations}
\newcommand{\due}[1]{\date{Due: #1}}
\newcommand{\hint}{\textit{Hint}}
\let\oldt\t
\renewcommand{\t}[1]{\text{#1}}

%%% Bold sets & abbrv
\newcommand{\N}{\mathbb{N}}
\newcommand{\Z}{\mathbb{Z}}
\newcommand{\R}{\mathbb{R}}
\newcommand{\Q}{\mathbb{Q}}
\let\oldP\P
\renewcommand{\P}{\mathbb{P}}
\newcommand{\LL}{\mathcal{L}}
\newcommand{\FF}{\mathcal{F}}
\newcommand{\HH}{\mathcal{H}}
\newcommand{\NN}{\mathcal{N}}
\newcommand{\ZZ}{\mathcal{Z}}
\newcommand{\RN}[1]{\textup{\uppercase\expandafter{\romannumeral#1}}}
\newcommand{\ua}{\uparrow}
\newcommand{\da}{\downarrow}

%%% Unit vectors
\newcommand{\xhat}{\vb{\hat{x}}}
\newcommand{\yhat}{\vb{\hat{y}}}
\newcommand{\zhat}{\vb{\hat{z}}}
\newcommand{\nhat}{\vb{\hat{n}}}
\newcommand{\rhat}{\vb{\hat{r}}}
\newcommand{\phihat}{\bm{\hat{\phi}}}
\newcommand{\thetahat}{\bm{\hat{\theta}}}

%%% Other math stuff
\providecommand{\units}[1]{\,\ensuremath{\mathrm{#1}}\xspace}
% Set new style for problem
\newtheoremstyle{problemstyle}  % <name>
        {10pt}                   % <space above>
        {10pt}                   % <space below>
        {\normalfont}           % <body font>
        {}                      % <indent amount}
        {\bfseries\itshape}     % <theorem head font>
        {\normalfont\bfseries:} % <punctuation after theorem head>
        {.5em}                  % <space after theorem head>
        {}                      % <theorem head spec (can be left empty, 
                                % meaning `normal')>

% Set problem environment
\theoremstyle{problemstyle}
\newtheorem{problemenv}{Problem}[section]
\newenvironment{problem}[1]{%
  \renewcommand\theproblemenv{#1}%
  \problemenv
}{\endproblemenv}
% Set lemma environment
\newenvironment{lemma}[2][Lemma]{\begin{trivlist}
\item[\hskip \labelsep {\bfseries #1}\hskip \labelsep {\bfseries #2.}]}{\end{trivlist}}
% Set solution environment
\newenvironment{solution}{
    \begin{proof}[Solution]$ $\par\nobreak\ignorespaces
}{\end{proof}}
\numberwithin{equation}{problemenv}

%%% Page format
\setlength{\parindent}{0.5cm}
\setlength{\oddsidemargin}{0in}
\setlength{\textwidth}{6.5in}
\setlength{\textheight}{8.8in}
\setlength{\topmargin}{0in}
\setlength{\headheight}{18pt}

%%% Code environments
\definecolor{dkgreen}{rgb}{0,0.6,0}
\definecolor{gray}{rgb}{0.5,0.5,0.5}
\definecolor{mauve}{rgb}{0.58,0,0.82}
\lstset{frame=tb,
  language=Python,
  aboveskip=3mm,
  belowskip=3mm,
  showstringspaces=false,
  columns=flexible,
  basicstyle={\small\ttfamily},
  numbers=none,
  numberstyle=\tiny\color{gray},
  keywordstyle=\color{blue},
  commentstyle=\color{dkgreen},
  stringstyle=\color{mauve},
  breaklines=true,
  breakatwhitespace=true,
  tabsize=4
}
\lstset{
  language=Mathematica,
  numbers=left,
  numberstyle=\tiny\color{gray},
  numbersep=5pt,
  breaklines=true,
  captionpos={t},
  frame={lines},
  rulecolor=\color{black},
  framerule=0.5pt,
  columns=flexible,
  tabsize=2
}



\title{Homework 1: Phys 7310 (Fall 2021)}


\begin{document}

\maketitle

\begin{problem}{1.1}[Properties of conductors]\label{p1}

Use Gauss' theorem [and (1.21) if necessary] to prove the following:

(a) Any excess charge placed on a conductor must lie entirely on its surface. (A
conductor by definition contains charges capable of moving freely under the
action of applied electric fields.)

(b) A closed, hollow conductor shields its interior from fields due to charges
outside, but does not shield its exterior from the fields due to charges placed
inside it.

(c) The electric field at the surface of a conductor is normal to the surface
and has a magnitude $\sigma/\epsilon_0$, where $\sigma$ is the charge density
per unit area on the surface.

\begin{solution}
(a)\\[2in]
Pick a volume $V$ within the conductor bounded by a surface $S=\partial V$.
Then, if there is any charge $q\in V$, there is a non-zero electric field
$\vb{E}$ such that
\begin{equation}
    \oint_S\vb{E}\vdot\nhat da=q/\epsilon_0\tag{Gauss Law}
\end{equation}
and there would be no static equilibrium (charges would be accelerated). Thus,
assuming static equilibrium, $\vb{E}=\vb{0}$ inside any volume $V$ in a 
conductor, in which there is no charge $q$. Any excess charge must then be
located on the surface of the conductor.
\newpage

(b)\\[2in]
First, assume there is no charge inside the cavity of a conductor. Then,
pick a closed curve $C=C_1+C_2$ that is partly in the cavity ($C_1$) and partly 
in the conductor ($C_2$). It follows that
\begin{equation}
    \oint_C\vb{E}\vdot d\vb{l}=\qty(\int_{C_1}+\int_{C_2})\vb{E}\vdot d\vb{l}=0 
    \Rightarrow\int_{C_1}\vb{E}_{\text{cavity}}\vdot d\vb{l}
    =-\int_{C_1}\vb{E}_{\text{conductor}}\vdot d\vb{l}
\end{equation}
where $\vb{E}_{\text{cavity}},\vb{E}_{\text{conductor}}$ are the
electric fields inside the cavity and inside the conductor, respectively.
However, $\vb{E}_{\text{conductor}}=\vb{0}$ via Gauss Law. So it must be the
case that $\vb{E}_{\text{cavity}}=\vb{0}$, irrespective of the charge
distribution outside the conductor. Thus, the interior is shielded from charges
out in the world.\\[2in]

Now, assume there is a charge $q$ inside the cavity. If we pick a Gaussian
surface outside of the conductor, then the total enclosed charge is $q$
(assuming the conductor is neutrally charged), so there has to be a non-zero 
electric field due to the charge in the cavity. Thus, the exterior is not
shielded. There are some induced charge on the surface of the cavity
($q_{\text{cavity}}=-q$) to ensure that $\vb{E}_{\text{conductor}}=\vb{0}$.
However, the outer surface must have charge $q$ because the conductor is 
neutrally charged. So the enclosed charge from outside of the conductor is still
$q_{\text{enclosed}}=q+(-q)+q=q$.
\newpage

(c)\\[2in]
In class, we have shown that because $\oint\vb{E}\vdot
d\vb{l}=0$, $\vb{E}_{\|,\text{out}}=\vb{E}_{\|,\text{in}}$ where
$\vb{E}_{\|,\text{out}},\vb{E}_{\|,\text{in}}$ are the electric fields inside
and outside of the conductor, respectively. But $\vb{E}_{\|,\text{in}}=\vb{0}$
for a conductor, so $\vb{E}_{\|,\text{out}}=\vb{0}$. Now,
the normal field satisfies (1.22, Jackson)
\begin{equation}
    \vb{E}_{\perp,\text{out}}-\vb{E}_{\perp,\text{in}}
    =\frac{\sigma}{\epsilon_0}\nhat 
\end{equation}
where $\sigma$ is the surface charge density of the conductor. But similarly, 
$\vb{E}_{\perp,\text{in}}=\vb{0}$ in a conductor. So
$\vb{E}_{\perp,\text{out}}=(\sigma/\epsilon_0)\nhat$.

\end{solution}

\end{problem}



\begin{problem}{1.2}[A neutral hydrogen atom]

The time-averaged potential of a neutral hydrogen atom is given by
\begin{equation}
    \Phi=\frac{q}{4\pi\epsilon_0}\frac{e^{-\alpha r}}{r}\qty(1+\frac{\alpha
    r}{2}) 
\end{equation}
where $q$ is the magnitude of the electronic charge, and $\alpha^{-1}=a_0/2$,
$a_0$ being the Bohr radius. Find the distribution of charge (both continuous
and discrete) that will give this potential and interpret your results
physically.

\begin{solution}
    First note that we can expand $e^{-\alpha r}$ in a Taylor series for
    $r\to0$ as
    \begin{equation}
        e^{-\alpha r}=1-\alpha r+\frac{\alpha^2r^2}{2}+\order{r^3}.
    \end{equation}
    Then it follows that
    \begin{equation}
        \Phi
        =\frac{q}{4\pi\epsilon_0}\qty(\frac1r+\frac{\alpha}{2})
            \qty[1-\alpha r+\frac{\alpha^2r^2}{2}+\order{r^3}]
        =\frac{q}{4\pi\epsilon_0}\qty[\frac1r-\frac\alpha2+\order{r^2}].
    \end{equation}
    For $r\to0$, $\order{r^2}\to0$, so from Poisson's equation (1.28, Jackson),
    we can write the density as
    \begin{equation}
        \rho=-\epsilon_0\laplacian{\Phi}=-\frac{q}{4\pi}
            \laplacian{\qty(\frac1{r})}.
    \end{equation}
    Applying (1.31, Jackson), the density is then $\rho=q\delta^3(r)$ where
    $\delta^3$ is the 3-D Dirac delta function. This is a point charge $q$ 
    located at $r=0$ (the hydrogen ion).
\end{solution}

\end{problem}


\begin{problem}{1.3}[Capacitors]

A simple capacitor is a device formed by two insulated conductors adjacent to
each other. If equal and opposite charges are placed on the conductors, there
will be a certain difference of potential between them. The ratio of the
magnitude of the charge on one conductor to the magnitude of the potential
difference is called the capacitance (in SI units it is measured in farads).
Using Gauss Law, calculate the capacitance of

(a) two large, flat, conducting sheets of area $A$, separated by a small
distance $d$;

(b) two concentric conducting spheres with radii $a,b$ ($b>a$);

(c) two concentric conducting cylinders of length $L$, large compared to their
radii $a,b$ ($b>a$).

(d) What is the inner diameter of the outer conductor in an air-filled coaxial
cable whose center conductor is a cylindrical wire of diameter 1\,\si{mm} and
whose capacitance is $3\times10^{-11}$\,\si{F/m}? $3\times10^{-12}$\,\si{F/m}?

\begin{solution}
   
    (a) Since the separation between two plates are small, if one plate is given
    a charge $Q>0$, the density on its surface is approximately uniform
    $\sigma=Q/A$. The other plate is similarly distributed, but with a surface
    density $-\sigma$, from the assumption of equal and opposite charge on each
    conductor. Draw a Gaussian cylinder $S$ normal to the positively charged 
    plate  along its length with a circular cross-section $a$ such that half of 
    $S$ is inside the capacitor, and the other half is outside. By Gauss Law, 
    the electric flux through the two ends of the cylinder $S$ is
    \begin{equation}
        \oint_S\vb{E}\vdot\nhat da=Ea=\frac{\sigma a}{\epsilon_0}
    \end{equation}
    By symmetry, it follows that the electric field contribution inside the 
    capacitor due to the positive plate is
    $\vb{E}_+=(\sigma/2\epsilon_0)\nhat_+$, where $\nhat_+$
    is the vector normal to the positive plate pointing inside the capacitor
    (see below figure). \\[2in]

    Similarly, we can write the contribution due to the negative plate as
    $\vb{E}_-=(-\sigma/2\epsilon_0)\nhat_-$. Since $\nhat_-=-\nhat_+$, we can
    write the total field inside the capacitor as
    \begin{equation}
        \vb{E}=\vb{E}_++\vb{E}_-=\frac{\sigma}{\epsilon_0}\nhat_+ 
    \end{equation}
    
    Since $\vb{E}=-\grad V$, we can integrate for the potential difference
    between the positive and negative plates, letting $\nhat_+=-\xhat$,
    \begin{equation}
        \Delta V\equiv V_+-V_-=-\int_-^+\vb{E}\vdot\nhat_+dx=\sigma d/\epsilon_0
    \end{equation}
    Then, we can calculate the capacitance
    \begin{equation}
        C=\frac{Q}{\Delta V}=\frac{\sigma A}{\sigma
        d/\epsilon_0}=\frac{\epsilon_0 A}{d} 
    \end{equation}

    (b)\\[3in]Let the inner sphere be positively charged (with $Q>0$) and the
    outer sphere be negatively charged (with $-Q$). Draw a spherical
    Gaussian surface $S$ with $a\leq r\leq b$ (see above figure), the electric 
    flux through it satisfies
    \begin{equation}
        \oint_S\vb{E}\vdot\rhat da=E4\pi r^2=\frac{Q}{\epsilon_0}
    \end{equation}
    Thus, by spherical symmetry, the electric field inside the capacitor
    (between the conductors) is
    \begin{equation}
        \vb{E}=E\rhat=\frac{Q}{4\pi\epsilon_0}\frac1{r^2}\rhat 
    \end{equation}
    Then the potential difference (between $r=a$ and $r=b$) is
    \begin{equation}
        \Delta
        V=V(a)-V(b)=-\int_b^aEdr=\frac{Q}{4\pi\epsilon_0}
            \qty(\frac{1}{a}-\frac{1}{b})
    \end{equation}
    The capacitance is then
    \begin{equation}
        C=\frac{Q}{\Delta V}=4\pi\epsilon_0\frac{ab}{b-a} 
    \end{equation}

    (c) Since $L\gg a,b$, the charge distribution along the length of
    the cylindrical conductors is approximately uniform. Assume the inner 
    cylinder $(r=a)$ is positively charged ($Q>0$), and the outer cylinder 
    ($r=b$) is negatively charged ($-Q$). Then the (linear) charge densities 
    along their length are $\lambda=Q/L$ and $-\lambda$,
    respectively.
    \newpage
    ~\\[2in]

    Now, draw a cylindrical Gaussian surface $S$ with the same length $L$ and 
    radius $r$ such that $a\leq r\leq b$. The charge enclosed is $Q$, and the 
    electric flux normal to the cylindrical part of $S$ is
    \begin{equation}
        \oint_S\vb{E}\vdot\nhat da=E2\pi rL=\frac{\lambda L}{\epsilon_0}
    \end{equation}
    Thus, by symmetry, the electric field inside the capacitor is
    \begin{equation}
        \vb{E}=\frac{\lambda}{2\pi\epsilon_0}\frac1r\rhat 
    \end{equation}
    The potential difference (between $r=a$ and $r=b$) is
    \begin{equation}
        \Delta V=V(a)-V(b)=-\int_b^aEdr=\frac{\lambda}{2\pi\epsilon_0}
            \ln\qty(\frac{b}{a})
    \end{equation}
    Then the capacitance is
    \begin{equation}
        C=\frac{Q}{\Delta V}=-2\pi\epsilon_0L\ln\qty(\frac{a}{b})
    \end{equation}

    (d) The capacitance of this coaxial cable is the same as that calculated in
    part (c). Given $C/L$, we can calculate the inner diameter of the outer
    conductor as
    \begin{equation}
        d=2b=2a\exp\qty(\frac1{2\pi\epsilon_0}\frac{C}{L}) 
    \end{equation}
    For $C/L=3\times10^{-11}$\,\si{F/m}, $d=3.4$\,\si{mm}. For
    $C/L=3\times10^{-12}$\,\si{F/m}, $d=2.1$\,\si{mm}.

\end{solution}
    
\end{problem}


\begin{problem}{1.4}[Green's reciprocation theorem and grounded
    parallel plates]~\\

(a) Prove \textit{Green's reciprocation theorem}: If $\Phi$ is the potential due
to a volume-charge density $\rho$ within a volume $V$ and a surface-charge
$\sigma$ on the conducting surface $S$ bounding the volume $V$, while $\Phi'$ is
the potential due to another charge distribution $\rho'$ and $\sigma'$, then
\begin{equation}\label{eq:wts}
    \int_V\rho\Phi'd^3x+\oint_S\sigma\Phi' da=\int_V\rho'\Phi
    d^3x+\oint_S\sigma'\Phi da 
\end{equation}


(b) Two infinite grounded parallel conducting planes are separated by a distance
$d$. A point charge $q$ is placed between the planes. Use the reciprocation
theorem of Green to prove that the total induced charge on one of the plane is
equal to $(-q)$ times the fractional perpendicular distance of the point charge
from the other plane. (\textit{Hint}: As your comparison electrostatic problem
with the same surfaces choose one whose charge densities and potential are known
and simple.)

\begin{solution}
    (a) To prove \eqref{eq:wts}, it suffices to show that
    \begin{equation}
        \int_V\qty(\rho\Phi'-\rho'\Phi)d^3x
        =\oint_S\qty(\sigma'\Phi-\sigma\Phi')da
    \end{equation}
    From Poisson's equation ($\rho=-\epsilon_0\laplacian\Phi$), we can write 
    \begin{equation}
        LHS=-\epsilon_0\int_V
        \qty(\Phi'\laplacian\Phi-\Phi\laplacian{\Phi'})d^3x
        =-\epsilon_0\oint_S\qty(
        \Phi'\frac{\partial\Phi}{\partial n}-\Phi\frac{\partial\Phi'}{\partial
        n})da
    \end{equation}
    where we have applied Green's theorem (1.35, Jackson) to the last equality.
    Now, since $S$ is a conductor, it follows that $E=\partial\Phi/\partial
    n=\sigma/\epsilon_0$ (in the notation in Section 1.8 of Jackson) and
    \begin{equation}
        LHS=\oint_S\qty(\sigma'\Phi-\sigma\Phi')da=RHS   
    \end{equation}

    (b) Let the un-primed density distribution and potential be as described by
    the problem:
    \begin{subequations}\label{eq:bc1}
        \begin{align}
            \rho&=q\delta(x-d')\\
            \sigma&=\sigma_1\delta_{x,0}+\sigma_2\delta_{x,d}\\
            \Phi&=0,\qquad x\in\qty{0,d}
        \end{align} 
    \end{subequations}
    where $\delta$ is the Dirac delta function, $\delta_{x,0}$ is the
    Kronecker delta function ($\delta_{x,0}=1$ when $x=0$), the potential 
    $\Phi$ is zero at the locations of the grounded plates, and
    $\sigma_1,\sigma_2$ are the surface charge densities in the first and 
    second plate, respectively, due to induction from the point charge $q$ (see 
    figure below).\\[2in]

    Let the primed density distribution and potential be those corresponding to 
    an infinite, parallel plate capacitor
    \begin{subequations}\label{eq:bc2}
        \begin{align}
            \rho'&=0\\
            \sigma'&=-\sigma_0\delta_{x,0}+\sigma_0\delta_{x,d}\\
            \Phi'&=Ex=\frac{\sigma_0}{\epsilon_0}x
        \end{align} 
    \end{subequations}
    where the charge densities on the first and second plate are $-\sigma_0$ and
    $\sigma_0$. The potential at the first plate is zero, so the potential
    difference between the two plates is $\Phi'(d)=\sigma_0
    d/\epsilon_0$ (see figure below).\\[2in]

    Now, applying \eqref{eq:bc1} and \eqref{eq:bc2} to \eqref{eq:wts}, it 
    follows that
    \begin{equation}
        \frac{\sigma_0}{\epsilon_0}\qty(\int_0^d q\delta(x-d')xdx
        +\oint_{\R^2}\sigma_1\delta_{x,0}x da
        +\oint_{\R^2}\sigma_2\delta_{x,d}xda)=0
    \end{equation}
    This implies that
    \begin{equation}\label{eq:almost}
        qd'+d\oint_{\R^2}\sigma_2da=qd'+q_2d=0
    \end{equation}
    where $q_2=\oint_{\R^2}\sigma_2da$ is the induced charge on the
    second plate due to the point charge $q$. Inverting \eqref{eq:almost}, we 
    get $q_2=(-q)(d'/d)$.
\end{solution}
    
\end{problem}


\end{document}
