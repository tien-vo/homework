\documentclass[12pt]{article}


\input{$HOME/.config/latex/preamble.tex}


\title{Homework 1: PHYS 7310 (Fall 2021)}


\begin{document}

\maketitle

\begin{problem}[Problem 1.1]{(Properties of conductors)}

Use Gauss' theorem [and (1.21) if necessary] to prove the following:

(a) Any excess charge placed on a conductor must lie entirely on its surface. (A
conductor by definition contains charges capable of moving freely under the
action of applied electric fields.)

(b) A closed, hollow conductor shields its interior from fields due to charges
outside, but does not shield its exterior from the fields due to charges placed
inside it.

(c) The electric field at the surface of a conductor is normal to the surface
and has a magnitude $\sigma/\epsilon_0$, where $\sigma$ is the charge density
per unit area on the surface.

\end{problem}


\begin{problem}[Problem 1.2]{(A neutral hydrogen atom)}

The time-averaged potential of a neutral hydrogen atom is given by
\begin{equation}
    \Phi=\frac{q}{4\pi\epsilon_0}\frac{e^{-\alpha r}}{r}\qty(1+\frac{\alpha
    r}{2}) 
\end{equation}
where $q$ is the magnitude of the electronic charge, and $\alpha^{-1}=a_0/2$,
$a_0$ being the Bohr radius. Find the distribution of charge (both continuous
and discrete) that will give this potential and interpret your results
physically.

\end{problem}


\begin{problem}[Problem 1.3]{(Capacitors)}

A simple capacitor is a device formed by two insulated conductors adjacent to
each other. If equal and opposite charges are placed on the conductors, there
will be a certain difference of potential between them. The ratio of the
magnitude of the charge on one conductor to the magnitude of the potential
difference is called the capacitance (in SI units it is measured in farads).
Using Gauss Law, calculate the capacitance of

(a) two large, flat, conducting sheets of area $A$, separated by a small
distance $d$;

(b) two concentric conducting spheres with radii $a,b$ ($b>a$);

(c) two concentric conducting cylinders of length $L$, large compared to their
radii $a,b$ ($b>a$).
    
\end{problem}


\begin{problem}[Problem 1.4]{(Green's reciprocation theorem and grounded
    parallel plates)}

(a) Prove \textit{Green's reciprocation theorem}: If $\Phi$ is the potential due
to a volume-charge density $\rho$ within a volume $V$ and a surface-charge
$\sigma$ on the conducting surface $S$ bounding the volume $V$, while $\Phi'$ is
the potential due to another charge distribution $\rho'$ and $\sigma'$, then
\begin{equation}
    \int_V\rho\Phi'd^3x+\int_S\sigma\Phi' da=\int_V\rho'\Phi
    d^3x+\int_S\sigma'\Phi da 
\end{equation}


(b) Two infinite grounded parallel conducting planes are separated by a distance
$d$. A point charge $q$ is placed between the planes. Use the reciprocation
theorem of Green to prove that the total induced charge on one of the plane is
equal to $(-q)$ times the fractional perpendicular distance of the point charge
from the other plane. (\textit{Hint}: As your comparison electrostatic problem
with the same surfaces choose one whose charge densities and potential are known
and simple.)
    
\end{problem}
    
\end{document}
