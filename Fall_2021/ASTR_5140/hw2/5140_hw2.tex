\documentclass[12pt]{article}

%%%%%%%%%%%%%%%%%%%%%%%%%%%%%%%%%%%%%%%%%%%%%%%%%%%%%%%%%%%%%%%%%%%%%%%%%%%%%%%%
%                           Package preset for homework
%%%%%%%%%%%%%%%%%%%%%%%%%%%%%%%%%%%%%%%%%%%%%%%%%%%%%%%%%%%%%%%%%%%%%%%%%%%%%%%%
% Miscellaneous
\usepackage[margin=1in]{geometry}
\usepackage[utf8]{inputenc}
\usepackage{indentfirst}
\usepackage{blindtext}
\usepackage{graphicx}
\usepackage{xr-hyper}
\usepackage{hyperref}
\usepackage{color}
\usepackage{float}
% Math
\usepackage{latexsym}
\usepackage{amsfonts}
\usepackage{amssymb}
\usepackage{amsmath}
\usepackage{commath}
\usepackage{amsthm}
\usepackage{bbold}
\usepackage{bm}
% Physics
\usepackage{physics}
\usepackage{siunitx}
% Code typesetting
\usepackage{listings}
% Citation
\usepackage[authoryear]{natbib}
\usepackage{appendix}
\usepackage[capitalize]{cleveref}
% Title & name
\title{Homework}
\author{Tien Vo}
\date{\today}


%%%%%%%%%%%%%%%%%%%%%%%%%%%%%%%%%%%%%%%%%%%%%%%%%%%%%%%%%%%%%%%%%%%%%%%%%%%%%%%%
%                   User-defined commands and environments
%%%%%%%%%%%%%%%%%%%%%%%%%%%%%%%%%%%%%%%%%%%%%%%%%%%%%%%%%%%%%%%%%%%%%%%%%%%%%%%%
%%% Misc
\sisetup{load-configurations=abbreviations}
\newcommand{\due}[1]{\date{Due: #1}}
\newcommand{\hint}{\textit{Hint}}
\let\oldt\t
\renewcommand{\t}[1]{\text{#1}}

%%% Bold sets & abbrv
\newcommand{\N}{\mathbb{N}}
\newcommand{\Z}{\mathbb{Z}}
\newcommand{\R}{\mathbb{R}}
\newcommand{\Q}{\mathbb{Q}}
\let\oldP\P
\renewcommand{\P}{\mathbb{P}}
\newcommand{\LL}{\mathcal{L}}
\newcommand{\FF}{\mathcal{F}}
\newcommand{\HH}{\mathcal{H}}
\newcommand{\NN}{\mathcal{N}}
\newcommand{\ZZ}{\mathcal{Z}}
\newcommand{\RN}[1]{\textup{\uppercase\expandafter{\romannumeral#1}}}
\newcommand{\ua}{\uparrow}
\newcommand{\da}{\downarrow}

%%% Unit vectors
\newcommand{\xhat}{\vb{\hat{x}}}
\newcommand{\yhat}{\vb{\hat{y}}}
\newcommand{\zhat}{\vb{\hat{z}}}
\newcommand{\nhat}{\vb{\hat{n}}}
\newcommand{\rhat}{\vb{\hat{r}}}
\newcommand{\phihat}{\bm{\hat{\phi}}}
\newcommand{\thetahat}{\bm{\hat{\theta}}}

%%% Other math stuff
\providecommand{\units}[1]{\,\ensuremath{\mathrm{#1}}\xspace}
% Set new style for problem
\newtheoremstyle{problemstyle}  % <name>
        {10pt}                   % <space above>
        {10pt}                   % <space below>
        {\normalfont}           % <body font>
        {}                      % <indent amount}
        {\bfseries\itshape}     % <theorem head font>
        {\normalfont\bfseries:} % <punctuation after theorem head>
        {.5em}                  % <space after theorem head>
        {}                      % <theorem head spec (can be left empty, 
                                % meaning `normal')>

% Set problem environment
\theoremstyle{problemstyle}
\newtheorem{problemenv}{Problem}[section]
\newenvironment{problem}[1]{%
  \renewcommand\theproblemenv{#1}%
  \problemenv
}{\endproblemenv}
% Set lemma environment
\newenvironment{lemma}[2][Lemma]{\begin{trivlist}
\item[\hskip \labelsep {\bfseries #1}\hskip \labelsep {\bfseries #2.}]}{\end{trivlist}}
% Set solution environment
\newenvironment{solution}{
    \begin{proof}[Solution]$ $\par\nobreak\ignorespaces
}{\end{proof}}
\numberwithin{equation}{problemenv}

%%% Page format
\setlength{\parindent}{0.5cm}
\setlength{\oddsidemargin}{0in}
\setlength{\textwidth}{6.5in}
\setlength{\textheight}{8.8in}
\setlength{\topmargin}{0in}
\setlength{\headheight}{18pt}

%%% Code environments
\definecolor{dkgreen}{rgb}{0,0.6,0}
\definecolor{gray}{rgb}{0.5,0.5,0.5}
\definecolor{mauve}{rgb}{0.58,0,0.82}
\lstset{frame=tb,
  language=Python,
  aboveskip=3mm,
  belowskip=3mm,
  showstringspaces=false,
  columns=flexible,
  basicstyle={\small\ttfamily},
  numbers=none,
  numberstyle=\tiny\color{gray},
  keywordstyle=\color{blue},
  commentstyle=\color{dkgreen},
  stringstyle=\color{mauve},
  breaklines=true,
  breakatwhitespace=true,
  tabsize=4
}
\lstset{
  language=Mathematica,
  numbers=left,
  numberstyle=\tiny\color{gray},
  numbersep=5pt,
  breaklines=true,
  captionpos={t},
  frame={lines},
  rulecolor=\color{black},
  framerule=0.5pt,
  columns=flexible,
  tabsize=2
}


\title{Homework 2: Astr 5140 (Fall 2021)}

\begin{document}

\maketitle

%%%%%%%%%%%%%%%%%%%%%%%%%%%%%%%%%%%%%%%%%%%%%%%%%%%%%%%%%%%%%%%%%%%%%%%%%%%%%%%%
\begin{problem}{1}[$\vb{J}\times\vb{B}$ force]
Using the two-fluid equations, calculate the Lorentz force per unit volume on a
quasi-neutral (MHD) plasma using the definition of $\vb{J}$. Separate the
electric force ($\vb{F}_E$) from the magnetic force ($\vb{F}_B$). Show that if
the plasma is quasi-neutral, then $\vb{F}$ reduces to the standard MHD result.
\end{problem}
%%%%%%%%%%%%%%%%%%%%%%%%%%%%%%%%%%%%%%%%%%%%%%%%%%%%%%%%%%%%%%%%%%%%%%%%%%%%%%%%
\begin{problem}{2}[EM review: Waves in a plasma]
Using Maxwell's equations and setting the current to be the electron motion 
only 
\begin{equation}
    \frac{\partial\vb{J}}{\partial t}=\frac{ne^2E}{m_e} 
\end{equation}
Show that the solution of a transverse light wave becomes:
\begin{equation}
    \omega^2=\omega_{pe}^2+k^2c^2 
\end{equation}
where $\omega_{pe}^2=ne^2/\epsilon_0m_e$.
\end{problem}
%%%%%%%%%%%%%%%%%%%%%%%%%%%%%%%%%%%%%%%%%%%%%%%%%%%%%%%%%%%%%%%%%%%%%%%%%%%%%%%%
\begin{problem}{3}[Pick-up Ion at Mars]
Suppose an $O$ atom that escaped from Mars is at rest (in our frame) at
$x=0,y=0$. It is photo-ionized (charge $e$) at $t=0$ in the solar wind
($v_{sw}=350$\,\si{km/s} in the $x$ direction) with a magnetic field
$\vb{B}=10$\,\si{nT} in the $z$ direction (see diagram). Assume that the
photo-ionization does not move the $O^+$ ion.

(1) Describe the subsequent motion of the $O^+$ ion, $x(t),y(t),v_x(t),v_y(t)$,
in our rest frame (not the plasma frame). \textit{Hint}: What is the solar wind
electric field? Calculate $v_x(t)$ and $v_y(t)$ then integrate. Apply boundary
conditions, $x(t=0)=0,y(t=0)=0,v_x(t=0)=0,v_y(t=0)=0$ to get an exact solution.

(2) Sketch the $O^+$ path.

(3) The drift and gyration cause the $O^+$ ion to slow down and speed up in our
rest frame. What is the drift speed? What is the gyration speed? What is the
maximum velocity (\si{km/s}) and energy (\si{keV}) that the $O^+$ ion reaches?

(4) Using the gyration speed only, what is the perpendicular (to $\vb{B}$)
temperature of that ion in $^\circ$\,\si{K}? (In 2D, temperature and energy are
equal.)
\end{problem}
%%%%%%%%%%%%%%%%%%%%%%%%%%%%%%%%%%%%%%%%%%%%%%%%%%%%%%%%%%%%%%%%%%%%%%%%%%%%%%%%
\begin{problem}{4}[Current sheet]
A current sheet is such that $\vb{J}$ is in the $y$ direction and $\vb{B}$ is in
the $z$ direction. $\vb{B},\vb{J}$, and $P$ vary only with $x$ (see diagram).
Derive a solution for $\vb{B}$, and $\vb{J}$ under the condition $\vb{J}\sim P$
that is valid for $-L<x<L$. Make sure that your solution satisfies the boundary
conditions of $\vb{B}(x=L)=-B_0\zhat$, and $\vb{B}(x=0)=\vb{0}$. $L$
is a characteristic length. Sketch your results.

\textit{Hint}: There is more than one possible solution -- just give any valid
solution. Be careful, the condition $\vb{J}^2\sim P$ is NOT the same as in the
Harris solution. 
\end{problem}
%%%%%%%%%%%%%%%%%%%%%%%%%%%%%%%%%%%%%%%%%%%%%%%%%%%%%%%%%%%%%%%%%%%%%%%%%%%%%%%%
\begin{problem}{5}[Magnetic diffusion]
Consider a magnetic field $\vb{B}=B_z(x,t)\zhat$ where
$B_z(x,t=0)=B_0\cos(k_1x)+B_0\cos(k_2x)$ in a resistive plasma with $k_2\gg
k_1$.

(a) Find the solution for $B_z(x,t)$.

(b) Sketch (accurate plot not needed) the solution for $t=0$ and $t>0$. What
happens to the high-$k$ wave?
\end{problem}
%%%%%%%%%%%%%%%%%%%%%%%%%%%%%%%%%%%%%%%%%%%%%%%%%%%%%%%%%%%%%%%%%%%%%%%%%%%%%%%%

\end{document}
