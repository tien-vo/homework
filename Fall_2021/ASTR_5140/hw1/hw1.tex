\documentclass[12pt]{article}

\input{$HOME/.config/latex/preamble.tex}

\title{Homework 1: ASTR 5140 (Fall 2021)}

\begin{document}

\maketitle


\begin{problem}[Problem 1.]{Maxwellian distribution}

The most often used particle distribution in plasma physics is the drifting
Maxwellian separate parallel and perpendicular temperature
\begin{equation}
    f(\vb{v})=Ae^{-\frac{m}{2}\qty(
        \frac{(v_x-u_x)^2}{T_\perp}+
        \frac{(v_y-u_y)^2}{T_\perp}+
        \frac{(v_z-u_z)^2}{T_\|}
    )}
\end{equation}
where $\vb{u}$ is the drift velocity (fluid velocity), $\vb{v}$ is the
individual particle's velocity, and
\begin{equation}
    A=n\qty(\frac{m}{2\pi T_\perp})\qty(\frac{m}{2\pi T_\|})^{1/2} 
\end{equation}

(a) Show that
\begin{equation}
    \int_{-\infty}^\infty f(\vb{v})=n  
\end{equation}
\textit{Hint}: Substitute the variable $\vb{w}=\vb{v}-\vb{u}$. Do the 
integration for one dimension, then deduce the result for the other two 
dimension.

(b) Show that:
\begin{equation}
    \int_{-\infty}^\infty\vb{v}f(\vb{v})=n\vb{u} 
\end{equation}
\textit{Hint}: Use symmetry arguments (e.g. the odd functions integrate to 0) to
avoid carrying out the integration. Be succint. Carry out the integration for
one direction and deduce the result for the other directions.

(c) Show that:
\begin{equation}
    \int_{-\infty}^\infty \vb{v}\vb{v}f(\vb{v})=n\vb{u}\vb{u}+\frac{n\vb{T}}{m} 
\end{equation}
where $\vb{T}=\text{diag}\qty(T_\perp,T_\perp,T_\|)$. \textit{Hint}: Solve one
diagonal term, for example $v_xv_x$, and one off-diagonal term, for example
$v_xv_y$. Deduce the results for the remaining terms. Again, use symmetry
arguments where possible.

(d) Sketch $f$ versus $v_x$ by hand.
\end{problem}


\begin{problem}[Problem 2.]{Vlasov Equation}

As done in class, let particle $n$ of a given species be defined as a Dirac
delta function
\begin{equation}
    \delta(\vb{X}_n(t)-\vb{x})\delta(\vb{V}_n(t)-\vb{v}) 
\end{equation}
where $\vb{X}_n(t)$ is the instantaneous position of a particle and
$\vb{V}_n(t)$ is the instantaneous velocity of the particle. The distribution
function for species $S$ can be described as
\begin{equation}
    F_S(x,v,t)=\sum_n\delta(\vb{x}-\vb{X}_n(t))\delta(\vb{v}-\vb{V}_n(t)) 
\end{equation}

Derive the ``Vlasov Equation'' for the unsmoothed distribution $F$ using only
electromagnetic force.
    
\end{problem}


\begin{problem}[Problem 3.]{Math Review}

We will use vector notation including cross products, curl, and divergences
quite often in this course so it is useful to be able to manipulate them.

(a) Show that: $\curl{\curl{\vb{A}}}=\grad(\div{\vb{A}})-\laplacian{\vb{A}}$

(b) Show that:
$\div{\qty(\vb{A}+\vb{B})}=\vb{B}\vdot\qty(\curl{\vb{A}})-\vb{A}\vdot\qty(\curl{\vb{B}})$

\textit{Hint}: One method is to use the Levi-Civita symbol, $\xi_{ijk}$, where
\begin{equation}
    \xi_{ijk}=\begin{cases}
        1&(i,j,k)\in\qty{(1,2,3),(3,1,2),(2,3,1)}\\
        -1&(i,j,k)\in\qty{(3,2,1),(1,3,2),(2,3,1)}
    \end{cases}
\end{equation}

\end{problem}


\begin{problem}[Problem 4.]{Quasi-neutral Plasma}

Calculate the condition of the ratio $\Delta N_c/N$ for gravity to dominate over
the electromagnetic force on a proton near a star. $N$ is the total number of
protons in the star and $\Delta N_c$ is the number of unbalanced charges. Show
that
\begin{equation}
    \frac{\Delta N_c}{N}\ll8\times10^{-37} 
\end{equation}
    
\end{problem}


\begin{problem}[Problem 5.]{Debye Shielding}
    
(a) Consider a conducting sphere of radius $a$ and charge $Q$ that is immersed
in a collisionless, Maxwellian plasma that has density $n_0$, $T_i=0$, but
finite $T_e$. Let $\vb{B}=0$. Solve the time-independent electron momentum
equation
\begin{equation}
    eEn_e+\gamma T_e\frac{\partial n_e}{\partial r}=0 
\end{equation}
to show that the isothermal equilibrium $(\gamma=1)$ electron density can be
expressed as
\begin{equation}
    n_e=n_0e^{e\phi/T_e},\qquad r>a 
\end{equation}

(b) Let the potential at the sphere be $\phi_0$. In the limit $\phi_0\ll T_e$,
derive the potential $\phi$ as a function of $r$ in spherical coordinates.
Express your answer in terms of $r,a,\phi_0$, and $\lambda_D$.

\end{problem}

\end{document}

