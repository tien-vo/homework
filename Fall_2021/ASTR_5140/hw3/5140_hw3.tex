\documentclass[12pt]{article}

%%%%%%%%%%%%%%%%%%%%%%%%%%%%%%%%%%%%%%%%%%%%%%%%%%%%%%%%%%%%%%%%%%%%%%%%%%%%%%%%
%                           Package preset for homework
%%%%%%%%%%%%%%%%%%%%%%%%%%%%%%%%%%%%%%%%%%%%%%%%%%%%%%%%%%%%%%%%%%%%%%%%%%%%%%%%
% Miscellaneous
\usepackage[margin=1in]{geometry}
\usepackage[utf8]{inputenc}
\usepackage{indentfirst}
\usepackage{blindtext}
\usepackage{graphicx}
\usepackage{xr-hyper}
\usepackage{hyperref}
\usepackage{color}
\usepackage{float}
% Math
\usepackage{latexsym}
\usepackage{amsfonts}
\usepackage{amssymb}
\usepackage{amsmath}
\usepackage{commath}
\usepackage{amsthm}
\usepackage{bbold}
\usepackage{bm}
% Physics
\usepackage{physics}
\usepackage{siunitx}
% Code typesetting
\usepackage{listings}
% Citation
\usepackage[authoryear]{natbib}
\usepackage{appendix}
\usepackage[capitalize]{cleveref}
% Title & name
\title{Homework}
\author{Tien Vo}
\date{\today}


%%%%%%%%%%%%%%%%%%%%%%%%%%%%%%%%%%%%%%%%%%%%%%%%%%%%%%%%%%%%%%%%%%%%%%%%%%%%%%%%
%                   User-defined commands and environments
%%%%%%%%%%%%%%%%%%%%%%%%%%%%%%%%%%%%%%%%%%%%%%%%%%%%%%%%%%%%%%%%%%%%%%%%%%%%%%%%
%%% Misc
\sisetup{load-configurations=abbreviations}
\newcommand{\due}[1]{\date{Due: #1}}
\newcommand{\hint}{\textit{Hint}}
\let\oldt\t
\renewcommand{\t}[1]{\text{#1}}

%%% Bold sets & abbrv
\newcommand{\N}{\mathbb{N}}
\newcommand{\Z}{\mathbb{Z}}
\newcommand{\R}{\mathbb{R}}
\newcommand{\Q}{\mathbb{Q}}
\let\oldP\P
\renewcommand{\P}{\mathbb{P}}
\newcommand{\LL}{\mathcal{L}}
\newcommand{\FF}{\mathcal{F}}
\newcommand{\HH}{\mathcal{H}}
\newcommand{\NN}{\mathcal{N}}
\newcommand{\ZZ}{\mathcal{Z}}
\newcommand{\RN}[1]{\textup{\uppercase\expandafter{\romannumeral#1}}}
\newcommand{\ua}{\uparrow}
\newcommand{\da}{\downarrow}

%%% Unit vectors
\newcommand{\xhat}{\vb{\hat{x}}}
\newcommand{\yhat}{\vb{\hat{y}}}
\newcommand{\zhat}{\vb{\hat{z}}}
\newcommand{\nhat}{\vb{\hat{n}}}
\newcommand{\rhat}{\vb{\hat{r}}}
\newcommand{\phihat}{\bm{\hat{\phi}}}
\newcommand{\thetahat}{\bm{\hat{\theta}}}

%%% Other math stuff
\providecommand{\units}[1]{\,\ensuremath{\mathrm{#1}}\xspace}
% Set new style for problem
\newtheoremstyle{problemstyle}  % <name>
        {10pt}                   % <space above>
        {10pt}                   % <space below>
        {\normalfont}           % <body font>
        {}                      % <indent amount}
        {\bfseries\itshape}     % <theorem head font>
        {\normalfont\bfseries:} % <punctuation after theorem head>
        {.5em}                  % <space after theorem head>
        {}                      % <theorem head spec (can be left empty, 
                                % meaning `normal')>

% Set problem environment
\theoremstyle{problemstyle}
\newtheorem{problemenv}{Problem}[section]
\newenvironment{problem}[1]{%
  \renewcommand\theproblemenv{#1}%
  \problemenv
}{\endproblemenv}
% Set lemma environment
\newenvironment{lemma}[2][Lemma]{\begin{trivlist}
\item[\hskip \labelsep {\bfseries #1}\hskip \labelsep {\bfseries #2.}]}{\end{trivlist}}
% Set solution environment
\newenvironment{solution}{
    \begin{proof}[Solution]$ $\par\nobreak\ignorespaces
}{\end{proof}}
\numberwithin{equation}{problemenv}

%%% Page format
\setlength{\parindent}{0.5cm}
\setlength{\oddsidemargin}{0in}
\setlength{\textwidth}{6.5in}
\setlength{\textheight}{8.8in}
\setlength{\topmargin}{0in}
\setlength{\headheight}{18pt}

%%% Code environments
\definecolor{dkgreen}{rgb}{0,0.6,0}
\definecolor{gray}{rgb}{0.5,0.5,0.5}
\definecolor{mauve}{rgb}{0.58,0,0.82}
\lstset{frame=tb,
  language=Python,
  aboveskip=3mm,
  belowskip=3mm,
  showstringspaces=false,
  columns=flexible,
  basicstyle={\small\ttfamily},
  numbers=none,
  numberstyle=\tiny\color{gray},
  keywordstyle=\color{blue},
  commentstyle=\color{dkgreen},
  stringstyle=\color{mauve},
  breaklines=true,
  breakatwhitespace=true,
  tabsize=4
}
\lstset{
  language=Mathematica,
  numbers=left,
  numberstyle=\tiny\color{gray},
  numbersep=5pt,
  breaklines=true,
  captionpos={t},
  frame={lines},
  rulecolor=\color{black},
  framerule=0.5pt,
  columns=flexible,
  tabsize=2
}


\title{Homework 3: Astr 5140 (Fall 2021)}

\begin{document}
\maketitle
%%%%%%%%%%%%%%%%%%%%%%%%%%%%%%%%%%%%%%%%%%%%%%%%%%%%%%%%%%%%%%%%%%%%%%%%%%%%%%%%
\begin{problem}{1}[Generalized Ohm's Law]
The MHD force equation is derived by a linear combination of the electron and
ion fluid equations. Generalized Ohm's law is, simply stated, a different linear
combination.

We begin by multiplying the ion force equation by $m_e$ and the electron force
equation by $m_i$, then subtract
\begin{subequations}
    \begin{align}
        m_em_in\frac{D\vb{u}_i}{D t}
        &=-m_e\grad P_i+nm_ee(\vb{E}+\vb{u}_i\times\vb{B})
            -m_em_in\nu_{ie}(\vb{u}_i-\vb{u}_e)\\
        m_im_en\frac{D\vb{u}_e}{D t}
        &=-m_i\grad P_e-nm_ie(\vb{E}+\vb{u}_e\times\vb{B})
            -m_em_in\nu_{ei}(\vb{u}_e-\vb{u}_i)
    \end{align}
\end{subequations}
We use quasi-neutrality and bundle the convective derivative into the symbol
$D$.
\begin{align}
    m_em_in\qty(\frac{D\vb{u}_i}{Dt}-\frac{D\vb{u}_e}{Dt})
    &=\qty(-m_e\grad P_i+m_i\grad P_e)
    +en\vb{E}(m_e+m_i)\notag\\
    &\qquad+en\qty(m_e\vb{u}_i+m_i\vb{u}_e)\times\vb{B}
    -m_em_in(\nu_{ei}+\nu_{ie})(\vb{u}_i-\vb{u}_e)
\end{align}

(a) Divide each term by $en(m_i+m_e)$.

(b) Show that Term 1 can be re-written in the limit of $m_i\gg m_e$ as
\begin{equation}
    \frac{m_e}{e^2}\frac{D(\vb{J}/n)}{Dt} 
\end{equation}

(c) Argue that since $m_i\gg m_e$ that, unless $P_i\gg P_e$ (very rare), Term 2
becomes
\begin{equation}
    \frac{\grad P_e}{en} 
\end{equation}

(d) Term 3 is trivial. Term 4 is tricky as it must be broken into two parts. Add
$\vb{u}=(m_i\vb{u}_i+m_e\vb{u}_e) /(m_i+m_e)$, separate $\vb{u}\times\vb{B}$,
then subtract ($m_i\vb{u}_i+m_e\vb{u}_e) /(m_i+m_e)$. Show that the remaining
four terms ($m_i\gg m_e$) can be approximated as
\begin{equation}
    \frac{-\vb{J}\times\vb{B}}{en} 
\end{equation}

(e) Show that Term 5 can be written as $\vb{J}/\sigma$. Define $\sigma$.

In the end, you should arrive at
\begin{equation}
    \vb{E}+\vb{u}\times\vb{B}=\frac{\vb{J}}{\sigma}+\frac{\vb{J}\times\vb{B}}{en}-\frac{\grad
    P_e}{en}+\frac{m_e}{ne^2}\frac{D\vb{J}}{Dt}+\text{small terms} 
\end{equation}

\textbf{Note}: If done exactly (full convective derivative and keeping small
terms), $n$ is not inside the derivative and furthermore, some of the
''leftovers'' from Term 1 cancel ''leftovers'' in Term 4.
\begin{solution}
\end{solution}
\end{problem}
%%%%%%%%%%%%%%%%%%%%%%%%%%%%%%%%%%%%%%%%%%%%%%%%%%%%%%%%%%%%%%%%%%%%%%%%%%%%%%%%    
%%%%%%%%%%%%%%%%%%%%%%%%%%%%%%%%%%%%%%%%%%%%%%%%%%%%%%%%%%%%%%%%%%%%%%%%%%%%%%%%
\begin{problem}{2}[Scale height in the solar corona]
The surface gravity of the Sun is $274$\,\si{m/s\tothe{2}}. Assume that the
corona is entirely protons with $T_{\text{corona}}=10^6$\,$^\circ$\si{K}.
Derive the isothermal scale height of the solar corona from the MHD equations
assuming $\vb{g}$ is constant (use 1D). How does this value compare with the
radius of the Sun (what \% of $R_{\text{sun}}$ is $H_0$)? Does your answer agree
with hat is seen in UV or X-ray images?
\begin{solution}
\end{solution}
\end{problem}
%%%%%%%%%%%%%%%%%%%%%%%%%%%%%%%%%%%%%%%%%%%%%%%%%%%%%%%%%%%%%%%%%%%%%%%%%%%%%%%%

%%%%%%%%%%%%%%%%%%%%%%%%%%%%%%%%%%%%%%%%%%%%%%%%%%%%%%%%%%%%%%%%%%%%%%%%%%%%%%%%
\begin{problem}{4}[Magnetic tension]
This problem is to develop an intuition for magnetic tension. Examine the
diagram, which shows a field line if $\vb{B}=B_0[(z /z_0)\xhat+\zhat$. Examine
the field line and convince yourself that the equation represents a curved
magnetic field line. Derive the tension force at $z=0$. What is the direction of
the tension force? Argue that $z_0$ is the local radius of curvature.
\begin{solution}
\end{solution}
\end{problem}
%%%%%%%%%%%%%%%%%%%%%%%%%%%%%%%%%%%%%%%%%%%%%%%%%%%%%%%%%%%%%%%%%%%%%%%%%%%%%%%%

\end{document}
