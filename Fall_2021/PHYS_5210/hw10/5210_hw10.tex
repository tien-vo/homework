\documentclass[12pt]{article}

%%%%%%%%%%%%%%%%%%%%%%%%%%%%%%%%%%%%%%%%%%%%%%%%%%%%%%%%%%%%%%%%%%%%%%%%%%%%%%%%
%                           Package preset for homework
%%%%%%%%%%%%%%%%%%%%%%%%%%%%%%%%%%%%%%%%%%%%%%%%%%%%%%%%%%%%%%%%%%%%%%%%%%%%%%%%
% Miscellaneous
\usepackage[margin=1in]{geometry}
\usepackage[utf8]{inputenc}
\usepackage{indentfirst}
\usepackage{blindtext}
\usepackage{graphicx}
\usepackage{xr-hyper}
\usepackage{hyperref}
\usepackage{enumitem}
\usepackage{color}
\usepackage{float}
% Math
\usepackage{latexsym}
\usepackage{amsfonts}
\usepackage{amssymb}
\usepackage{amsmath}
\usepackage{commath}
\usepackage{amsthm}
\usepackage{bbold}
\usepackage{bm}
% Physics
\usepackage{physics}
\usepackage{siunitx}
% Code typesetting
\usepackage{listings}
% Citation
\usepackage[authoryear]{natbib}
\usepackage{appendix}
\usepackage[capitalize]{cleveref}
% Title & name
\title{Homework}
\author{Tien Vo}
\date{\today}


%%%%%%%%%%%%%%%%%%%%%%%%%%%%%%%%%%%%%%%%%%%%%%%%%%%%%%%%%%%%%%%%%%%%%%%%%%%%%%%%
%                   User-defined commands and environments
%%%%%%%%%%%%%%%%%%%%%%%%%%%%%%%%%%%%%%%%%%%%%%%%%%%%%%%%%%%%%%%%%%%%%%%%%%%%%%%%
%%% Misc
\sisetup{load-configurations=abbreviations}
\newcommand{\due}[1]{\date{Due: #1}}
\newcommand{\hint}{\textit{Hint}}
\let\oldt\t
\renewcommand{\t}[1]{\text{#1}}

%%% Bold sets & abbrv
\newcommand{\N}{\mathbb{N}}
\newcommand{\Z}{\mathbb{Z}}
\newcommand{\R}{\mathbb{R}}
\newcommand{\Q}{\mathbb{Q}}
\let\oldP\P
\renewcommand{\P}{\mathbb{P}}
\newcommand{\LL}{\mathcal{L}}
\newcommand{\FF}{\mathcal{F}}
\newcommand{\HH}{\mathcal{H}}
\newcommand{\NN}{\mathcal{N}}
\newcommand{\ZZ}{\mathcal{Z}}
\newcommand{\RN}[1]{\textup{\uppercase\expandafter{\romannumeral#1}}}
\newcommand{\ua}{\uparrow}
\newcommand{\da}{\downarrow}

%%% Unit vectors
\newcommand{\xhat}{\vb{\hat{x}}}
\newcommand{\yhat}{\vb{\hat{y}}}
\newcommand{\zhat}{\vb{\hat{z}}}
\newcommand{\nhat}{\vb{\hat{n}}}
\newcommand{\rhat}{\vb{\hat{r}}}
\newcommand{\phihat}{\bm{\hat{\phi}}}
\newcommand{\thetahat}{\bm{\hat{\theta}}}

%%% Other math stuff
\providecommand{\units}[1]{\,\ensuremath{\mathrm{#1}}\xspace}
% Set new style for problem
\newtheoremstyle{problemstyle}  % <name>
        {10pt}                   % <space above>
        {10pt}                   % <space below>
        {\normalfont}           % <body font>
        {}                      % <indent amount}
        {\bfseries\itshape}     % <theorem head font>
        {\normalfont\bfseries:} % <punctuation after theorem head>
        {.5em}                  % <space after theorem head>
        {}                      % <theorem head spec (can be left empty, 
                                % meaning `normal')>

% Set problem environment
\theoremstyle{problemstyle}
\newtheorem{problemenv}{Problem}[section]
\newenvironment{problem}[1]{%
  \renewcommand\theproblemenv{#1}%
  \problemenv
}{\endproblemenv}
% Set lemma environment
\newenvironment{lemma}[2][Lemma]{\begin{trivlist}
\item[\hskip \labelsep {\bfseries #1}\hskip \labelsep {\bfseries #2.}]}{\end{trivlist}}
% Set solution environment
\newenvironment{solution}{
    \begin{proof}[Solution]$ $\par\nobreak\ignorespaces
}{\end{proof}}
\numberwithin{equation}{problemenv}

%%% Page format
\setlength{\parindent}{0.5cm}
\setlength{\oddsidemargin}{0in}
\setlength{\textwidth}{6.5in}
\setlength{\textheight}{8.8in}
\setlength{\topmargin}{0in}
\setlength{\headheight}{18pt}

%%% Code environments
\definecolor{dkgreen}{rgb}{0,0.6,0}
\definecolor{gray}{rgb}{0.5,0.5,0.5}
\definecolor{mauve}{rgb}{0.58,0,0.82}
\lstset{frame=tb,
  language=Python,
  aboveskip=3mm,
  belowskip=3mm,
  showstringspaces=false,
  columns=flexible,
  basicstyle={\small\ttfamily},
  numbers=none,
  numberstyle=\tiny\color{gray},
  keywordstyle=\color{blue},
  commentstyle=\color{dkgreen},
  stringstyle=\color{mauve},
  breaklines=true,
  breakatwhitespace=true,
  tabsize=4
}
\lstset{
  language=Mathematica,
  numbers=left,
  numberstyle=\tiny\color{gray},
  numbersep=5pt,
  breaklines=true,
  captionpos={t},
  frame={lines},
  rulecolor=\color{black},
  framerule=0.5pt,
  columns=flexible,
  tabsize=2
}


\title{Homework 10: Phys 5210 (Fall 2021)}

\date{November 22, 2021}

\begin{document}
\maketitle
%%%%%%%%%%%%%%%%%%%%%%%%%%%%%%%%%%%%%%%%%%%%%%%%%%%%%%%%%%%%%%%%%%%%%%%%%%%%%%%
\begin{problem}{1}[Goldstein, 9.30]
(a) Prove that the Poisson bracket of two constants of the motion is itself a
constant of motion even if the constants are explicitly dependent on time.
Constants of motions are two functions of position, momenta and time, $u(q,p,t)$
and $v(q,p,t)$ such that
\begin{equation}\label{p1:ddt}
    \frac{du}{dt}=\qty{\HH,u}+\frac{\partial u}{\partial t}=0,\qquad
    \frac{dv}{dt}=\qty{\HH,v}+\frac{\partial v}{\partial t}=0
\end{equation}

(b) Show that if both the Hamiltonian $\HH$ and a quantity $u(q,p,t)$ are
constants of motion, the $n$th partial derivative of $u$ with respect to time
must also be a constant of motion.

(An illustration of this result is the motion of a free particle with
$\HH=p^2/2m)$ where there exists a constant of motion $x-pt/m$, whose derivative
with respect to time $t$ is also a constant of motion.
\begin{solution}
(a) First, since $\qty{u,v}$ is also a function of $q,p,$ and $t$, we can write
\begin{align}
    \frac{d}{dt}\qty{u,v}=\qty{\HH,\qty{u,v}}+\frac{\partial}{\partial
    t}\qty{u,v}
\end{align}
Now, the Poisson bracket has a cyclic property
\begin{align}
    \qty{\HH,\qty{u,v}}+\qty{v,\qty{\HH,u}}-\qty{u,\qty{\HH,v}}=0
\end{align}
So we can write
\begin{align}
    \qty{\HH,\qty{u,v}}
    =\qty{u,\qty{\HH,v}}-\qty{v,\qty{\HH,u}}
    =\qty{v,\frac{\partial u}{\partial t}}-\qty{u,\frac{\partial v}{\partial t}}
\end{align}
from \eqref{p1:ddt}. Now, by definition of the Poisson bracket,
\begin{align}
    \frac{\partial}{\partial t}\qty{u,v}
    &=\frac{\partial}{\partial q}\qty(\frac{\partial u}{\partial
    t})\frac{\partial v}{\partial p}+\frac{\partial u}{\partial
q}\frac{\partial}{\partial p}\qty(\frac{\partial v}{\partial t})
-\frac{\partial}{\partial p}\qty(\frac{\partial u}{\partial t})\frac{\partial
v}{\partial q}-\frac{\partial u}{\partial p}\frac{\partial}{\partial
q}\qty(\frac{\partial v}{\partial t})\notag\\
    &=\qty{\frac{\partial u}{\partial t},v}+\qty{u,\frac{\partial v}{\partial t}}
\end{align}
Thus,
\begin{align}
    \frac{d}{dt}\qty{u,v}=\qty{v,\frac{\partial u}{\partial t}}
    -\qty{u,\frac{\partial v}{\partial t}}-\qty{v,\frac{\partial u}{\partial
    t}}+\qty{u,\frac{\partial v}{\partial t}}=0
\end{align}

(b) Given that $d\HH/dt=\partial\HH/\partial t=0$ and $du/dt=0$, we can
write
\begin{align}
    \frac{d}{dt}\qty(\frac{\partial^nu}{\partial t^n})
    &=\qty{\HH,\frac{\partial^nu}{\partial t^n}}+\frac{\partial^{n+1}}{\partial
    t^{n+1}}u\notag\\
    &=\frac{\partial^n}{\partial t^n}\qty{\HH,u}+\frac{\partial^{n+1}}{\partial
    t^{n+1}}u\notag\\
    &=-\frac{\partial^{n+1}}{\partial t^{n+1}}u+\frac{\partial^{n+1}}{\partial
    t^{n+1}}u\notag\\
    &=0
\end{align}
\end{solution}
\end{problem}
%%%%%%%%%%%%%%%%%%%%%%%%%%%%%%%%%%%%%%%%%%%%%%%%%%%%%%%%%%%%%%%%%%%%%%%%%%%%%%%    
%%%%%%%%%%%%%%%%%%%%%%%%%%%%%%%%%%%%%%%%%%%%%%%%%%%%%%%%%%%%%%%%%%%%%%%%%%%%%%%
\begin{problem}{2}[Goldstein, 10.16]
A particle of mass $m$ is constrained to move on a curve in the vertical plane
defined by the parametric equations
\begin{equation}\label{p2:xy}
    x=l(2\phi+\sin(2\phi)),\qquad
    y=l(1-\cos(2\phi))
\end{equation}
There's the usual constant gravitational force acting in the vertical $y$
direction. By the method of action-angle variable, find the frequency of
oscillations for all initial conditions such that the maximum of $\phi$ is less
than or equal to $\pi/2$.
\begin{solution}
Given \eqref{p2:xy}, the velocity is
\begin{equation}
    \dot{x}=2l(1+\cos2\phi)\dot\phi\qquad\text{and}\qquad
    \dot{y}=2l\sin2\phi \dot\phi
\end{equation}
Then the Lagrange function is
\begin{align}
    \LL&=2ml^2\dot\phi^2\qty[(1+\cos2\phi)^2+\sin^2\phi]-mgl(1-\cos2\phi)\notag\\
       &=4ml^2\dot\phi^2\qty(1+\cos2\phi)-mgl(1-\cos2\phi)\notag\\
       &=8ml^2\dot\phi^2\cos^2\phi-2mgl\sin^2\phi
\end{align}
The canonical momentum is then
\begin{equation}
    p=\frac{\partial\LL}{\partial\dot\phi}=12ml^2\dot\phi\cos^2\phi 
\end{equation}
Then the Hamiltonian is
\begin{align}
    \HH
    &=p\dot\phi-8ml^2\dot\phi^2\cos^2\phi+2mgl\sin^2\phi\notag\\
    &=\frac{p^2}{16ml^2\cos^2\phi}-8ml^2\cos^2\phi\frac{p^2}{256m^2l^4\cos^4\phi}+2mgl\sin^2\phi\notag\\
    &=\frac{p^2}{32ml^2\cos^2\phi}+2mgl\sin^2\phi
\end{align}
Since $\HH$ is time-independent, we can also write $\HH=E=2mgl$ where $E$ is the
energy at the point where $\phi=\phi_{\max}=\pi/2$ and $\dot\phi=0$. Then,
define the new adiabatic invariant
\begin{align}
    P&=4\int_0^{\pi/2} pd\phi\notag\\
     &=4\int_0^{\pi/2} d\phi\sqrt{32ml^2\cos^2\phi(E-2mgl\sin^2\phi)}\notag\\
     &=4\sqrt{32ml^2E}\int_0^{\pi/2}d\phi\cos\phi\sqrt{1-\sin^2\phi}\notag\\
     &=4\sqrt2\pi\sqrt{ml^2E}
\end{align}
Then it follows that
\begin{equation}
    \HH=\frac1{32\pi^2}\frac{P^2}{ml^2}\qquad\text{and}\qquad
    P=8\pi ml\sqrt{gl}
\end{equation}
Then the new coordinate $\Phi$ conjugate to $P$ is cyclic and the frequency is
\begin{equation}
    \dot\Phi=\frac{\partial\HH}{\partial P} 
    =\frac1{16\pi^2}\frac{P}{ml^2}=\frac1{2\pi}\sqrt{\frac{g}{l}}
\end{equation}
\end{solution}
\end{problem}
%%%%%%%%%%%%%%%%%%%%%%%%%%%%%%%%%%%%%%%%%%%%%%%%%%%%%%%%%%%%%%%%%%%%%%%%%%%%%%%
\end{document}
