\documentclass[12pt]{article}

%%%%%%%%%%%%%%%%%%%%%%%%%%%%%%%%%%%%%%%%%%%%%%%%%%%%%%%%%%%%%%%%%%%%%%%%%%%%%%%%
%                           Package preset for homework
%%%%%%%%%%%%%%%%%%%%%%%%%%%%%%%%%%%%%%%%%%%%%%%%%%%%%%%%%%%%%%%%%%%%%%%%%%%%%%%%
% Miscellaneous
\usepackage[margin=1in]{geometry}
\usepackage[utf8]{inputenc}
\usepackage{indentfirst}
\usepackage{blindtext}
\usepackage{graphicx}
\usepackage{xr-hyper}
\usepackage{hyperref}
\usepackage{color}
\usepackage{float}
% Math
\usepackage{latexsym}
\usepackage{amsfonts}
\usepackage{amssymb}
\usepackage{amsmath}
\usepackage{commath}
\usepackage{amsthm}
\usepackage{bbold}
\usepackage{bm}
% Physics
\usepackage{physics}
\usepackage{siunitx}
% Code typesetting
\usepackage{listings}
% Citation
\usepackage[authoryear]{natbib}
\usepackage{appendix}
\usepackage[capitalize]{cleveref}
% Title & name
\title{Homework}
\author{Tien Vo}
\date{\today}


%%%%%%%%%%%%%%%%%%%%%%%%%%%%%%%%%%%%%%%%%%%%%%%%%%%%%%%%%%%%%%%%%%%%%%%%%%%%%%%%
%                   User-defined commands and environments
%%%%%%%%%%%%%%%%%%%%%%%%%%%%%%%%%%%%%%%%%%%%%%%%%%%%%%%%%%%%%%%%%%%%%%%%%%%%%%%%
%%% Misc
\sisetup{load-configurations=abbreviations}
\newcommand{\due}[1]{\date{Due: #1}}
\newcommand{\hint}{\textit{Hint}}
\let\oldt\t
\renewcommand{\t}[1]{\text{#1}}

%%% Bold sets & abbrv
\newcommand{\N}{\mathbb{N}}
\newcommand{\Z}{\mathbb{Z}}
\newcommand{\R}{\mathbb{R}}
\newcommand{\Q}{\mathbb{Q}}
\let\oldP\P
\renewcommand{\P}{\mathbb{P}}
\newcommand{\LL}{\mathcal{L}}
\newcommand{\FF}{\mathcal{F}}
\newcommand{\HH}{\mathcal{H}}
\newcommand{\NN}{\mathcal{N}}
\newcommand{\ZZ}{\mathcal{Z}}
\newcommand{\RN}[1]{\textup{\uppercase\expandafter{\romannumeral#1}}}
\newcommand{\ua}{\uparrow}
\newcommand{\da}{\downarrow}

%%% Unit vectors
\newcommand{\xhat}{\vb{\hat{x}}}
\newcommand{\yhat}{\vb{\hat{y}}}
\newcommand{\zhat}{\vb{\hat{z}}}
\newcommand{\nhat}{\vb{\hat{n}}}
\newcommand{\rhat}{\vb{\hat{r}}}
\newcommand{\phihat}{\bm{\hat{\phi}}}
\newcommand{\thetahat}{\bm{\hat{\theta}}}

%%% Other math stuff
\providecommand{\units}[1]{\,\ensuremath{\mathrm{#1}}\xspace}
% Set new style for problem
\newtheoremstyle{problemstyle}  % <name>
        {10pt}                   % <space above>
        {10pt}                   % <space below>
        {\normalfont}           % <body font>
        {}                      % <indent amount}
        {\bfseries\itshape}     % <theorem head font>
        {\normalfont\bfseries:} % <punctuation after theorem head>
        {.5em}                  % <space after theorem head>
        {}                      % <theorem head spec (can be left empty, 
                                % meaning `normal')>

% Set problem environment
\theoremstyle{problemstyle}
\newtheorem{problemenv}{Problem}[section]
\newenvironment{problem}[1]{%
  \renewcommand\theproblemenv{#1}%
  \problemenv
}{\endproblemenv}
% Set lemma environment
\newenvironment{lemma}[2][Lemma]{\begin{trivlist}
\item[\hskip \labelsep {\bfseries #1}\hskip \labelsep {\bfseries #2.}]}{\end{trivlist}}
% Set solution environment
\newenvironment{solution}{
    \begin{proof}[Solution]$ $\par\nobreak\ignorespaces
}{\end{proof}}
\numberwithin{equation}{problemenv}

%%% Page format
\setlength{\parindent}{0.5cm}
\setlength{\oddsidemargin}{0in}
\setlength{\textwidth}{6.5in}
\setlength{\textheight}{8.8in}
\setlength{\topmargin}{0in}
\setlength{\headheight}{18pt}

%%% Code environments
\definecolor{dkgreen}{rgb}{0,0.6,0}
\definecolor{gray}{rgb}{0.5,0.5,0.5}
\definecolor{mauve}{rgb}{0.58,0,0.82}
\lstset{frame=tb,
  language=Python,
  aboveskip=3mm,
  belowskip=3mm,
  showstringspaces=false,
  columns=flexible,
  basicstyle={\small\ttfamily},
  numbers=none,
  numberstyle=\tiny\color{gray},
  keywordstyle=\color{blue},
  commentstyle=\color{dkgreen},
  stringstyle=\color{mauve},
  breaklines=true,
  breakatwhitespace=true,
  tabsize=4
}
\lstset{
  language=Mathematica,
  numbers=left,
  numberstyle=\tiny\color{gray},
  numbersep=5pt,
  breaklines=true,
  captionpos={t},
  frame={lines},
  rulecolor=\color{black},
  framerule=0.5pt,
  columns=flexible,
  tabsize=2
}


\title{Homework 9: Phys 5210 (Fall 2021)}

\begin{document}
\maketitle
%%%%%%%%%%%%%%%%%%%%%%%%%%%%%%%%%%%%%%%%%%%%%%%%%%%%%%%%%%%%%%%%%%%%%%%%%%%%%%%%
\begin{problem}{1}
In atomic physics experiments such as the ones done in JILA, a cloud of
interacting atoms is sometimes placed in a rotating external potential. It is
convenient to study these atoms in a reference frame which rotates together
with the potential where their potential energy does not depend on time. As we
know the velocity of a particle at a position $\vb{r}$ in this rotating
reference frame is related to the velocity of the same particle in the
stationary reference frame by $\vb{v}_0=\vb{v}+\bm\Omega\times\vb{r}$, where
$\bm\Omega$ is the constant angular velocity of rotation, $\vb{v}$ is the
velocity of the particle in the rotating reference frame and $\vb{v}_0$ is the
velocity in the stationary reference frame. The Lagrange function in the
rotating reference frame of a system of particles, each with mass $\vb{m}$,
labelled by the index $j=1,2,\hdots,N$, is then given by
\begin{equation}\label{p1:L}
    \LL=\sum_{j=1}^N\frac{mv_{0j}^2}{2}-U(\vb{r})
    =\sum_{j=1}^N\qty(\frac{mv_j^2}{2}+m\vb{v}_j\vdot\qty(\bm\Omega\times\vb{r}_j)+\frac{m(\bm\Omega\times\vb{r}_j)^2}{2})-U(\vb{r})
\end{equation}
Find the Hamiltonian of this system in the rotating reference frame.
\begin{solution}
Given the Lagrange function \eqref{p1:L}, the canonical momenta are
\begin{align}
    \vb{p}_j=\frac{\partial\LL}{\partial\vb{v}_j}
            =m\vb{v}_j+m\qty(\bm\Omega\times\vb{r}_j)
    \Rightarrow\vb{v}_j=\frac{\vb{p}_j}{m}-\bm\Omega\times\vb{r}_j
\end{align}
where $\vb{v}_j=\dot{\vb{r}}_j$. Then we can write the Hamiltonian as
\begin{align}
    \HH
    &=\sum_{j=1}^N\vb{p}_j\vdot \vb{v}_j-\LL\notag\\
    &=\sum_{j=1}^N\Bigg[
    \frac{p_j^2}{m}-\vb{p}_j\vdot\qty(\bm\Omega\times\vb{r}_j)\notag\\
    &\qquad-\frac{m}{2}\qty(\frac{\vb{p}_j}{m}-\bm\Omega\times\vb{r}_j)^2
    -\qty(\vb{p}_j-m\bm\Omega\times\vb{r}_j)\vdot\qty(\bm\Omega\times\vb{r}_j)
    -\frac{m}{2}\qty(\bm\Omega\times\vb{r}_j)^2
    \Bigg]+U(\vb{r})\notag\\
    &=\sum_{j=1}^N\qty[\frac{p_j^2}{2m}-\vb{p}_j\vdot\qty(\bm\Omega\times\vb{r}_j)]+U(\vb{r})
\end{align}
\end{solution}
\end{problem}
%%%%%%%%%%%%%%%%%%%%%%%%%%%%%%%%%%%%%%%%%%%%%%%%%%%%%%%%%%%%%%%%%%%%%%%%%%%%%%%%
%%%%%%%%%%%%%%%%%%%%%%%%%%%%%%%%%%%%%%%%%%%%%%%%%%%%%%%%%%%%%%%%%%%%%%%%%%%%%%%%
\begin{problem}{2}
Inspired by \textit{Goldstein}, Chapter 8, Problem 35. Consider a system with
this Lagrangian
\begin{equation}\label{p2:L}
    \LL=\frac{m}{2}\qty(\dot{x}^2-\omega^2x^2)e^{2\gamma t} 
\end{equation}
where $\gamma>0,\omega\neq 0, m>0$.

(a) Write down the Euler-Lagrange equation and solve it with arbitrary initial
conditions at $t=0$.

(b) Find the Hamiltonian, write down the Hamilton equations of motion and
verify that they are equivalent to the Euler-Lagrange equation.

(c) For arbitrary initial conditions such as in part (a), how do the position
$x$, the momentum $p$, and the Hamiltonian $\HH$ of this system behave as
$t\to\infty$?
\begin{solution}
(a) From the Euler-Lagrange equation for $x$,
\begin{align}
    \frac{d}{dt}\qty[m\dot{x}e^{2\gamma t}]=-m\omega^2xe^{2\gamma t}
    \Rightarrow \ddot{x}+2\gamma\dot{x}+\omega^2x=0
\end{align}
A guess to the solution of this differential equation is $x=Ce^{i\Omega t}$ for
some $\Omega$. Plugging back yields
\begin{equation}
    -\Omega^2+2i\gamma\Omega+\omega^2=0\Rightarrow\Omega_\pm=i\gamma\pm\sqrt{\omega^2-\gamma^2}
\end{equation}
Since there are two solutions to $\Omega$, the general solution for is spanned 
by a linear combination
\begin{equation}
    x(t)=C_1e^{i\Omega_+t}+C_2e^{i\Omega_-t}
    =e^{-\gamma
    t}\qty[C_1e^{i\sqrt{\omega^2-\gamma^2}t}+C_2e^{-i\sqrt{\omega^2-\gamma^2}t}]
\end{equation}
where $C_1,C_2$ are some constants dependent on initial conditions. The physical
solution is the real part of this
\begin{equation}
    x(t)=e^{-\gamma
    t}\qty[C_1\cos\qty(\sqrt{\omega^2-\gamma^2}t)+C_2\sin\qty(\sqrt{\omega^2-\gamma^2}t)] 
\end{equation}
Thus, given arbitrary conditions $x(0)=x_0$ and $\dot{x}(0)=v_0$, we can write
the real constants $C_1,C_2$ as
\begin{equation}
    C_1=x_0\qquad\text{and}\qquad
    C_2=\frac{v_0+\gamma x_0}{\sqrt{\omega^2-\gamma^2}}
\end{equation}
Then the final form of the solution, given initial conditions, is
\begin{equation}\label{p2a:x}
    x(t)=e^{-\gamma
    t}\qty[x_0\cos\qty(\sqrt{\omega^2-\gamma^2}t)+\frac{v_0+\gamma
x_0}{\sqrt{\omega^2-\gamma^2}}\sin\qty(\sqrt{\omega^2-\gamma^2}t)] 
\end{equation}

(b) From \eqref{p2:L}, the canonical momentum is
\begin{equation}
    p=\frac{\partial\LL}{\partial\dot{x}}
    =m\dot{x}e^{2\gamma t}
\end{equation}
Then we can write the Hamiltonian as
\begin{equation}\label{p2b:H}
    \HH=p\dot{x}-\LL 
    =\frac{p^2}{m}e^{-2\gamma t}-\frac{m}{2}\qty(\frac{p}{m}e^{-2\gamma
    t})^2e^{2\gamma t}+\frac12m\omega^2x^2e^{2\gamma t}
    =\frac{p^2}{2m}e^{-2\gamma t}+\frac12m\omega^2x^2e^{2\gamma t}
\end{equation}
The Hamiltonian equation of motion is thus
\begin{subequations}
    \begin{align}
        \dot{x}&=\frac{\partial\HH}{\partial p}=\frac{p}{m}e^{-2\gamma
        t}\label{p2:xdot}\\
        \dot{p}&=-\frac{\partial\HH}{\partial x}
        =-m\omega^2xe^{2\gamma t}\label{p2:pdot}
    \end{align} 
\end{subequations}
Taking the time derivative of \eqref{p2:xdot} one more time and substituting in
\eqref{p2:pdot}, we get
\begin{equation}
    \ddot{x}=\frac{\dot{p}}{m}e^{-2\gamma t}-2\gamma\frac{p}{m}e^{-2\gamma t} 
    =-\omega^2x-2\gamma\dot{x}\Rightarrow
    \ddot{x}+2\gamma\dot{x}+\omega^2x=0
\end{equation}
This is the same differential equation obtained from the Euler-Lagrange
equation. Thus, the Hamiltonian equation of motion is equivalent to that in the
Lagrangian formulation.

(c) From \eqref{p2a:x}, $x\to0$ as $t\to\infty$. Also, the velocity is
\begin{equation}
    \dot{x}(t)=-\gamma x(t)+e^{-\gamma
    t}\qty[-x_0\sqrt{\omega^2-\gamma^2}\cos\qty(\sqrt{\omega^2-\gamma^2}t)+\qty(v_0+\gamma
    x_0)\cos\qty(\sqrt{\omega^2-\gamma^2}t)]
\end{equation}
Thus, the mechanical momentum $p_\text{mech}=m\dot{x}\to0$ as $t\to\infty$.
However, the canonical momentum $p=p_\text{mech}e^{2\gamma t}\sim e^{\gamma
t}\to\infty$ as $t\to\infty$. Also, since $x(t)=e^{-\gamma t}f(t)$ and 
$p(t)=e^{\gamma t}g(t)$ where $f,g$ are bounded functions, from
\eqref{p2b:H}, we can conclude that $\HH$ is bounded. So $\HH$ doesn't blow up
where $t\to\infty$.
\end{solution}
\end{problem}
%%%%%%%%%%%%%%%%%%%%%%%%%%%%%%%%%%%%%%%%%%%%%%%%%%%%%%%%%%%%%%%%%%%%%%%%%%%%%%%%
    
\end{document}
