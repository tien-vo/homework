\documentclass[12pt]{article}

%%%%%%%%%%%%%%%%%%%%%%%%%%%%%%%%%%%%%%%%%%%%%%%%%%%%%%%%%%%%%%%%%%%%%%%%%%%%%%%%
%                           Package preset for homework
%%%%%%%%%%%%%%%%%%%%%%%%%%%%%%%%%%%%%%%%%%%%%%%%%%%%%%%%%%%%%%%%%%%%%%%%%%%%%%%%
% Miscellaneous
\usepackage[margin=1in]{geometry}
\usepackage[utf8]{inputenc}
\usepackage{indentfirst}
\usepackage{blindtext}
\usepackage{graphicx}
\usepackage{xr-hyper}
\usepackage{hyperref}
\usepackage{enumitem}
\usepackage{color}
\usepackage{float}
% Math
\usepackage{latexsym}
\usepackage{amsfonts}
\usepackage{amssymb}
\usepackage{amsmath}
\usepackage{commath}
\usepackage{amsthm}
\usepackage{bbold}
\usepackage{bm}
% Physics
\usepackage{physics}
\usepackage{siunitx}
% Code typesetting
\usepackage{listings}
% Citation
\usepackage[authoryear]{natbib}
\usepackage{appendix}
\usepackage[capitalize]{cleveref}
% Title & name
\title{Homework}
\author{Tien Vo}
\date{\today}


%%%%%%%%%%%%%%%%%%%%%%%%%%%%%%%%%%%%%%%%%%%%%%%%%%%%%%%%%%%%%%%%%%%%%%%%%%%%%%%%
%                   User-defined commands and environments
%%%%%%%%%%%%%%%%%%%%%%%%%%%%%%%%%%%%%%%%%%%%%%%%%%%%%%%%%%%%%%%%%%%%%%%%%%%%%%%%
%%% Misc
\sisetup{load-configurations=abbreviations}
\newcommand{\due}[1]{\date{Due: #1}}
\newcommand{\hint}{\textit{Hint}}
\let\oldt\t
\renewcommand{\t}[1]{\text{#1}}

%%% Bold sets & abbrv
\newcommand{\N}{\mathbb{N}}
\newcommand{\Z}{\mathbb{Z}}
\newcommand{\R}{\mathbb{R}}
\newcommand{\Q}{\mathbb{Q}}
\let\oldP\P
\renewcommand{\P}{\mathbb{P}}
\newcommand{\LL}{\mathcal{L}}
\newcommand{\FF}{\mathcal{F}}
\newcommand{\HH}{\mathcal{H}}
\newcommand{\NN}{\mathcal{N}}
\newcommand{\ZZ}{\mathcal{Z}}
\newcommand{\RN}[1]{\textup{\uppercase\expandafter{\romannumeral#1}}}
\newcommand{\ua}{\uparrow}
\newcommand{\da}{\downarrow}

%%% Unit vectors
\newcommand{\xhat}{\vb{\hat{x}}}
\newcommand{\yhat}{\vb{\hat{y}}}
\newcommand{\zhat}{\vb{\hat{z}}}
\newcommand{\nhat}{\vb{\hat{n}}}
\newcommand{\rhat}{\vb{\hat{r}}}
\newcommand{\phihat}{\bm{\hat{\phi}}}
\newcommand{\thetahat}{\bm{\hat{\theta}}}

%%% Other math stuff
\providecommand{\units}[1]{\,\ensuremath{\mathrm{#1}}\xspace}
% Set new style for problem
\newtheoremstyle{problemstyle}  % <name>
        {10pt}                   % <space above>
        {10pt}                   % <space below>
        {\normalfont}           % <body font>
        {}                      % <indent amount}
        {\bfseries\itshape}     % <theorem head font>
        {\normalfont\bfseries:} % <punctuation after theorem head>
        {.5em}                  % <space after theorem head>
        {}                      % <theorem head spec (can be left empty, 
                                % meaning `normal')>

% Set problem environment
\theoremstyle{problemstyle}
\newtheorem{problemenv}{Problem}[section]
\newenvironment{problem}[1]{%
  \renewcommand\theproblemenv{#1}%
  \problemenv
}{\endproblemenv}
% Set lemma environment
\newenvironment{lemma}[2][Lemma]{\begin{trivlist}
\item[\hskip \labelsep {\bfseries #1}\hskip \labelsep {\bfseries #2.}]}{\end{trivlist}}
% Set solution environment
\newenvironment{solution}{
    \begin{proof}[Solution]$ $\par\nobreak\ignorespaces
}{\end{proof}}
\numberwithin{equation}{problemenv}

%%% Page format
\setlength{\parindent}{0.5cm}
\setlength{\oddsidemargin}{0in}
\setlength{\textwidth}{6.5in}
\setlength{\textheight}{8.8in}
\setlength{\topmargin}{0in}
\setlength{\headheight}{18pt}

%%% Code environments
\definecolor{dkgreen}{rgb}{0,0.6,0}
\definecolor{gray}{rgb}{0.5,0.5,0.5}
\definecolor{mauve}{rgb}{0.58,0,0.82}
\lstset{frame=tb,
  language=Python,
  aboveskip=3mm,
  belowskip=3mm,
  showstringspaces=false,
  columns=flexible,
  basicstyle={\small\ttfamily},
  numbers=none,
  numberstyle=\tiny\color{gray},
  keywordstyle=\color{blue},
  commentstyle=\color{dkgreen},
  stringstyle=\color{mauve},
  breaklines=true,
  breakatwhitespace=true,
  tabsize=4
}
\lstset{
  language=Mathematica,
  numbers=left,
  numberstyle=\tiny\color{gray},
  numbersep=5pt,
  breaklines=true,
  captionpos={t},
  frame={lines},
  rulecolor=\color{black},
  framerule=0.5pt,
  columns=flexible,
  tabsize=2
}


\title{Homework 4: Phys 5210 (Fall 2021)}

\begin{document}
\maketitle

%%%%%%%%%%%%%%%%%%%%%%%%%%%%%%%%%%%%%%%%%%%%%%%%%%%%%%%%%%%%%%%%%%%%%%%%%%%%%%%%
\begin{problem}{1}
Two particles move about each other in circular orbits under the influence of
gravitational forces (so that the potential energy is $U=-\alpha / r$), with a
period $\tau$.

Their motion is suddenly stopped at a given instant of time, and they are then
released and allowed to fall into each other.

Prove that they collide after a time $\tau /(4\sqrt{2})$.
\begin{solution}
From the general solution derived in class, a circular orbit has a radius
\begin{equation}
    R=\frac1{x_+}=\frac1{x_-}=\frac{l^2}{\mu\alpha} 
\end{equation}
where $l=\mu R^2\omega$ is the angular momentum. Then the period is
\begin{equation}
    \tau=\frac{2\pi}{\omega}=2\pi\frac{l^3}{\mu\alpha^2} 
\end{equation}
Now, the particle is set to free fall with an intial energy
$E_0=-\alpha /R$. By energy conservation,
\begin{equation}
    \frac12\mu\dot{r}^2-\frac{\alpha}{r}=E_0=-\frac{\alpha}{R} 
\end{equation}
Thus, we can write the differential equation
\begin{equation}
    dt=\sqrt{\frac{\mu}{2\alpha}}\qty(\frac1r-\frac1R)^{-1/2}dr
\end{equation}
Integrating both sides, we can find the time for the particles collide as
\begin{equation}
    \tau'=\sqrt{\frac{\mu}{2\alpha}}\int_0^R\qty(\frac1r-\frac1R)^{-1/2}dr 
    =\frac\pi2\sqrt{\frac{\mu}{2\alpha}}R^{3/2}=\frac\pi{2\sqrt2}\frac{l^3}{\mu\alpha^2}=\frac{\tau}{4\sqrt2}
\end{equation}
\end{solution}
\end{problem}
%%%%%%%%%%%%%%%%%%%%%%%%%%%%%%%%%%%%%%%%%%%%%%%%%%%%%%%%%%%%%%%%%%%%%%%%%%%%%%%%
%%%%%%%%%%%%%%%%%%%%%%%%%%%%%%%%%%%%%%%%%%%%%%%%%%%%%%%%%%%%%%%%%%%%%%%%%%%%%%%%
\begin{problem}{2}
A central force potential is given by
\begin{equation}
    U(r)=\begin{cases}
        0, & r>a\\
        -U_0, & r<a
    \end{cases}
\end{equation}
Here $U_0>0$. We would like to study the scattering in this potential.

(a) If a particle approaches the potential with the kinetic energy $E$ and the
impact parameter $s$, find the angle $\varphi$ and $\tilde{\varphi}$ as shown in
the Figure above. The figure represents the potential as a circle of radius $a$,
with the trajectory of a particle shown in blue. Use conservation of energy and
conservation of angular momentum to find $\tilde{\varphi}$.

(b) Determine the scattering angle $\theta$ in terms of $\varphi$ and
$\tilde{\varphi}$.

(c) Express $s$ as a function of $\theta$. This step is tricky as it is easy to
find $\theta$ in terms of $s$, harder the other way around. To do that, take the
ratio $\sin\frac{\tilde{\varphi}}{\sin\varphi}$. On the one hand, use the
results of part (a) to figure out what it is equal to. On the other hand,
express $\tilde{\varphi}$ in terms of $\varphi$ and $\theta$, and work to solve
the resulting equation for $\varphi$. Finally, use the previous result relating
$\varphi$ to $s$.

(d) Finally, find the differential cross section $d\sigma /d\Omega$ where
$d\Omega=2\pi\sin(\theta)d\theta$ for a particle scattering in this potential.
\begin{solution}
(a) The angular momentum is defined as $\vb{L}=\vb{r}\times\vb{p}$. At the point
of impact (when the particle enters the potential field $U$), the radius is 
$r=a$. Also, let $v_0$ be the initial velocity. Then the initial angular 
momentum is
\begin{equation}
    L=mav_0\sin\varphi 
\end{equation}
Let $v$ be the velocity briefly after it has entered $U$. The radius is still
$a$, but the angle between $\vb{r}$ and $\vb{p}$ is $\tilde\varphi$ and the
angular momentum is
\begin{equation}
    L=mav\sin\varphi 
\end{equation}
From conservation of the angular momentum, we can write
\begin{equation}
    v=v_0\frac{\sin\varphi}{\sin\tilde\varphi} 
\end{equation}
From energy conservation and the above result, we can write
\begin{equation}
    E_0=\frac12mv_0^2=\frac12mv^2-U_0=\frac12mv_0^2\frac{\sin^2\varphi}{\sin^2\tilde\varphi}-U_0
\end{equation}
Inverting, we can write
\begin{equation}\label{p2:a}
    \sin\tilde\varphi=\qty(1+\frac{U_0}{E_0})^{-1/2}\sin\varphi 
\end{equation}
From geometry, $\sin\varphi=s /a$, so we can write $\tilde\varphi$ as
\begin{equation}
    \tilde\varphi=\sin^{-1}\qty[\frac{s}{a}\qty(1+\frac{U_0}{E_0})^{-1/2}] 
\end{equation}

(b) From geometry, we can also write
$\theta=2(\varphi-\tilde\varphi)\Rightarrow\tilde\varphi=\varphi-\theta /2$.

(c) From part (b), we can rewrite \eqref{p2:a} as
\begin{equation}
    \sin\qty(\varphi-\frac\theta2)
    =\sin\varphi\cos\qty(\frac\theta2)-\cos\varphi\sin\qty(\frac\theta2)
    =\qty(1+\frac{U_0}{E_0})^{-1/2}\sin\varphi
\end{equation}
Solving for $\varphi$, we have
\begin{equation}
    \tan\varphi=f(\theta)=\frac{\sin(\theta
    /2)}{\cos(\theta/2)-(1+U_0/E_0)^{-1/2}}=\frac{s}{\sqrt{a^2-s^2}}
\end{equation}
Inverting, we can write $s$ as a function of $\theta$
\begin{equation}\label{p2:c}
    s^2=a^2\frac{f^2(\theta)}{1+f^2(\theta)} 
    =a^2\frac{\sin^2(\theta/2)}{1+C^2-2C\cos(\theta/2)}
\end{equation}
where $C=\sqrt{E_0 /(E_0+U)}$.  

(d) Taking the derivative of \eqref{p2:c} with Mathematica, we can write the
differential cross section as
\begin{equation}
    \sigma_{\text{diff}}
    =\frac1{2\sin\theta}\frac{ds^2}{d\theta}
    =\frac{a^2}{4}\frac{1+C^2-C[\cos(\theta/2)+\sec(\theta/2)]}{[1+C^2-2C\cos(\theta/2)]^2}
\end{equation}
Let $n=1 /C$, this can be simplified into
\begin{align}
    \sigma_{\text{diff}}
    &=\frac{a^2}{4}\frac{1+\frac1{n^2}-\frac1n\qty[\cos(\theta/2)+\sec(\theta/2)]}{[1+\frac1{n^2}-\frac2n\cos(\theta/2)]^2}\notag\\
    &=\frac{n^2a^2}{4}\frac{n^2+1-n\cos(\theta/2)-\frac{n}{\cos(\theta/2)}}{[n^2+1-2n\cos(\theta/2)]^2}\notag\\
    &=\frac{n^2a^2}{4}\frac{(n\cos(\theta/2)-1)(n-\cos(\theta/2))}{(n^2+1-2n\cos(\theta/2))^2}
\end{align}
\end{solution}
\end{problem}
%%%%%%%%%%%%%%%%%%%%%%%%%%%%%%%%%%%%%%%%%%%%%%%%%%%%%%%%%%%%%%%%%%%%%%%%%%%%%%%%
\end{document}
