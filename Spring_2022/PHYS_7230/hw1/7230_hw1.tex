\documentclass[12pt]{article}

%%%%%%%%%%%%%%%%%%%%%%%%%%%%%%%%%%%%%%%%%%%%%%%%%%%%%%%%%%%%%%%%%%%%%%%%%%%%%%%%
%                           Package preset for homework
%%%%%%%%%%%%%%%%%%%%%%%%%%%%%%%%%%%%%%%%%%%%%%%%%%%%%%%%%%%%%%%%%%%%%%%%%%%%%%%%
% Miscellaneous
\usepackage[margin=1in]{geometry}
\usepackage[utf8]{inputenc}
\usepackage{indentfirst}
\usepackage{blindtext}
\usepackage{graphicx}
\usepackage{xr-hyper}
\usepackage{hyperref}
\usepackage{color}
\usepackage{float}
% Math
\usepackage{latexsym}
\usepackage{amsfonts}
\usepackage{amssymb}
\usepackage{amsmath}
\usepackage{commath}
\usepackage{amsthm}
\usepackage{bbold}
\usepackage{bm}
% Physics
\usepackage{physics}
\usepackage{siunitx}
% Code typesetting
\usepackage{listings}
% Citation
\usepackage[authoryear]{natbib}
\usepackage{appendix}
\usepackage[capitalize]{cleveref}
% Title & name
\title{Homework}
\author{Tien Vo}
\date{\today}


%%%%%%%%%%%%%%%%%%%%%%%%%%%%%%%%%%%%%%%%%%%%%%%%%%%%%%%%%%%%%%%%%%%%%%%%%%%%%%%%
%                   User-defined commands and environments
%%%%%%%%%%%%%%%%%%%%%%%%%%%%%%%%%%%%%%%%%%%%%%%%%%%%%%%%%%%%%%%%%%%%%%%%%%%%%%%%
%%% Misc
\sisetup{load-configurations=abbreviations}
\newcommand{\due}[1]{\date{Due: #1}}
\newcommand{\hint}{\textit{Hint}}
\let\oldt\t
\renewcommand{\t}[1]{\text{#1}}

%%% Bold sets & abbrv
\newcommand{\N}{\mathbb{N}}
\newcommand{\Z}{\mathbb{Z}}
\newcommand{\R}{\mathbb{R}}
\newcommand{\Q}{\mathbb{Q}}
\let\oldP\P
\renewcommand{\P}{\mathbb{P}}
\newcommand{\LL}{\mathcal{L}}
\newcommand{\FF}{\mathcal{F}}
\newcommand{\HH}{\mathcal{H}}
\newcommand{\NN}{\mathcal{N}}
\newcommand{\ZZ}{\mathcal{Z}}
\newcommand{\RN}[1]{\textup{\uppercase\expandafter{\romannumeral#1}}}
\newcommand{\ua}{\uparrow}
\newcommand{\da}{\downarrow}

%%% Unit vectors
\newcommand{\xhat}{\vb{\hat{x}}}
\newcommand{\yhat}{\vb{\hat{y}}}
\newcommand{\zhat}{\vb{\hat{z}}}
\newcommand{\nhat}{\vb{\hat{n}}}
\newcommand{\rhat}{\vb{\hat{r}}}
\newcommand{\phihat}{\bm{\hat{\phi}}}
\newcommand{\thetahat}{\bm{\hat{\theta}}}

%%% Other math stuff
\providecommand{\units}[1]{\,\ensuremath{\mathrm{#1}}\xspace}
% Set new style for problem
\newtheoremstyle{problemstyle}  % <name>
        {10pt}                   % <space above>
        {10pt}                   % <space below>
        {\normalfont}           % <body font>
        {}                      % <indent amount}
        {\bfseries\itshape}     % <theorem head font>
        {\normalfont\bfseries:} % <punctuation after theorem head>
        {.5em}                  % <space after theorem head>
        {}                      % <theorem head spec (can be left empty, 
                                % meaning `normal')>

% Set problem environment
\theoremstyle{problemstyle}
\newtheorem{problemenv}{Problem}[section]
\newenvironment{problem}[1]{%
  \renewcommand\theproblemenv{#1}%
  \problemenv
}{\endproblemenv}
% Set lemma environment
\newenvironment{lemma}[2][Lemma]{\begin{trivlist}
\item[\hskip \labelsep {\bfseries #1}\hskip \labelsep {\bfseries #2.}]}{\end{trivlist}}
% Set solution environment
\newenvironment{solution}{
    \begin{proof}[Solution]$ $\par\nobreak\ignorespaces
}{\end{proof}}
\numberwithin{equation}{problemenv}

%%% Page format
\setlength{\parindent}{0.5cm}
\setlength{\oddsidemargin}{0in}
\setlength{\textwidth}{6.5in}
\setlength{\textheight}{8.8in}
\setlength{\topmargin}{0in}
\setlength{\headheight}{18pt}

%%% Code environments
\definecolor{dkgreen}{rgb}{0,0.6,0}
\definecolor{gray}{rgb}{0.5,0.5,0.5}
\definecolor{mauve}{rgb}{0.58,0,0.82}
\lstset{frame=tb,
  language=Python,
  aboveskip=3mm,
  belowskip=3mm,
  showstringspaces=false,
  columns=flexible,
  basicstyle={\small\ttfamily},
  numbers=none,
  numberstyle=\tiny\color{gray},
  keywordstyle=\color{blue},
  commentstyle=\color{dkgreen},
  stringstyle=\color{mauve},
  breaklines=true,
  breakatwhitespace=true,
  tabsize=4
}
\lstset{
  language=Mathematica,
  numbers=left,
  numberstyle=\tiny\color{gray},
  numbersep=5pt,
  breaklines=true,
  captionpos={t},
  frame={lines},
  rulecolor=\color{black},
  framerule=0.5pt,
  columns=flexible,
  tabsize=2
}


\title{Homework 1: Phys 7230 (Spring 2022)}
\due{January 24, 2022}

\begin{document}
\maketitle
%%%%%%%%%%%%%%%%%%%%%%%%%%%%%%%%%%%%%%%%%%%%%%%%%%%%%%%%%%%%%%%%%%%%%%%%%%%%%%%
\begin{problem}{1}
Consider two (otherwise) closed systems A and B at respective temperatures $T_A$
and $T_B$ in thermal contact with each other, so that heat $Q$ can flow between
them.

Using the 2nd law of thermodynamics -- total entropy $S$ of a closed system
increases under equilibration -- show that heat $Q$ flows from the hotter to the
colder system as $A$ and $B$ come to thermal equilibrium.

\textit{Hint}: $TdS\geq Q$.

\begin{solution}
\end{solution}
\end{problem}
%%%%%%%%%%%%%%%%%%%%%%%%%%%%%%%%%%%%%%%%%%%%%%%%%%%%%%%%%%%%%%%%%%%%%%%%%%%%%%%
%%%%%%%%%%%%%%%%%%%%%%%%%%%%%%%%%%%%%%%%%%%%%%%%%%%%%%%%%%%%%%%%%%%%%%%%%%%%%%%
\begin{problem}{2}[Spin-1/2 paramagnet]
Consider a magnet of $N$ noninteracting spin-1/2 magnetic moments in an external
magnetic field $\vb{B}$, with Hamiltonian given by Zeeman energy,
\begin{equation}
    \HH=-\sum_{i=1}^N\bm\mu_i\vdot\vb{B}=-\sum_{i=1}^N\mu_BB\sigma_i
    \equiv-\sum_{i=1}^Nh\sigma_i
\end{equation}
where $\mu_B$ is Bohr magneton (carrying units of magnetic moment) and
$\sigma_i=\pm1$ labels the two Zeeman spin states of $n$th spin.

(a) For fixed dimensionless magnetization $M=N_\ua-N_\da$, (i) what is the
multiplicity $\Omega(M,N)$? (ii) What is the corresponding probability $P(M,N)$?
Check that $\sum_{M=-N}^NP(M,N)=1$.

(b) Compute the multiplicity $\Omega(E)$ for this system at a total energy $E$
and sketch/plot it as a function of full range of accessible energies. 

\textit{Hint}: Note that the magnetization $M$ is proportional to the energy
$E$.

(c) Derive the relation between temperature $T(E)$ and energy $E$ and plot
$T(E)$.

(d) Calculate the (i) magnetization $m(T,B)=\mu_B\sum_{i=1}^N\sigma_i$, (ii)
obtain its asymptotic forms in the classical $\mu_BB/k_BT\ll 1$ and quantum
$\mu_BB/k_BT\gg 1$ limits and (iii) plot it as a function of $T$ at a couple of
fixed values of $B$ and as a function of $B$ at a couple of fixed values of $T$.

\textit{Hint}: (i) Notice that magnetization is proportional to the energy $E$,
(ii) Eliminate $E$ in favor of $T$, (iii) Use lowest order Stirling
approximation throughout to simplify the factorials in your expression.

(e) Compute the linear magnetic susceptibility $\chi(T,B)=\eval{\partial
m/\partial B}_{B\to0}$, show that it exhibits Curie form
$\chi_\text{Curie}=a/T$, extracting the coefficient $a$.

(f) Compute the heat capacity (specific heat), $C_v(T)=T(\partial S/\partial
T)_{V,N}$, extract its low and high temperatures asymptotics, and sketch it,
noting its limiting forms and the crossover temperature.
\begin{solution}
\end{solution}
\end{problem}
%%%%%%%%%%%%%%%%%%%%%%%%%%%%%%%%%%%%%%%%%%%%%%%%%%%%%%%%%%%%%%%%%%%%%%%%%%%%%%%
%%%%%%%%%%%%%%%%%%%%%%%%%%%%%%%%%%%%%%%%%%%%%%%%%%%%%%%%%%%%%%%%%%%%%%%%%%%%%%%
\begin{problem}{3}[Quamtum harmonic oscillators: Einstein solid]
Consider $N$ decoupled 3D quantum harmonic oscillators as a model of atomic
vibrations in a crystalline solid (Einstein phonons), described by the familiar
quantum Hamiltonian
\begin{equation}
    \hat\HH=\sum_{i}^N\qty[\frac{\hat{p}_i^2}{2m}+\frac12m\omega_0^2\hat{r}_i^2-\frac32\hbar\omega_0] 
\end{equation}
where for convenience I defined $\hat\HH$ with zero point energy subtracted off.

(a) Let's warm up on a single harmonic oscillator, computing its degeneracy
$g(n)=\Omega(E=\hbar\omega_0n)$ a fixed total energy
$E\equiv\hbar\omega_0n=\hbar\omega_0\sum_{\alpha=1}^dn_\alpha$ (where
$n_\alpha\in\Z$ are integer quantum numbers for $\alpha=x,y,\hdots$) for the
cases of (i) 2D and (ii) 3D.

(b) Recalling the eigenvalues
$E[\qty{n_\alpha}]=\hbar\omega_0\sum_{\alpha=1}^{3N}n_\alpha$
($\alpha=x_1,y_1,z_1,x_2,y_2,z_2,\hdots$ ranging from 1 to $3N$) for the
harmonic oscillator Hamiltonian, compute the multiplicity
\begin{equation}
    \Omega(E)=\sum_{\qty{n_\alpha}}\delta_{E,E[\qty{n_\alpha}]} 
\end{equation}
taking $E=\hbar\omega_0n$ ($n\in\Z$).

\textit{Hint}: Think about how to distribute $n$ total quanta of excitations
among $3N$ 1D oscillators, and use the lowest Stirling formula approximation for
$N\gg 1$ and $n\gg 1$.

(c) Compute the entropy $S(E)$ and the corresponding $T(E)$, thereby extracting
energy $E(T)$ as a function of temperature $T$, exploring its classical
$\hbar\omega_0/k_BT\ll1$ and quantum $\hbar\omega_0/k_BT\gg1$ limiting
functional forms. Plot $E(T)$, noting limiting forms.

(d) Compute heat capacity
$C_v=T(\partial S/\partial T)_{V,N}=\partial E/\partial T$ and explore its 
classical (high $T$) and quantum (low $T$) limits, showing
the expected equipartition $C_v=N_\text{dof}k_B$ in former and its breakdon in
the latter limits. Plot $C_v(T)$, noting the crossover temperature.

(e) Consider a classical limit of the problem with small $\hbar\omega_0/k_BT$
such that $E[\qty{n_\alpha}]=\sum_{\alpha=1}^{3N}\epsilon_\alpha$ and oscillator
eigenvalues $\epsilon_\alpha$ vary nearly continuously. Using this
simplification compute
\begin{equation}
    \Omega(E)=\Delta\prod_{\alpha=1}^{3N}\int\frac{d\epsilon_\alpha}{\hbar\omega_0}
    \delta(E-E[\qty{\epsilon_\alpha}]) 
\end{equation}
as a multidimensional integral over $\epsilon_\alpha$.

\textit{Hint}: It is helpful to use a result for a hypervolume of an
$N$-dimensional space spanned by positive values of $x_i$ coordinates, limited
by a hyperplane $x_1+x_2+\hdots+x_N=R$,
\begin{equation}
    V(R)=\int_{[\sum_{i=1}^Nx_i]\leq R}dx_1dx_2\hdots
    dx_N=\int_0^RdrS(r)=R^N/N!,
\end{equation}
where $S(R)=R^{N-1}/(N-1)!$ is the corresponding hyper-area at radius $R$ needed
for computation of the multiplicity $\Omega(E)$ and above integral is a
constrained one indicated by a prime.
\begin{solution}
\end{solution}
\end{problem}
%%%%%%%%%%%%%%%%%%%%%%%%%%%%%%%%%%%%%%%%%%%%%%%%%%%%%%%%%%%%%%%%%%%%%%%%%%%%%%%
%%%%%%%%%%%%%%%%%%%%%%%%%%%%%%%%%%%%%%%%%%%%%%%%%%%%%%%%%%%%%%%%%%%%%%%%%%%%%%%
\begin{problem}{4}[Boltzmann gas]
Consider $N$ identical noninteracting particles in a 3D box of linear size $L$,
described by a Hamiltonian
\begin{equation}
    \HH(\qty{\vb{p}_i})=\sum_i^N\frac{p_i^2}{2m}
\end{equation}

(a) By integrating over $6N$ dimensional phase space, compute the multiplicity
(number of microstates) in a shell of width $\Delta$ around energy $E$,
\begin{equation}
    \Omega(E)=\frac{\Delta}{N!}\prod_{i}^N\qty[\int\frac{d\vb{r}_id\vb{p}_i}{(2\pi\hbar)^3}]\delta(E-\HH(\qty{\vb{p}_i})] 
\end{equation}
where $1/N!$ is the Gibbs ``fudge'' factor to crudely (we will see later why
this fix fails at low temperatures; also see below) account for the identity of
these classical particles, and we used the fact that 1 state corresponds to
phase space area $dxdp=2\pi\hbar$ to normalize the integration measure.

\textit{Hint}: Use the expression for a surface area of a $d$-dimensional unit
hypersphere, $S_d=2\pi^{d/2}/\Gamma(d/2)$ (with $S_2=2\pi,S_3=4\pi,\hdots$).

(b) Compute the corresponding (i) entropy $S(E,V,N)=k_B\ln\Omega$, and (ii) find
the temperature $T_c$ at which the entropy becomes negative, i.e., unphysical.

(c) Using the expression for $S(E,V,N)$ found above, calculate the pressure
$P(V,N)=T(\partial S/\partial V)_{E,N}$ for the Boltzmann gas, and show that it
leads to the familiar ideal gas law, $PV=Nk_BT$.

(d) Compute the corresponding chemical potential $\mu=-T(\partial S/\partial
N)_{E,V}$, expressing in terms of the thermal deBroglie wavelength
$\lambda_{dB}(T)=h/\sqrt{2\pi mk_BT}$, $T$ and density $n$.
\begin{solution}
\end{solution}
\end{problem}
%%%%%%%%%%%%%%%%%%%%%%%%%%%%%%%%%%%%%%%%%%%%%%%%%%%%%%%%%%%%%%%%%%%%%%%%%%%%%%%
%%%%%%%%%%%%%%%%%%%%%%%%%%%%%%%%%%%%%%%%%%%%%%%%%%%%%%%%%%%%%%%%%%%%%%%%%%%%%%%
\begin{problem}{5}[Classical harmonic oscillators: Einstein solid]
Consider $N$ decouped 3D classical harmonic oscillators, described by the
familiar Hamiltonian
\begin{equation}
    \HH=\sum_i^N\qty[\frac{p_i^2}{2m}+\frac12m\omega_0^2r_i^2] 
\end{equation}
as the classical version of the problem above, with $\vb{r}_i,\vb{p}_i$ the
classical phase space coordinates. Let's repeat the analysis utilizing and
generalizing the analysis for the Boltzmann gas, above.

(a) By integrating over $6N$ dimensional phase space, compute the multiplicity
(number of microstates) in a shell of width $\Delta$ around energy $E$,
\begin{equation}
    \Omega(E,N)=\Delta\prod_i^N\qty[\int\frac{d\vb{r}_id\vb{p}_i}{(2\pi\hbar)^3}]\delta[E-\HH(\qty{\vb{p}_i})],
\end{equation}
where there is no $1/N!$ Gibbs ``fudge'' factor. Why?

(b) Compare your classical result with the high $T$ classical limit
($\hbar\omega_0/k_BT$ is the relevant dimensionless parameter) of the quantum
treatment of the system in problem 3(c,d,e) above.

(c) What is the relation of this classical analysis to the classical limit
analysis in problem 3e?
\begin{solution}
\end{solution}
\end{problem}
%%%%%%%%%%%%%%%%%%%%%%%%%%%%%%%%%%%%%%%%%%%%%%%%%%%%%%%%%%%%%%%%%%%%%%%%%%%%%%%
    
\end{document}
