\documentclass[12pt]{article}

%%%%%%%%%%%%%%%%%%%%%%%%%%%%%%%%%%%%%%%%%%%%%%%%%%%%%%%%%%%%%%%%%%%%%%%%%%%%%%%%
%                           Package preset for homework
%%%%%%%%%%%%%%%%%%%%%%%%%%%%%%%%%%%%%%%%%%%%%%%%%%%%%%%%%%%%%%%%%%%%%%%%%%%%%%%%
% Miscellaneous
\usepackage[margin=1in]{geometry}
\usepackage[utf8]{inputenc}
\usepackage{indentfirst}
\usepackage{blindtext}
\usepackage{graphicx}
\usepackage{xr-hyper}
\usepackage{hyperref}
\usepackage{color}
\usepackage{float}
% Math
\usepackage{latexsym}
\usepackage{amsfonts}
\usepackage{amssymb}
\usepackage{amsmath}
\usepackage{commath}
\usepackage{amsthm}
\usepackage{bbold}
\usepackage{bm}
% Physics
\usepackage{physics}
\usepackage{siunitx}
% Code typesetting
\usepackage{listings}
% Citation
\usepackage[authoryear]{natbib}
\usepackage{appendix}
\usepackage[capitalize]{cleveref}
% Title & name
\title{Homework}
\author{Tien Vo}
\date{\today}


%%%%%%%%%%%%%%%%%%%%%%%%%%%%%%%%%%%%%%%%%%%%%%%%%%%%%%%%%%%%%%%%%%%%%%%%%%%%%%%%
%                   User-defined commands and environments
%%%%%%%%%%%%%%%%%%%%%%%%%%%%%%%%%%%%%%%%%%%%%%%%%%%%%%%%%%%%%%%%%%%%%%%%%%%%%%%%
%%% Misc
\sisetup{load-configurations=abbreviations}
\newcommand{\due}[1]{\date{Due: #1}}
\newcommand{\hint}{\textit{Hint}}
\let\oldt\t
\renewcommand{\t}[1]{\text{#1}}

%%% Bold sets & abbrv
\newcommand{\N}{\mathbb{N}}
\newcommand{\Z}{\mathbb{Z}}
\newcommand{\R}{\mathbb{R}}
\newcommand{\Q}{\mathbb{Q}}
\let\oldP\P
\renewcommand{\P}{\mathbb{P}}
\newcommand{\LL}{\mathcal{L}}
\newcommand{\FF}{\mathcal{F}}
\newcommand{\HH}{\mathcal{H}}
\newcommand{\NN}{\mathcal{N}}
\newcommand{\ZZ}{\mathcal{Z}}
\newcommand{\RN}[1]{\textup{\uppercase\expandafter{\romannumeral#1}}}
\newcommand{\ua}{\uparrow}
\newcommand{\da}{\downarrow}

%%% Unit vectors
\newcommand{\xhat}{\vb{\hat{x}}}
\newcommand{\yhat}{\vb{\hat{y}}}
\newcommand{\zhat}{\vb{\hat{z}}}
\newcommand{\nhat}{\vb{\hat{n}}}
\newcommand{\rhat}{\vb{\hat{r}}}
\newcommand{\phihat}{\bm{\hat{\phi}}}
\newcommand{\thetahat}{\bm{\hat{\theta}}}

%%% Other math stuff
\providecommand{\units}[1]{\,\ensuremath{\mathrm{#1}}\xspace}
% Set new style for problem
\newtheoremstyle{problemstyle}  % <name>
        {10pt}                   % <space above>
        {10pt}                   % <space below>
        {\normalfont}           % <body font>
        {}                      % <indent amount}
        {\bfseries\itshape}     % <theorem head font>
        {\normalfont\bfseries:} % <punctuation after theorem head>
        {.5em}                  % <space after theorem head>
        {}                      % <theorem head spec (can be left empty, 
                                % meaning `normal')>

% Set problem environment
\theoremstyle{problemstyle}
\newtheorem{problemenv}{Problem}[section]
\newenvironment{problem}[1]{%
  \renewcommand\theproblemenv{#1}%
  \problemenv
}{\endproblemenv}
% Set lemma environment
\newenvironment{lemma}[2][Lemma]{\begin{trivlist}
\item[\hskip \labelsep {\bfseries #1}\hskip \labelsep {\bfseries #2.}]}{\end{trivlist}}
% Set solution environment
\newenvironment{solution}{
    \begin{proof}[Solution]$ $\par\nobreak\ignorespaces
}{\end{proof}}
\numberwithin{equation}{problemenv}

%%% Page format
\setlength{\parindent}{0.5cm}
\setlength{\oddsidemargin}{0in}
\setlength{\textwidth}{6.5in}
\setlength{\textheight}{8.8in}
\setlength{\topmargin}{0in}
\setlength{\headheight}{18pt}

%%% Code environments
\definecolor{dkgreen}{rgb}{0,0.6,0}
\definecolor{gray}{rgb}{0.5,0.5,0.5}
\definecolor{mauve}{rgb}{0.58,0,0.82}
\lstset{frame=tb,
  language=Python,
  aboveskip=3mm,
  belowskip=3mm,
  showstringspaces=false,
  columns=flexible,
  basicstyle={\small\ttfamily},
  numbers=none,
  numberstyle=\tiny\color{gray},
  keywordstyle=\color{blue},
  commentstyle=\color{dkgreen},
  stringstyle=\color{mauve},
  breaklines=true,
  breakatwhitespace=true,
  tabsize=4
}
\lstset{
  language=Mathematica,
  numbers=left,
  numberstyle=\tiny\color{gray},
  numbersep=5pt,
  breaklines=true,
  captionpos={t},
  frame={lines},
  rulecolor=\color{black},
  framerule=0.5pt,
  columns=flexible,
  tabsize=2
}


\title{Homework 1: Phys 7230 (Spring 2022)}
\due{January 24, 2022}

\begin{document}
\maketitle
%%%%%%%%%%%%%%%%%%%%%%%%%%%%%%%%%%%%%%%%%%%%%%%%%%%%%%%%%%%%%%%%%%%%%%%%%%%%%%%
\begin{problem}{1}
Consider two (otherwise) closed systems A and B at respective temperatures $T_A$
and $T_B$ in thermal contact with each other, so that heat $Q$ can flow between
them.

Using the 2nd law of thermodynamics -- total entropy $S$ of a closed system
increases under equilibration -- show that heat $Q$ flows from the hotter to the
colder system as $A$ and $B$ come to thermal equilibrium.

\textit{Hint}: $TdS\geq Q$.

\begin{solution}
First, for a closed system, the total energy $U=U_A+U_B$ is conserved. So
$dU_B=-dU_A$. The total entropy is $S=S_A+S_B$. Since the volume and number of
particles of each system is constant, we have
\begin{align}
    dS
    =\qty(\frac{\partial S}{\partial U_A})_{N_A,V_A}dU_A
    +\qty(\frac{\partial S}{\partial U_B})dU_B
    =\qty(\frac1{T_A}-\frac1{T_B})dU_A\geq 0
\end{align}
The last inequality is the 2nd law of thermodynamics. Now, without loss of
generality, assume $T_A>T_B$, then it must be the case that $dU_A<0$. Since
the change in energy is the heat exchanged ($dU_A=Q$), it follows that the
hotter system (A) loses energy in the form of heat to the lower system (B).
\end{solution}
\end{problem}
%%%%%%%%%%%%%%%%%%%%%%%%%%%%%%%%%%%%%%%%%%%%%%%%%%%%%%%%%%%%%%%%%%%%%%%%%%%%%%%
%%%%%%%%%%%%%%%%%%%%%%%%%%%%%%%%%%%%%%%%%%%%%%%%%%%%%%%%%%%%%%%%%%%%%%%%%%%%%%%
\begin{problem}{2}[Spin-1/2 paramagnet]
Consider a magnet of $N$ noninteracting spin-1/2 magnetic moments in an external
magnetic field $\vb{B}$, with Hamiltonian given by Zeeman energy,
\begin{equation}
    \HH=-\sum_{i=1}^N\bm\mu_i\vdot\vb{B}=-\sum_{i=1}^N\mu_BB\sigma_i
    \equiv-\sum_{i=1}^Nh\sigma_i
\end{equation}
where $\mu_B$ is Bohr magneton (carrying units of magnetic moment) and
$\sigma_i=\pm1$ labels the two Zeeman spin states of $n$th spin.

(a) For fixed dimensionless magnetization $M=N_\ua-N_\da$, (i) what is the
multiplicity $\Omega(M,N)$? (ii) What is the corresponding probability $P(M,N)$?
Check that $\sum_{M=-N}^NP(M,N)=1$.

(b) Compute the multiplicity $\Omega(E)$ for this system at a total energy $E$
and sketch/plot it as a function of full range of accessible energies. 

\textit{Hint}: Note that the magnetization $M$ is proportional to the energy
$E$.

(c) Derive the relation between temperature $T(E)$ and energy $E$ and plot
$T(E)$.

(d) Calculate the (i) magnetization $m(T,B)=\mu_B\sum_{i=1}^N\sigma_i$, (ii)
obtain its asymptotic forms in the classical $\mu_BB/k_BT\ll 1$ and quantum
$\mu_BB/k_BT\gg 1$ limits and (iii) plot it as a function of $T$ at a couple of
fixed values of $B$ and as a function of $B$ at a couple of fixed values of $T$.

\textit{Hint}: (i) Notice that magnetization is proportional to the energy $E$,
(ii) Eliminate $E$ in favor of $T$, (iii) Use lowest order Stirling
approximation throughout to simplify the factorials in your expression.

(e) Compute the linear magnetic susceptibility $\chi(T,B)=\eval{\partial
m/\partial B}_{B\to0}$, show that it exhibits Curie form
$\chi_\text{Curie}=a/T$, extracting the coefficient $a$.

(f) Compute the heat capacity (specific heat), $C_v(T)=T(\partial S/\partial
T)_{V,N}$, extract its low and high temperatures asymptotics, and sketch it,
noting its limiting forms and the crossover temperature.
\begin{solution}
(a) Given $M=N_\ua-N_\da$ and $N=N_\ua+N_\da$, we can write
\begin{equation}
    N_\ua=\frac12\qty(N+M)\qquad\text{and}\qquad
    N_\da=\frac12\qty(N-M).
\end{equation}
(i) The multiplicity of $\Omega(M,N)$ is just the combination of $N_\ua$ in $N$
total magnets
\begin{equation}\label{p1a:Omega}
    \Omega(M,N)=\binom{N}{N_\ua}=\frac{N!}{N_\ua!N_\da!}
    =\frac{N!}{\qty(\frac{N+M}{2})!\qty(\frac{N-M}{2})!}
\end{equation}
(ii) For $N$ total magnets, each with 2 possible states, the sample size is
$2^N$. So the probability $P(M,N)$ is
\begin{equation}
    P(M,N)
    =\frac1{2^N}\Omega(M,N)
    =\frac1{2^N}\frac{N!}{\qty(\frac{N+M}{2})!\qty(\frac{N-M}{2})!} 
\end{equation}
Using the binomial theorem,
\begin{equation}
    2^N=\sum_{N_\ua=0}^N\binom{N}{N_\ua}=\sum_{N_\ua=0}^N\Omega(M,N),
\end{equation}
we can convert the summation back to $M$. For $0\leq N_\ua\leq N$, $-N\leq M\leq
N$. Thus,
\begin{equation}
    1=\frac1{2^N}\sum_{M=-N}^N\binom{N}{N_\ua}=\sum_{M=-N}^NP(M,N)
\end{equation}

(b) By definition, $E=-hM$. So from \eqref{p1a:Omega},
\begin{equation}
    \Omega(E)=\frac{N!}{\qty(\frac{N-E/h}{2})!\qty(\frac{N+E/h}{2})!}
\end{equation}
for $-hN\leq E\leq hN$. A plot of $\Omega(E)$ is included below
\begin{center}
    \includegraphics[width=0.8\textwidth]{p1b.png} 
\end{center}

(c) By definition, the entropy is
\begin{align}
    S/k_B&=\ln\Omega\notag\\
    &\approx N\ln N-N-\qty(\frac{N-E/h}{2})\ln\qty(\frac{N-E/h}{2})
    +\frac{N-E/h}{2}\notag\\
    &\qquad-\qty(\frac{N+E/h}{2})\ln\qty(\frac{N+E/h}{2})
    +\frac{N+E/h}{2}\notag\\
    &=N\ln N-\qty(\frac{N-E/h}{2})\ln\qty(\frac{N-E/h}{2})
    -\qty(\frac{N+E/h}{2})\ln\qty(\frac{N+E/h}{2})
\end{align}
by Stirling approximation. Then the temperature is
\begin{align}\label{p1c:T}
    \frac1{k_BT}=\frac{\partial(S/k_B)}{\partial E}
    =\frac1{2h}\ln\qty(\frac{N-E/h}{N+E/h})
    \Rightarrow\frac{k_BT}{h}=\frac{2}{\ln\qty(\frac{N-E/h}{N+E/h})}
\end{align}
A plot of the normalized temperature $k_BT/h$ is included below
\begin{center}
    \includegraphics[width=0.8\textwidth]{p1c.png} 
\end{center}

(d) First, we can invert \eqref{p1c:T} to solve for $E$ as below
\begin{equation}
    E=-hN\frac{e^{2h/k_BT}-1}{e^{2h/k_BT}+1}=-hN\tanh(\frac{h}{k_BT})
    =-hN\tanh(\frac{\mu_BB}{k_BT})
\end{equation}
But the energy $E$ can be written in terms of $m$ as $E=-mB$ since $m=\mu_BM$.
So (i) the magnetization is
\begin{equation}\label{p1d:m}
    m=\mu_BN\tanh\qty(\frac{\mu_BB}{k_BT}) 
\end{equation}
(ii) For $\mu_BB/k_BT\ll 1$, $\tanh$ is a linear function
\begin{equation}
    m\qty(\frac{\mu_BB}{k_BT})=\frac{\mu_B^2nB}{k_BT} 
\end{equation}
and for $\mu_BB/k_BT\gg 1$, $\tanh(x)\to 1$, so the magnetization is a constant
\begin{equation}
    m\qty(\frac{\mu_BB}{k_BT})=\mu_BN 
\end{equation}
(iii) A plot of $m(T,B)/\mu_BN$ is included below
\begin{center}
    \includegraphics[width=\textwidth]{p1d.png} 
\end{center}

(e) From \eqref{p1d:m}, we can calculate
\begin{equation}
    \chi=\lim_{B\to0}\frac{\partial m}{\partial B}
    =\frac{\mu^2N}{k_BT}\lim_{B\to0}\sech^2\qty(\frac{\mu_BB}{k_BT})
    =\frac{\mu^2N}{k_BT}
\end{equation}
So $\chi$ follows Curie's law with $a=\mu^2N/k_B$.

(f) By the chain rule,
\begin{equation}
    C_V=T\frac{\partial S}{\partial E}\frac{\partial E}{\partial T} 
    =\frac{\partial E}{\partial T}
    =Nk_B\qty(\frac{\mu_BB}{k_BT})^2\sech^2\qty(\frac{\mu_BB}{k_BT})
\end{equation}
For $x=\mu_BB/k_BT\ll1$ (high temperature), $\sech(x)\to1$ and the heat capacity
is
\begin{equation}
    C_V\qty(\frac{\mu_BB}{k_BT}\ll 1)\approx Nk_B\qty(\frac{\mu_BB}{k_BT})^2
\end{equation}
For $x\gg 1$ (low temperature), $\sech^2(x)$ dominates and so $C_V\to0$.
Letting the crossover temperature to be $T^\ast=\mu_BB/k_B$, then we can rewrite
the heat capacity in normalized form
\begin{equation}\label{p1f:C}
    \frac{C_V}{Nk_B}=\qty(\frac{T^\ast}{T})^2\sech^2\qty(\frac{T^\ast}{T}) 
\end{equation}
A plot of \eqref{p1f:C} is included below
\begin{center}
    \includegraphics[width=0.8\textwidth]{p1f.png} 
\end{center}
\end{solution}
\end{problem}
%%%%%%%%%%%%%%%%%%%%%%%%%%%%%%%%%%%%%%%%%%%%%%%%%%%%%%%%%%%%%%%%%%%%%%%%%%%%%%%
%%%%%%%%%%%%%%%%%%%%%%%%%%%%%%%%%%%%%%%%%%%%%%%%%%%%%%%%%%%%%%%%%%%%%%%%%%%%%%%
\begin{problem}{3}[Quamtum harmonic oscillators: Einstein solid]
Consider $N$ decoupled 3D quantum harmonic oscillators as a model of atomic
vibrations in a crystalline solid (Einstein phonons), described by the familiar
quantum Hamiltonian
\begin{equation}
    \hat\HH=\sum_{i}^N\qty[\frac{\hat{p}_i^2}{2m}+\frac12m\omega_0^2\hat{r}_i^2-\frac32\hbar\omega_0] 
\end{equation}
where for convenience I defined $\hat\HH$ with zero point energy subtracted off.

(a) Let's warm up on a single harmonic oscillator, computing its degeneracy
$g(n)=\Omega(E=\hbar\omega_0n)$ a fixed total energy
$E\equiv\hbar\omega_0n=\hbar\omega_0\sum_{\alpha=1}^dn_\alpha$ (where
$n_\alpha\in\Z$ are integer quantum numbers for $\alpha=x,y,\hdots$) for the
cases of (i) 2D and (ii) 3D.

(b) Recalling the eigenvalues
$E[\qty{n_\alpha}]=\hbar\omega_0\sum_{\alpha=1}^{3N}n_\alpha$
($\alpha=x_1,y_1,z_1,x_2,y_2,z_2,\hdots$ ranging from 1 to $3N$) for the
harmonic oscillator Hamiltonian, compute the multiplicity
\begin{equation}
    \Omega(E)=\sum_{\qty{n_\alpha}}\delta_{E,E[\qty{n_\alpha}]} 
\end{equation}
taking $E=\hbar\omega_0n$ ($n\in\Z$).

\textit{Hint}: Think about how to distribute $n$ total quanta of excitations
among $3N$ 1D oscillators, and use the lowest Stirling formula approximation for
$N\gg 1$ and $n\gg 1$.

(c) Compute the entropy $S(E)$ and the corresponding $T(E)$, thereby extracting
energy $E(T)$ as a function of temperature $T$, exploring its classical
$\hbar\omega_0/k_BT\ll1$ and quantum $\hbar\omega_0/k_BT\gg1$ limiting
functional forms. Plot $E(T)$, noting limiting forms.

(d) Compute heat capacity
$C_v=T(\partial S/\partial T)_{V,N}=\partial E/\partial T$ and explore its 
classical (high $T$) and quantum (low $T$) limits, showing
the expected equipartition $C_v=N_\text{dof}k_B$ in the former and its 
breakdown in the latter limits. Plot $C_v(T)$, noting the crossover temperature.

(e) Consider a classical limit of the problem with small $\hbar\omega_0/k_BT$
such that $E[\qty{n_\alpha}]=\sum_{\alpha=1}^{3N}\epsilon_\alpha$ and oscillator
eigenvalues $\epsilon_\alpha$ vary nearly continuously. Using this
simplification compute
\begin{equation}
    \Omega(E)=\Delta\prod_{\alpha=1}^{3N}\int\frac{d\epsilon_\alpha}{\hbar\omega_0}
    \delta(E-E[\qty{\epsilon_\alpha}]) 
\end{equation}
as a multidimensional integral over $\epsilon_\alpha$.

\textit{Hint}: It is helpful to use a result for a hypervolume of an
$N$-dimensional space spanned by positive values of $x_i$ coordinates, limited
by a hyperplane $x_1+x_2+\hdots+x_N=R$,
\begin{equation}
    V(R)=\int_{[\sum_{i=1}^Nx_i]\leq R}dx_1dx_2\hdots
    dx_N=\int_0^RdrS(r)=R^N/N!,
\end{equation}
where $S(R)=R^{N-1}/(N-1)!$ is the corresponding hyper-area at radius $R$ needed
for computation of the multiplicity $\Omega(E)$ and above integral is a
constrained one indicated by a prime.
\begin{solution}
(a) For the 2D case, there are $n$ quanta to distribute into 2 partitions
($n_x,n_y$). The problem can be understood in terms of a sequence of 0's and 1's
where 0's are the quanta and 1 is the separation between the two groups $n_x$
and $n_y$. For example, if $n=5$, $000100$ is a possible sequence. So there are 
$n+1$ objects in total and the degeneracy is
\begin{equation}
    g_{2D}(n)=\binom{n+1}{n}=n+1 
\end{equation}
For the 3D case, the number of 1's is 2, to separate the 0's (quanta) into 
three groups $n_x,n_y,n_z$. So the degeneracy is
\begin{equation}
    g_{3D}(n)=\binom{n+2}{n}=\frac{(n+1)(n+2)}{2} 
\end{equation}

(b) For the $3N$-dimensional case, there needs to be $3N-1$ 1's to seperate the
0's into $3N$ groups. So the multiplicity is
\begin{equation}
    \Omega=\binom{n+3N-1}{n} 
    =\frac{(n+3N-1)!}{n!(3N-1)!}
    \approx\frac{(n+3N)!}{n!(3N)!}
\end{equation}
where we have assumed $n,N\gg 1$. Calculating $\ln\Omega$ and using the lowest
order Stirling approximation, we get
\begin{equation}
    \ln\Omega\approx(n+3N)\ln\qty(n+3N)-n\ln n-3N\ln 3N 
\end{equation}
So taking the exponent again gives us the multiplicity
\begin{equation}
    \Omega(E)\approx\qty(1+\frac{3N\hbar\omega_0}{E})^{E/\hbar\omega_0}
    \qty(1+\frac{E}{3N\hbar\omega_0})^{3N}
\end{equation}
where we have also written $n=E/\hbar\omega_0$.

(c) By definition, the entropy is
\begin{align}
    S/k_B
    &=\ln\Omega=\frac{E}{\hbar\omega_0}\ln\qty(1+\frac{3N\hbar\omega_0}{E})
    +3N\ln\qty(1+\frac{E}{3N\hbar\omega_0})\notag\\
    &=3N\qty[\epsilon\ln\qty(1+\frac{1}{\epsilon})+\ln\qty(1+\epsilon)]
\end{align}
where $\epsilon\equiv E/3N\hbar\omega_0$. The temperature is then
\begin{equation}
    \frac{3N\hbar\omega_0}{k_BT}=\frac{\partial(S/k_B)}{\partial\epsilon}
    =3N\ln\qty(1+\frac1\epsilon)
    \Rightarrow T(E)=\frac{\hbar\omega_0}{k_B\ln\qty(1+3N\hbar\omega_0/E)}
\end{equation}
Inverting this result, we get
\begin{equation}
    E(T)=\frac{3N\hbar\omega_0}{e^{\hbar\omega_0/k_BT}-1}
\end{equation}
For the classical limit ($x=\hbar\omega_0/k_BT\ll 1$), $e^x-1\approx x$
and $E(T)=3Nk_BT$. For the quantum limit ($x\gg 1$), $e^x\to\infty$ and
$E\sim e^{-x}\to0$. Letting $T^\ast=\hbar\omega_0/k_B$, a plot of $E(T)$ is 
included below.
\begin{center}
    \includegraphics[width=0.8\textwidth]{p3c.png} 
\end{center}
At high temperature (classical limit), $E$ grows linearly in temperature and at 
low temperature (quantum limit), $E\to0$, as expected.

(d) By definition, the heat capacity is
\begin{equation}
    C_v=\frac{\partial E}{\partial
    T}=3Nk_B\qty(\frac{T^\ast}{T})^2\frac{e^{T^\ast/T}}{(e^{T^\ast/T}-1)^2} 
\end{equation}
For the quantum limit, $C_v$ grows as $\sim (T^\ast/T)^2 e^{-T^\ast/T}\sim0$
because of the exponential term. At large $T$ (classical limit), we can set
$x=T^\ast/T$ and write $C_v$ as
\begin{equation}\label{p3d:C}
    C_v=3Nk_B x^2\frac{e^x}{(e^x-1)^2}\approx
    3Nk_Bx^2\frac{1+x}{x^2}\approx 3Nk_B
\end{equation}
when $x\to0$. This is the expected equipartition where $N_\text{dof}=3N$. A plot
of $C_v$ is included below.
\begin{center}
    \includegraphics[width=0.8\textwidth]{p3d.png} 
\end{center}
The formal break occurs at $T=T^\ast$ as expected, where
$T^\ast=\hbar\omega_0/k_B$ is the
crossover temperature. Also, $C_v$ is asymptotically constant as $T\to\infty$ as
predicted in \eqref{p3d:C} and 0 as $T\to0$.

(e) First, we can isolate the integration
\begin{equation}
    \Omega(E)=\frac{\Delta}{(\hbar\omega_0)^{3N}}\prod_{\alpha=1}^{3N}
    \int d\epsilon_\alpha\delta(E-E[\qty{\epsilon_\alpha}])
    =\frac{\Delta}{(\hbar\omega_0)^{3N}}I_{3N}
\end{equation}
$I_{3N}$ is just the $3N-1$-dimensional hyper-area restricted by the 
constraint $E=\sum_{\alpha=1}^{3N}\epsilon_\alpha$. We can prove this by 
induction. First consider the 2D case where $E=\epsilon_1+\epsilon_2$ and
\begin{equation}
    I_{2}=\int_0^\infty d\epsilon_1\int_0^\infty
    d\epsilon_2\delta(E-\epsilon_1-\epsilon_2) 
\end{equation}
Let $u=E-\epsilon_1$, then $du=-d\epsilon_1$ and
\begin{equation}
    I_{2}=-\int_E^{-\infty}du\int_0^\infty d\epsilon_2\delta(u-\epsilon_2) 
    =\int_0^Edu\int_0^\infty d\epsilon_2\delta(u-\epsilon_2)=\int_0^Edu
    =E=S_2(E)
\end{equation}
where $S_n$ is the given hyper-area in the problem. Now, to prove the
$3N$-dimensional case, suppose the $3N-1$-dimensional case is true. Then,
\begin{align}
    I_{3N}
    &=\int_0^\infty d\epsilon_{3N}\prod_{\alpha=1}^{3N-1}\int_0^\infty
    d\epsilon_\alpha
    \delta\qty(E-\epsilon_{3N}-\sum_{\beta=1}^{3N-1}\epsilon_\beta)\notag\\
    &=\int_0^Edu\prod_{\alpha=1}^{3N-1}\int_0^\infty d\epsilon_\alpha\delta\qty(u-\sum_{\beta=1}^{3N-1}\epsilon_\beta)\tag{$u=E-\epsilon_{3N}$}\\
    &=\int_0^Edu S_{3N-1}(u)\notag\\
    &=\frac1{(3N-2)!}\int_0^E u^{3N-2} du\notag\\
    &=\frac{E^{3N-1}}{(3N-1)!}\notag\\
    &=S_{3N}(E)
\end{align}
Finally, we can use this result to write
\begin{equation}
    \Omega(E)=\frac{\Delta}{(\hbar\omega_0)^{3N}}S_{3N}(E) 
    =\frac1{(3N-1)!}\frac{\Delta}{E}\qty(\frac{E}{\hbar\omega_0})^{3N}
\end{equation}
But $(3N-1)!\approx(3N)!\approx (3N)^{3N}$ for $N\gg1$. So
\begin{equation}
    \Omega(E)\approx\frac{\Delta}{E}\qty(\frac{E}{3N\hbar\omega_0})^{3N} 
\end{equation}
\end{solution}
\end{problem}
%%%%%%%%%%%%%%%%%%%%%%%%%%%%%%%%%%%%%%%%%%%%%%%%%%%%%%%%%%%%%%%%%%%%%%%%%%%%%%%
%%%%%%%%%%%%%%%%%%%%%%%%%%%%%%%%%%%%%%%%%%%%%%%%%%%%%%%%%%%%%%%%%%%%%%%%%%%%%%%
\begin{problem}{4}[Boltzmann gas]
Consider $N$ identical noninteracting particles in a 3D box of linear size $L$,
described by a Hamiltonian
\begin{equation}
    \HH(\qty{\vb{p}_i})=\sum_i^N\frac{p_i^2}{2m}
\end{equation}

(a) By integrating over $6N$ dimensional phase space, compute the multiplicity
(number of microstates) in a shell of width $\Delta$ around energy $E$,
\begin{equation}
    \Omega(E)=\frac{\Delta}{N!}\prod_{i}^N\qty[\int\frac{d\vb{r}_id\vb{p}_i}{(2\pi\hbar)^3}]\delta(E-\HH(\qty{\vb{p}_i})] 
\end{equation}
where $1/N!$ is the Gibbs ``fudge'' factor to crudely (we will see later why
this fix fails at low temperatures; also see below) account for the identity of
these classical particles, and we used the fact that 1 state corresponds to
phase space area $dxdp=2\pi\hbar$ to normalize the integration measure.

\textit{Hint}: Use the expression for a surface area of a $d$-dimensional unit
hypersphere, $S_d=2\pi^{d/2}/\Gamma(d/2)$ (with $S_2=2\pi,S_3=4\pi,\hdots$).

(b) Compute the corresponding (i) entropy $S(E,V,N)=k_B\ln\Omega$, and (ii) find
the temperature $T_c$ at which the entropy becomes negative, i.e., unphysical.

(c) Using the expression for $S(E,V,N)$ found above, calculate the pressure
$P(V,N)=T(\partial S/\partial V)_{E,N}$ for the Boltzmann gas, and show that it
leads to the familiar ideal gas law, $PV=Nk_BT$.

(d) Compute the corresponding chemical potential $\mu=-T(\partial S/\partial
N)_{E,V}$, expressing in terms of the thermal deBroglie wavelength
$\lambda_{dB}(T)=h/\sqrt{2\pi mk_BT}$, $T$ and density $n$.
\begin{solution}
(a) First, we can simplify
\begin{align}
    \Omega(E)
    &=\frac{\Delta}{E}\frac{V^N}{N!}\frac1{h^{3N}}\prod_i^N\int
    d\vb{p}_i\delta\qty(1-\sum_j^N\frac{p_j^2}{2mE})\notag\\
    &=\frac{\Delta}{E}\frac{V^N}{N!}\qty(\frac{2mE}{h^2})^{3N/2}\prod_i^N\int
    d\overline{\vb{p}}_i\delta\qty(1-\sum_j^N\overline{p}_j^2)\tag{$\overline{p}_j^2=p_j^2/2mE$}
\end{align}
The integration is now just the surface area of a $3N$-dimensional unit
hypersphere, since it is constrained to the surface
$1=\sum_j^N\overline{p}_j^2$. So we can write
\begin{align}
    \Omega(E)
    &=\frac{\Delta}{E}\frac{V^N}{N!}\qty(\frac{2mE}{h^2})^{3N/2}\frac{2\pi^{3N/2}}{\Gamma(3N/2)}\notag\\
    &\approx\frac{\Delta}{E}\frac{V^N}{\sqrt{2\pi N}e^{-N}N^N}\qty(\frac{2\pi
    mE}{h^2})^{3N/2}\frac{2}{\sqrt{3\pi N}e^{-3N/2}(3N/2)^{3N/2}}\notag\\
    &=\frac{2\Delta}{\sqrt6\pi NE}e^{5N/2}\frac{V^N}{N^N}\qty(\frac{4\pi
    mE}{3Nh^2})^{3N/2}
\end{align}
where we have written $\Gamma(3N/2)=(3N/2-1)!\approx(3N/2)!$ and used Stirling
approximation on the factorials.

(b) By definition, the (i) entropy is
\begin{equation}\label{p4b:S}
    S/k_B=\ln\Omega
    =\ln\qty(\frac{2\Delta}{\sqrt6\pi NE})+N\qty{\frac52
    +\ln\qty[\frac{V}{N}\qty(\frac{4\pi mE}{3Nh^2})^{3/2}]}
\end{equation}
the first term is small compared to $N$ and we retrieve the Sackur-Tetrode
equation in the last two terms. Also, the temperature is
\begin{equation}
    \frac1{k_BT}=\frac{\partial(S/k_B)}{\partial E}=\frac{3}2\frac{N}{E} 
    \Rightarrow\frac{E}{N}=\frac32k_BT
\end{equation}
Now, from \eqref{p4b:S}, $S<0$ only when
\begin{equation}
    \frac{V}{N}\qty(\frac{4\pi mE}{3Nh^2})^{3/2}
    =\frac{V}{N}\qty(\frac{2\pi mk_BT}{h^2})^{3/2}<e^{-5/2}
\end{equation}
(ii) Solving this inequality, we get
\begin{equation}
    T_c<\frac{h^2}{2\pi mk_B}\qty(\frac{N}{V})^{2/3}e^{-5/3}
\end{equation}

(c) Differentiating $S$, we get
\begin{equation}
    \frac{P}{T}=\frac{\partial S}{\partial V}=\frac{Nk_B}{V} 
    \Rightarrow PV=Nk_BT
\end{equation}
This is the ideal gas law.

(d) Differentiating $S$, we get
\begin{equation}
    \mu=-k_BT\ln\qty[\frac8{3\sqrt3}\frac{V}{N}\qty(\frac{\pi mE}{h^2N})^{3/2}]
    =-k_BT\ln\qty[\frac{1}{n}\qty(\frac{2\pi mk_BT}{h^2})^{3/2}]
    =k_BT\ln\qty(n\lambda_{dB}^3)
\end{equation}
\end{solution}
\end{problem}
%%%%%%%%%%%%%%%%%%%%%%%%%%%%%%%%%%%%%%%%%%%%%%%%%%%%%%%%%%%%%%%%%%%%%%%%%%%%%%%
%%%%%%%%%%%%%%%%%%%%%%%%%%%%%%%%%%%%%%%%%%%%%%%%%%%%%%%%%%%%%%%%%%%%%%%%%%%%%%%
\begin{problem}{5}[Classical harmonic oscillators: Einstein solid]
Consider $N$ decouped 3D classical harmonic oscillators, described by the
familiar Hamiltonian
\begin{equation}
    \HH=\sum_i^N\qty[\frac{p_i^2}{2m}+\frac12m\omega_0^2r_i^2] 
\end{equation}
as the classical version of the problem above, with $\vb{r}_i,\vb{p}_i$ the
classical phase space coordinates. Let's repeat the analysis utilizing and
generalizing the analysis for the Boltzmann gas, above.

(a) By integrating over $6N$ dimensional phase space, compute the multiplicity
(number of microstates) in a shell of width $\Delta$ around energy $E$,
\begin{equation}
    \Omega(E,N)=\Delta\prod_i^N\qty[\int\frac{d\vb{r}_id\vb{p}_i}{(2\pi\hbar)^3}]\delta[E-\HH(\qty{\vb{p}_i})],
\end{equation}
where there is no $1/N!$ Gibbs ``fudge'' factor. Why?

(b) Compare your classical result with the high $T$ classical limit
($\hbar\omega_0/k_BT$ is the relevant dimensionless parameter) of the quantum
treatment of the system in problem 3(c,d,e) above.

(c) What is the relation of this classical analysis to the classical limit
analysis in problem 3e?
\begin{solution}
(a)
\end{solution}
\end{problem}
%%%%%%%%%%%%%%%%%%%%%%%%%%%%%%%%%%%%%%%%%%%%%%%%%%%%%%%%%%%%%%%%%%%%%%%%%%%%%%%
    
\end{document}
