\documentclass[12pt]{article}

%%%%%%%%%%%%%%%%%%%%%%%%%%%%%%%%%%%%%%%%%%%%%%%%%%%%%%%%%%%%%%%%%%%%%%%%%%%%%%%%
%                           Package preset for homework
%%%%%%%%%%%%%%%%%%%%%%%%%%%%%%%%%%%%%%%%%%%%%%%%%%%%%%%%%%%%%%%%%%%%%%%%%%%%%%%%
% Miscellaneous
\usepackage[margin=1in]{geometry}
\usepackage[utf8]{inputenc}
\usepackage{indentfirst}
\usepackage{blindtext}
\usepackage{graphicx}
\usepackage{xr-hyper}
\usepackage{hyperref}
\usepackage{color}
\usepackage{float}
% Math
\usepackage{latexsym}
\usepackage{amsfonts}
\usepackage{amssymb}
\usepackage{amsmath}
\usepackage{commath}
\usepackage{amsthm}
\usepackage{bbold}
\usepackage{bm}
% Physics
\usepackage{physics}
\usepackage{siunitx}
% Code typesetting
\usepackage{listings}
% Citation
\usepackage[authoryear]{natbib}
\usepackage{appendix}
\usepackage[capitalize]{cleveref}
% Title & name
\title{Homework}
\author{Tien Vo}
\date{\today}


%%%%%%%%%%%%%%%%%%%%%%%%%%%%%%%%%%%%%%%%%%%%%%%%%%%%%%%%%%%%%%%%%%%%%%%%%%%%%%%%
%                   User-defined commands and environments
%%%%%%%%%%%%%%%%%%%%%%%%%%%%%%%%%%%%%%%%%%%%%%%%%%%%%%%%%%%%%%%%%%%%%%%%%%%%%%%%
%%% Misc
\sisetup{load-configurations=abbreviations}
\newcommand{\due}[1]{\date{Due: #1}}
\newcommand{\hint}{\textit{Hint}}
\let\oldt\t
\renewcommand{\t}[1]{\text{#1}}

%%% Bold sets & abbrv
\newcommand{\N}{\mathbb{N}}
\newcommand{\Z}{\mathbb{Z}}
\newcommand{\R}{\mathbb{R}}
\newcommand{\Q}{\mathbb{Q}}
\let\oldP\P
\renewcommand{\P}{\mathbb{P}}
\newcommand{\LL}{\mathcal{L}}
\newcommand{\FF}{\mathcal{F}}
\newcommand{\HH}{\mathcal{H}}
\newcommand{\NN}{\mathcal{N}}
\newcommand{\ZZ}{\mathcal{Z}}
\newcommand{\RN}[1]{\textup{\uppercase\expandafter{\romannumeral#1}}}
\newcommand{\ua}{\uparrow}
\newcommand{\da}{\downarrow}

%%% Unit vectors
\newcommand{\xhat}{\vb{\hat{x}}}
\newcommand{\yhat}{\vb{\hat{y}}}
\newcommand{\zhat}{\vb{\hat{z}}}
\newcommand{\nhat}{\vb{\hat{n}}}
\newcommand{\rhat}{\vb{\hat{r}}}
\newcommand{\phihat}{\bm{\hat{\phi}}}
\newcommand{\thetahat}{\bm{\hat{\theta}}}

%%% Other math stuff
\providecommand{\units}[1]{\,\ensuremath{\mathrm{#1}}\xspace}
% Set new style for problem
\newtheoremstyle{problemstyle}  % <name>
        {10pt}                   % <space above>
        {10pt}                   % <space below>
        {\normalfont}           % <body font>
        {}                      % <indent amount}
        {\bfseries\itshape}     % <theorem head font>
        {\normalfont\bfseries:} % <punctuation after theorem head>
        {.5em}                  % <space after theorem head>
        {}                      % <theorem head spec (can be left empty, 
                                % meaning `normal')>

% Set problem environment
\theoremstyle{problemstyle}
\newtheorem{problemenv}{Problem}[section]
\newenvironment{problem}[1]{%
  \renewcommand\theproblemenv{#1}%
  \problemenv
}{\endproblemenv}
% Set lemma environment
\newenvironment{lemma}[2][Lemma]{\begin{trivlist}
\item[\hskip \labelsep {\bfseries #1}\hskip \labelsep {\bfseries #2.}]}{\end{trivlist}}
% Set solution environment
\newenvironment{solution}{
    \begin{proof}[Solution]$ $\par\nobreak\ignorespaces
}{\end{proof}}
\numberwithin{equation}{problemenv}

%%% Page format
\setlength{\parindent}{0.5cm}
\setlength{\oddsidemargin}{0in}
\setlength{\textwidth}{6.5in}
\setlength{\textheight}{8.8in}
\setlength{\topmargin}{0in}
\setlength{\headheight}{18pt}

%%% Code environments
\definecolor{dkgreen}{rgb}{0,0.6,0}
\definecolor{gray}{rgb}{0.5,0.5,0.5}
\definecolor{mauve}{rgb}{0.58,0,0.82}
\lstset{frame=tb,
  language=Python,
  aboveskip=3mm,
  belowskip=3mm,
  showstringspaces=false,
  columns=flexible,
  basicstyle={\small\ttfamily},
  numbers=none,
  numberstyle=\tiny\color{gray},
  keywordstyle=\color{blue},
  commentstyle=\color{dkgreen},
  stringstyle=\color{mauve},
  breaklines=true,
  breakatwhitespace=true,
  tabsize=4
}
\lstset{
  language=Mathematica,
  numbers=left,
  numberstyle=\tiny\color{gray},
  numbersep=5pt,
  breaklines=true,
  captionpos={t},
  frame={lines},
  rulecolor=\color{black},
  framerule=0.5pt,
  columns=flexible,
  tabsize=2
}


\title{Homework 2: Phys 7230 (Spring 2022)}
\due{February 7, 2022}

\begin{document}
\maketitle
%%%%%%%%%%%%%%%%%%%%%%%%%%%%%%%%%%%%%%%%%%%%%%%%%%%%%%%%%%%%%%%%%%%%%%%%%%%%%%%
\begin{problem}{1}[Statistical mechanics and thermodynamic relations]
(a) Given the basic definitions of the partition function and of the average
energy in the canonical ensemble, derive the average energy relation,
$E=-\partial(\ln Z)/\partial\beta$.

(b) Given the definition of Helmholtz free energy $F=-k_BT\ln Z$ and above
result for the average energy $E$, derive the relation $F=E-TS$. \textit{Hint}:
Note that the above expression for $F$ is equivalent to $\ln Z=-\beta F$ and
from thermodynamics $S=-\eval{\partial F/\partial T}_{N,V}$, a relation that you
will derive below.

(c) From the above Legendre transform relation (also discussed in the lectures)
between $E(S)$ and $F(T)$, and the 1st law of thermodynamics for e.g., a gas,
namely $dE=TdS-PdV+\mu dN$, derive $dF$ and identify from it the thermodynamic
expressions for $S, P$, and $\mu$, as well as the heat capacity $C_v$.

(d) In lecture, we discussed the equivalence of canonical and microcanonical
ensembles, with the key relation expressing the partition function $Z$ as an
integration over energies (rather than microstates) weighted by the density of
states $g(E)=\sum_\qty{q_i}\delta(E-H_q)$,
\begin{align}
    Z(\beta)&=\sum_\qty{q_i}e^{-\beta H_q}=\int
    dE\sum_\qty{q_i}\delta(E-H_q)e^{-\beta E},\notag\\
    &=\int dEg(E)e^{-\beta E},
\end{align}
related to multiplicity of the microcanonical ensemble, $\Omega(E)=\Delta g(E)$.

Recall our argument that $g(E)$ is generically an extremely fast growing
function of $E$ and therefore above integrand is highly peaked function around
average energy $E_0$, set by the peak. This condition is prime for the
saddle-point method evaluation of the integral over $E$.

\qquad(i) By applying the lowest order saddle-point approximation and using the
definition of $S$, show that this gives $Z\approx e^{-\beta F}$, where $F=E-TS$
from above and the standard thermodynamic relation.

\qquad(ii) By expanding a logarithm of the integrand in Taylor series to
quadratic order around its maximum $E_0$, derive the general expression for the
width of this peaked integrand, and use it to argue that indeed in the
thermodynamic limit the width is vanishingly small relative to
$E_0$. \textit{Hint}: (i) What happens to the 1st order term? (ii) What is the
general expression for heat capacity in the microcanonical ensemble?
\begin{solution}
(a) From the Boltzmann-Gibbs probability distribution, we can calculate the
average energy
\begin{align}
    E
    =\frac1Z\sum_\qty{q_i}H_qe^{-\beta H_q}
    =-\frac1Z\sum_\qty{q_i}\frac{\partial(e^{-\beta H_q})}{\partial\beta}
    =-\frac1Z\frac{\partial Z}{\partial\beta}
    =-\frac{\partial(\ln Z)}{\partial\beta}
\end{align}

(b) First note that we can write $\beta F=-\ln Z$. Plugging this into the
previous result yields
\begin{equation}
    F+\beta\frac{\partial F}{\partial\beta}=F+T\frac{\partial F}{\partial T}
    -\frac{\partial(\ln Z)}{\partial\beta}=E 
\end{equation}

(d) First, we write $Z(\beta)=\int_0^\infty dE e^{-f(E)}$ where
\begin{equation}
    f(E)=\beta E-\ln g(E)\approx \beta E_0-\ln g(E_0)+\frac12f''(E_0)(E-E_0)^2
\end{equation}
with $E_0$ satisfying $g'(E_0)=\beta g(E_0)$ and
\begin{equation}\label{p1d:fpp}
    f''(E_0)=\qty[\frac{g'(E_0)}{g(E_0)}]^2-\frac{g''(E_0)}{g(E_0)}
    =\beta^2-\frac{g''(E_0)}{g(E_0)}
\end{equation}
being a constant. (i) To the lowest order, we can then write
\begin{align}
    Z(\beta)
    &\approx g(E_0)e^{-\beta E_0}\int_0^\infty dE e^{-(1/2)f''(E-E_0)^2}\notag\\
    &=\sigma\sqrt{\frac\pi2} g(E_0)e^{-\beta E_0}\qty[
    \frac1{\sigma \sqrt{2\pi}}\int_{-\infty}^\infty
    dE\exp\qty[-\frac12\qty(\frac{E-E_0}{\sigma})^2]]\notag\\
    &=\sqrt{\frac\pi8}(2\sigma)g(E_0)e^{-\beta E_0}
\end{align}
where we have set $f''(E_0)=1/\sqrt\sigma$ and the integration of the Gaussian
distribution in the square bracket equates to unity. Now, note that the full
width of the (approximately) Gaussian distribution around $E=E_0$ is
$\Delta=2\sigma$. Letting $E=\expval{E}=E_0$, we can write
\begin{equation}
    Z(\beta)=\sqrt{\frac\pi8}\Omega(E)e^{-\beta E} 
    =\sqrt{\frac\pi8}e^{S/k_B-\beta E}
    =\sqrt{\frac\pi8}e^{\beta(TS-E)}
    =0.6 e^{-\beta F}
    \approx e^{-\beta F}
\end{equation}

(ii) The entropy is $S(E)=k_B\ln\Omega=k_B\ln[\Delta g(E)]$. So by calculating
the temperature,
\begin{equation}
    \beta=\frac{\partial(S/k_B)}{\partial E} 
    =\frac{g'(E)}{g(E)}
\end{equation}
Differentiating once more, we get
\begin{equation}
    g''(E)=\beta g'(E)+g(E)\frac{\partial}{\partial E}\qty(\frac1{k_BT})
    =\beta g'(E)-\beta^2k_Bg(E)\frac{\partial T}{\partial E}
    =\beta g'(E)-g(E)\frac{\beta^2k_B}{C_v}
\end{equation}
Then, from \eqref{p1d:fpp},
\begin{equation}
    \frac1{\sqrt\sigma}=f''(E)=\beta^2\frac{k_B}{C_v}
\end{equation}
But from equipartition theorem, $C_v=Nk_B$ and $E=(N/2)k_BT$. Thus, the width of
the peak is
\begin{equation}
    \sigma=\frac1\beta\sqrt{\frac{C_v}{k_B}}=\frac{2E}{\sqrt{N}}\Rightarrow
    \frac{\Delta}{E}=\sqrt{\frac2N}\to 0
\end{equation}
in the thermodynamic limit ($N\gg 1$).
\end{solution}
\newpage
\end{problem}
%%%%%%%%%%%%%%%%%%%%%%%%%%%%%%%%%%%%%%%%%%%%%%%%%%%%%%%%%%%%%%%%%%%%%%%%%%%%%%%
%%%%%%%%%%%%%%%%%%%%%%%%%%%%%%%%%%%%%%%%%%%%%%%%%%%%%%%%%%%%%%%%%%%%%%%%%%%%%%%
\begin{problem}{2}
\begin{solution}
\end{solution}
\end{problem}
%%%%%%%%%%%%%%%%%%%%%%%%%%%%%%%%%%%%%%%%%%%%%%%%%%%%%%%%%%%%%%%%%%%%%%%%%%%%%%%
    
\end{document}
