\documentclass[12pt]{article}

%%%%%%%%%%%%%%%%%%%%%%%%%%%%%%%%%%%%%%%%%%%%%%%%%%%%%%%%%%%%%%%%%%%%%%%%%%%%%%%%
%                           Package preset for homework
%%%%%%%%%%%%%%%%%%%%%%%%%%%%%%%%%%%%%%%%%%%%%%%%%%%%%%%%%%%%%%%%%%%%%%%%%%%%%%%%
% Miscellaneous
\usepackage[margin=1in]{geometry}
\usepackage[utf8]{inputenc}
\usepackage{indentfirst}
\usepackage{blindtext}
\usepackage{graphicx}
\usepackage{xr-hyper}
\usepackage{hyperref}
\usepackage{color}
\usepackage{float}
% Math
\usepackage{latexsym}
\usepackage{amsfonts}
\usepackage{amssymb}
\usepackage{amsmath}
\usepackage{commath}
\usepackage{amsthm}
\usepackage{bbold}
\usepackage{bm}
% Physics
\usepackage{physics}
\usepackage{siunitx}
% Code typesetting
\usepackage{listings}
% Citation
\usepackage[authoryear]{natbib}
\usepackage{appendix}
\usepackage[capitalize]{cleveref}
% Title & name
\title{Homework}
\author{Tien Vo}
\date{\today}


%%%%%%%%%%%%%%%%%%%%%%%%%%%%%%%%%%%%%%%%%%%%%%%%%%%%%%%%%%%%%%%%%%%%%%%%%%%%%%%%
%                   User-defined commands and environments
%%%%%%%%%%%%%%%%%%%%%%%%%%%%%%%%%%%%%%%%%%%%%%%%%%%%%%%%%%%%%%%%%%%%%%%%%%%%%%%%
%%% Misc
\sisetup{load-configurations=abbreviations}
\newcommand{\due}[1]{\date{Due: #1}}
\newcommand{\hint}{\textit{Hint}}
\let\oldt\t
\renewcommand{\t}[1]{\text{#1}}

%%% Bold sets & abbrv
\newcommand{\N}{\mathbb{N}}
\newcommand{\Z}{\mathbb{Z}}
\newcommand{\R}{\mathbb{R}}
\newcommand{\Q}{\mathbb{Q}}
\let\oldP\P
\renewcommand{\P}{\mathbb{P}}
\newcommand{\LL}{\mathcal{L}}
\newcommand{\FF}{\mathcal{F}}
\newcommand{\HH}{\mathcal{H}}
\newcommand{\NN}{\mathcal{N}}
\newcommand{\ZZ}{\mathcal{Z}}
\newcommand{\RN}[1]{\textup{\uppercase\expandafter{\romannumeral#1}}}
\newcommand{\ua}{\uparrow}
\newcommand{\da}{\downarrow}

%%% Unit vectors
\newcommand{\xhat}{\vb{\hat{x}}}
\newcommand{\yhat}{\vb{\hat{y}}}
\newcommand{\zhat}{\vb{\hat{z}}}
\newcommand{\nhat}{\vb{\hat{n}}}
\newcommand{\rhat}{\vb{\hat{r}}}
\newcommand{\phihat}{\bm{\hat{\phi}}}
\newcommand{\thetahat}{\bm{\hat{\theta}}}

%%% Other math stuff
\providecommand{\units}[1]{\,\ensuremath{\mathrm{#1}}\xspace}
% Set new style for problem
\newtheoremstyle{problemstyle}  % <name>
        {10pt}                   % <space above>
        {10pt}                   % <space below>
        {\normalfont}           % <body font>
        {}                      % <indent amount}
        {\bfseries\itshape}     % <theorem head font>
        {\normalfont\bfseries:} % <punctuation after theorem head>
        {.5em}                  % <space after theorem head>
        {}                      % <theorem head spec (can be left empty, 
                                % meaning `normal')>

% Set problem environment
\theoremstyle{problemstyle}
\newtheorem{problemenv}{Problem}[section]
\newenvironment{problem}[1]{%
  \renewcommand\theproblemenv{#1}%
  \problemenv
}{\endproblemenv}
% Set lemma environment
\newenvironment{lemma}[2][Lemma]{\begin{trivlist}
\item[\hskip \labelsep {\bfseries #1}\hskip \labelsep {\bfseries #2.}]}{\end{trivlist}}
% Set solution environment
\newenvironment{solution}{
    \begin{proof}[Solution]$ $\par\nobreak\ignorespaces
}{\end{proof}}
\numberwithin{equation}{problemenv}

%%% Page format
\setlength{\parindent}{0.5cm}
\setlength{\oddsidemargin}{0in}
\setlength{\textwidth}{6.5in}
\setlength{\textheight}{8.8in}
\setlength{\topmargin}{0in}
\setlength{\headheight}{18pt}

%%% Code environments
\definecolor{dkgreen}{rgb}{0,0.6,0}
\definecolor{gray}{rgb}{0.5,0.5,0.5}
\definecolor{mauve}{rgb}{0.58,0,0.82}
\lstset{frame=tb,
  language=Python,
  aboveskip=3mm,
  belowskip=3mm,
  showstringspaces=false,
  columns=flexible,
  basicstyle={\small\ttfamily},
  numbers=none,
  numberstyle=\tiny\color{gray},
  keywordstyle=\color{blue},
  commentstyle=\color{dkgreen},
  stringstyle=\color{mauve},
  breaklines=true,
  breakatwhitespace=true,
  tabsize=4
}
\lstset{
  language=Mathematica,
  numbers=left,
  numberstyle=\tiny\color{gray},
  numbersep=5pt,
  breaklines=true,
  captionpos={t},
  frame={lines},
  rulecolor=\color{black},
  framerule=0.5pt,
  columns=flexible,
  tabsize=2
}


\title{Homework 3: Phys 7230 (Spring 2022)}
\due{February 21, 2022}

\begin{document}
\maketitle
%%%%%%%%%%%%%%%%%%%%%%%%%%%%%%%%%%%%%%%%%%%%%%%%%%%%%%%%%%%%%%%%%%%%%%%%%%%%%%%
\begin{problem}{1}[Derivation of ensembles]~\\
(a) Grand canonical ensemble~\\
Start with the expression
\begin{equation}
    W\qty[\qty{n_q}]=\frac{\mathcal{N}!}{\prod_qn_q!} 
\end{equation}
for the number of configurations in a grandcanonical ensemble consisting of
$\mathcal{N}$ systems with a set $\qty{n_q}$ describing the number of systems
with energy $E_q$ and number of particles $N_q$, discussed in lectures.
Utilizing lowest order Stirling approximation, maximize $W\qty[\qty{n_q}]$ over
$n_q$, subject to three constraints,
\begin{equation}
    \sum_qn_q=\NN,\qquad\sum_qn_qE_q=E\NN,\qquad\sum_qn_qN_q=N\NN ,
\end{equation}
imposed via Lagrange multiplier $\gamma,\beta,\alpha,$ with $E$ and $N$ the
average energy and particle number in the ensemble, derive the most likely
$n_q^\ast$ and thereby obtain the grandcanonical probability distribution
$P_q=n_q^\ast/\NN$.

\textit{Hint}: Maximizing $\ln W$ may be more convenient.
\begin{solution}
First, using Stirling approximation, we write
\begin{equation}
    \ln W=\NN\ln\NN -\NN-\sum_q\qty(n_q\ln n_q - n_q)
    =\NN\ln\NN-\sum_qn_q\ln n_q.
\end{equation}
Extremizing this function, we calculate
\begin{align}
    \grad(\ln W)
    &-\gamma\grad\qty(\sum_qn_q-\NN)
    -\beta\grad\qty(\sum_qn_qE_q-E\NN)
    -\alpha\grad\qty(\sum_qn_qN_q-N\NN)\notag\\
    \qquad&=\ln\NN-1-\ln n_q+1+\beta(E-E_q)+\alpha(N-N_q)=0.
\end{align}
Inverting, we get
\begin{equation}\label{p1a:n_q}
    \frac{n_q}{\NN}=e^{\beta(E-E_q)+\alpha(N-N_q)}.
\end{equation}
By the normalization condition,
\begin{equation}
    1=\sum_q\frac{n_q}{\NN}=e^{\beta E+\alpha N}\sum_qe^{-\beta E_q-\alpha N_q}
    =e^{\beta E+\alpha N}\ZZ.
\end{equation}
where $\beta=1/k_BT$ and $\alpha=-\mu\beta$. Then, letting $n_q=n_q^\ast$, the
most probable state satisfying \eqref{p1a:n_q}, we can write the probability
distribution as
\begin{equation}
    P_q=\frac{n_q^\ast}{\NN}=\frac{e^{-\beta E_q-\alpha N_q}}\ZZ  .
\end{equation}
\end{solution}
%%%%%%   

(b) Derivation redux

Let us rederive all three ensembles distributions in a more streamlined way by
focusing on $P_q$ and noting that $P_q$ can be determined by maximizing the
(Shannon's) entropy $S=-k_B\sum_qP_q\ln P_q$ subject to the appropriate number
of constraints for each ensemble. Thus, derive

\qquad(i) Microcanonical ensemble subject to its one constraint, showing that it
is just given by a normalized \textit{constant} $P_q=1/\Omega$.

\qquad(ii) Canonical ensemble subject to its two constraints, showing that it 
is given by the Gibbs form.

\qquad(iii) Grandcanonical ensemble subject to its three constraints, showing
that it is given by the Gibbs form.
\begin{solution}
\qquad(i) The constraint for the microcanonical ensemble is the normalization
condition
\begin{equation}\label{p1b:normalization_condition}
    \sum_qP_q=1 .
\end{equation}
Then, we differentiate to find $P_q$ that extremizes $S$
\begin{equation}
   \grad(S)-\lambda\grad\qty(\sum_qP_q-1)
   =-k_B\qty(\ln P_q+1)-\lambda_1
   \Rightarrow P_q=e^{-1-\lambda/k_B}.
\end{equation}
Plugging this back into the normalization condition, we get
\begin{equation}
    \sum_qP_q=e^{-1-\lambda_1/k_B}\sum_q1
    =e^{-1-\lambda_1/k_B}\Omega,
\end{equation}
where $\Omega=\sum_\qty{q_i}1$ is the multiplicity of the system. Then we can
rewrite the probability distribution function
\begin{equation}
    P_q=\frac1\Omega, 
\end{equation}
which is a constant, indicating a uniform distribution. Also, we can invert to
solve for $\lambda_1$
\begin{equation}
    \lambda_1=S_t-k_B, 
\end{equation}
where $S_t=k_B\ln\Omega$ is the thermodynamic entropy.

\qquad(ii) In addition to the normalization condition
\eqref{p1b:normalization_condition}, we add a constraint
\begin{equation}
    \sum_qE_qP_q=E, 
\end{equation}
where $E$ is the average energy of the system. Then, the $P_q$ that extremizes
$S$ satisfies
\begin{equation}
    -k_B\qty(\ln P_q+1)-\lambda_1-\lambda_2E_q=0
    \Rightarrow P_q=e^{-1-\lambda_1/k_B-\lambda_2E_q/k_B}.
\end{equation}
Applying the normalization,
\begin{equation}
    1=\sum_qP_q=e^{-1-\lambda_1/k_B}\sum_qe^{-\lambda_2E_q/k_B}
    = e^{-1-\lambda_1/k_B}\sum_qe^{-\beta E_q}
    \Rightarrow e^{-1-\lambda_1/k_B}=\frac1{Z},
\end{equation}
where we have let the constant $\lambda_2=1/T$, with $T$ the equilibrium
temperature, and $Z$ is the partition function of the canonical ensemble. It
then follows that the distribution function
\begin{equation}
    P_q=\frac{e^{-\beta E_q}}{Z} 
\end{equation}
has the Gibbs form.

(iii) The third constraint to add is
\begin{equation}
    \sum_qN_qP_q=N,
\end{equation}
where $N$ is the average number of particles per partition. Then $P_q$
extremizing $S$ satisfies
\begin{equation}
    -k_B\qty(\ln P_q+1)-\lambda_1-\lambda_2E_q-\lambda_3N_q
    =-k_B\ln P_q-S_t-E_q/T-\lambda_3 N_q=0,
\end{equation}
where we have replaced the first two Lagrange multipliers with physical
quantities derived in (i-ii). The probability distribution function is
\begin{equation}
    P_q=e^{-S-\beta E_q-\lambda_3N_q/k_B}=e^{-S-\beta E_q+\beta\mu N_q},
\end{equation}
where we have let the constant $\lambda_3=-\mu/T$. Then, with the normalization
condition,
\begin{equation}
    1=e^{-S}\sum_qe^{-\beta(E_q+\mu N_q)}=e^{-S}\ZZ, 
\end{equation}
and the probability distribution function can be written as
\begin{equation}
    P_q=\frac{e^{-\beta(E_q+\mu N_q)}}{\ZZ}, 
\end{equation}
which has the Gibbs form.
\end{solution}

%%%%%%   

(c) Recall the expression for $\Omega(E)$ for the microcanonical ensemble of a
3d ideal Boltzmann gas of $N$ particles. Using it, calculate the probability
$P_\epsilon$ of a \textit{particular} particle to have energy in the
neighborhood of $\epsilon\ll E$.

From your answer for $P_\epsilon$ and comparing it with the standard
Boltzmann-Gibbs weight, read off the corresponding effective temperature of this
single particle in terms of $E$ and $N$.

\textit{Hint}: (i) From above total $\Omega(E)$ first think about the number of
states (sub-multiplicity) $\Omega_\epsilon(E)$ for one of the particle to have
energy $\epsilon$ and the remaining $N-1$ sharing the remaining energy. (ii)
Note that $P_\epsilon=\Omega_\epsilon/\Omega$. (iii) Use Stirling's
approximation and $\epsilon\ll E$ to simplify your answer.

\begin{solution}
From Sackur-Tetrode, we can write the multiplicity as $\Omega=e^{S/k_B}
=(V^N/N!)f(N,E)$ where
\begin{equation}
    f(N,E)=e^{5N/2}\qty(\frac{4\pi mE}{3Nh^2})^{3N/2}.
\end{equation}
Then, we note that there is only one way to distribute an energy of $\epsilon$
into a particular particle. Thus, the multiplicity for the entire system of one
particle of energy $\epsilon$ and $N-1$ particles of energy $E-\epsilon$ is
\begin{align}
    \Omega_\epsilon
    &=\frac{V^{N-1}}{(N-1)!}f(N-1,E-\epsilon)\notag\\
    &=\frac{V^{N-1}}{(N-1)!}e^{5(N-1)/2}\qty(\frac{4\pi
    m(E-\epsilon)}{3(N-1)h^2})^{3(N-1)/2}\notag\\
    &\approx\frac{V^{N-1}}{(N-1)!}e^{5(N-1)/2}\qty(\frac{4\pi
    m(E-\epsilon)}{3Nh^2})^{3(N-1)/2}.
\end{align}
The probability is thus
\begin{align}
    P_\epsilon
    &=\frac{\Omega_\epsilon}{\Omega}\notag\\
    &\approx\frac{N}{V}e^{-5/2}\qty(\frac{4\pi m E}{3Nh^2})^{-3/2}
        \qty(1-\frac{\epsilon}{E})^{3(N-1)/2}\notag\\
    &\approx\frac{N}{V}e^{-5/2}\qty(\frac{4\pi m E}{3Nh^2})^{-3/2}
        e^{-3(N-1)\epsilon/2E},
\end{align}
which follows the Gibbs form if the effective temperature is defined as
\begin{equation}
    k_BT_\text{eff}=\frac{2E}{3(N-1)}.
\end{equation}
\end{solution}

%%%%%%   

(d) Variance (i.e., mean-squared fluctuations) in the number of particles $N$ in
the grand canonical ensemble is given by $\expval{(\Delta N)^2}$. Show that
quite generally it is given by
\begin{equation}
    N_\text{rms}^2=\expval{\qty(\Delta
    N)^2}=-\eval{\frac{\partial\overline{N}}{\partial\alpha}}_{\beta,V}
    =\eval{\frac{\partial^2(\ln\mathcal{Z})}{\partial\alpha^2}}_{\beta,V}
\end{equation}
\begin{solution}
First, from $\ZZ$, we can calculate
\begin{align}\label{p1d:dZda}
    \frac{\partial\ZZ}{\partial\alpha}
    =\frac{\partial}{\partial\alpha}\sum_qe^{-\beta E_q-\alpha N_q}
    =-\sum_qN_qe^{-\beta E_q-\alpha N_q}
    =-\expval{N}\ZZ,
\end{align}
and
\begin{align}
    \frac{\partial^2\ZZ}{\partial\alpha^2}
    =\sum_qN_q^2e^{-\beta E_q-\alpha N_q}
    =\expval{N^2}\ZZ.
\end{align}
Differentiating \eqref{p1d:dZda} again, we get
\begin{equation}
    \expval{N^2}\ZZ=-\frac{\partial\expval{N}}{\partial\alpha}\ZZ
    +\expval{N}^2\ZZ
    \Rightarrow\expval{(\Delta
    N)^2}=\expval{N^2}-\expval{N}^2=-\frac{\partial\expval{N}}{\partial\alpha}.
\end{equation}
Also,
\begin{equation}
    \frac{\partial(\ln\ZZ)}{\partial\alpha}=-\expval{N}.
\end{equation}
So taking the differentation again leads us to the second equality
\begin{equation}
    \frac{\partial^2(\ln\ZZ)}{\partial\alpha^2}=-\frac{\partial\expval{N}}{\partial\alpha}
    .
\end{equation}
\end{solution}
\end{problem}
\newpage
%%%%%%%%%%%%%%%%%%%%%%%%%%%%%%%%%%%%%%%%%%%%%%%%%%%%%%%%%%%%%%%%%%%%%%%%%%%%%%%   
%%%%%%%%%%%%%%%%%%%%%%%%%%%%%%%%%%%%%%%%%%%%%%%%%%%%%%%%%%%%%%%%%%%%%%%%%%%%%%%
\begin{problem}{2}[Ultra-relativistic gas]
Consider a non-interacting 3d gas of identical ultra-relativistic particles
(i.e., ignoring their mass), with a dispersion $\epsilon_i=p_ic$, at temperature
$T$ and chemical potential $\mu$, where
$p_i=\abs{\vb{p}_i}=\sqrt{p_{xi}^2+p_{yi}^2+p_{zi}^2}$.

Following similar steps that we did in class for nonrelativistic Boltzmann gas,
compute:

(a) Grandcanonical partition function, $\ZZ(\mu,T)$ by first computing the
canonical one, $Z(N,T)$, introducing fugacity and then summing over $N$.

\textit{Hint}: it is nice to do the integral in the spherical coordinate system.

\begin{solution}
    
\end{solution}

(b) Grandcanonical free energy $\FF(\mu,T)$.

\begin{solution}
\end{solution}
\end{problem}
\newpage
%%%%%%%%%%%%%%%%%%%%%%%%%%%%%%%%%%%%%%%%%%%%%%%%%%%%%%%%%%%%%%%%%%%%%%%%%%%%%%%
\end{document}
