\documentclass[12pt]{article}

%%%%%%%%%%%%%%%%%%%%%%%%%%%%%%%%%%%%%%%%%%%%%%%%%%%%%%%%%%%%%%%%%%%%%%%%%%%%%%%%
%                           Package preset for homework
%%%%%%%%%%%%%%%%%%%%%%%%%%%%%%%%%%%%%%%%%%%%%%%%%%%%%%%%%%%%%%%%%%%%%%%%%%%%%%%%
% Miscellaneous
\usepackage[margin=1in]{geometry}
\usepackage[utf8]{inputenc}
\usepackage{indentfirst}
\usepackage{blindtext}
\usepackage{graphicx}
\usepackage{xr-hyper}
\usepackage{hyperref}
\usepackage{color}
\usepackage{float}
% Math
\usepackage{latexsym}
\usepackage{amsfonts}
\usepackage{amssymb}
\usepackage{amsmath}
\usepackage{commath}
\usepackage{amsthm}
\usepackage{bbold}
\usepackage{bm}
% Physics
\usepackage{physics}
\usepackage{siunitx}
% Code typesetting
\usepackage{listings}
% Citation
\usepackage[authoryear]{natbib}
\usepackage{appendix}
\usepackage[capitalize]{cleveref}
% Title & name
\title{Homework}
\author{Tien Vo}
\date{\today}


%%%%%%%%%%%%%%%%%%%%%%%%%%%%%%%%%%%%%%%%%%%%%%%%%%%%%%%%%%%%%%%%%%%%%%%%%%%%%%%%
%                   User-defined commands and environments
%%%%%%%%%%%%%%%%%%%%%%%%%%%%%%%%%%%%%%%%%%%%%%%%%%%%%%%%%%%%%%%%%%%%%%%%%%%%%%%%
%%% Misc
\sisetup{load-configurations=abbreviations}
\newcommand{\due}[1]{\date{Due: #1}}
\newcommand{\hint}{\textit{Hint}}
\let\oldt\t
\renewcommand{\t}[1]{\text{#1}}

%%% Bold sets & abbrv
\newcommand{\N}{\mathbb{N}}
\newcommand{\Z}{\mathbb{Z}}
\newcommand{\R}{\mathbb{R}}
\newcommand{\Q}{\mathbb{Q}}
\let\oldP\P
\renewcommand{\P}{\mathbb{P}}
\newcommand{\LL}{\mathcal{L}}
\newcommand{\FF}{\mathcal{F}}
\newcommand{\HH}{\mathcal{H}}
\newcommand{\NN}{\mathcal{N}}
\newcommand{\ZZ}{\mathcal{Z}}
\newcommand{\RN}[1]{\textup{\uppercase\expandafter{\romannumeral#1}}}
\newcommand{\ua}{\uparrow}
\newcommand{\da}{\downarrow}

%%% Unit vectors
\newcommand{\xhat}{\vb{\hat{x}}}
\newcommand{\yhat}{\vb{\hat{y}}}
\newcommand{\zhat}{\vb{\hat{z}}}
\newcommand{\nhat}{\vb{\hat{n}}}
\newcommand{\rhat}{\vb{\hat{r}}}
\newcommand{\phihat}{\bm{\hat{\phi}}}
\newcommand{\thetahat}{\bm{\hat{\theta}}}

%%% Other math stuff
\providecommand{\units}[1]{\,\ensuremath{\mathrm{#1}}\xspace}
% Set new style for problem
\newtheoremstyle{problemstyle}  % <name>
        {10pt}                   % <space above>
        {10pt}                   % <space below>
        {\normalfont}           % <body font>
        {}                      % <indent amount}
        {\bfseries\itshape}     % <theorem head font>
        {\normalfont\bfseries:} % <punctuation after theorem head>
        {.5em}                  % <space after theorem head>
        {}                      % <theorem head spec (can be left empty, 
                                % meaning `normal')>

% Set problem environment
\theoremstyle{problemstyle}
\newtheorem{problemenv}{Problem}[section]
\newenvironment{problem}[1]{%
  \renewcommand\theproblemenv{#1}%
  \problemenv
}{\endproblemenv}
% Set lemma environment
\newenvironment{lemma}[2][Lemma]{\begin{trivlist}
\item[\hskip \labelsep {\bfseries #1}\hskip \labelsep {\bfseries #2.}]}{\end{trivlist}}
% Set solution environment
\newenvironment{solution}{
    \begin{proof}[Solution]$ $\par\nobreak\ignorespaces
}{\end{proof}}
\numberwithin{equation}{problemenv}

%%% Page format
\setlength{\parindent}{0.5cm}
\setlength{\oddsidemargin}{0in}
\setlength{\textwidth}{6.5in}
\setlength{\textheight}{8.8in}
\setlength{\topmargin}{0in}
\setlength{\headheight}{18pt}

%%% Code environments
\definecolor{dkgreen}{rgb}{0,0.6,0}
\definecolor{gray}{rgb}{0.5,0.5,0.5}
\definecolor{mauve}{rgb}{0.58,0,0.82}
\lstset{frame=tb,
  language=Python,
  aboveskip=3mm,
  belowskip=3mm,
  showstringspaces=false,
  columns=flexible,
  basicstyle={\small\ttfamily},
  numbers=none,
  numberstyle=\tiny\color{gray},
  keywordstyle=\color{blue},
  commentstyle=\color{dkgreen},
  stringstyle=\color{mauve},
  breaklines=true,
  breakatwhitespace=true,
  tabsize=4
}
\lstset{
  language=Mathematica,
  numbers=left,
  numberstyle=\tiny\color{gray},
  numbersep=5pt,
  breaklines=true,
  captionpos={t},
  frame={lines},
  rulecolor=\color{black},
  framerule=0.5pt,
  columns=flexible,
  tabsize=2
}


\title{Homework 1: Phys 7320 (Spring 2022)}
\due{January 19, 2022}

\begin{document}
\maketitle
%%%%%%%%%%%%%%%%%%%%%%%%%%%%%%%%%%%%%%%%%%%%%%%%%%%%%%%%%%%%%%%%%%%%%%%%%%%%%%%
\begin{problem}{1.1}[Radiating spherical metallic shell]
Two halves of a spherical metallic shell of radius $R$ and infinite conductivity
are separated by a very small insulating gap. An alternating potential is
applied between the two halves of the sphere so that the potentials are $\pm
V\cos\omega t$. In the long-wavelength limit, find the radiation fields, the
angular distribution of radiated power, and the total radiated power from the
sphere.
\begin{solution}
Under Fourier analysis, the potential at the boundary ($r=R$) is given by
\begin{equation}
    V(\theta)=\begin{cases}
        Ve^{-i\omega t} & 0\leq\theta < \pi/2\\
        -Ve^{-i\omega t}&\pi/2<\theta\leq\pi
    \end{cases} 
\end{equation}
Thus, from Section 3.3 in Jackson, the solution for the potential is
\begin{align}\label{p1:Phi}
    \Phi(r,\theta,t)
    &=\frac{Ve^{-i\omega t}}{\sqrt\pi}
        \sum_{j=1}^\infty(-1)^{j-1}\frac{(2j-1/2)\Gamma(j-1/2)}{j!}
        \qty(\frac{R}{r})^{2j}P_{2j-1}(\cos\theta)\notag\\
    &\approx\frac32V\qty(\frac{R}{r})^2\cos\theta e^{-i\omega t}
\end{align}
Comparing this with (4.10, Jackson), only the dipole term grows as $r^{-2}$
\begin{equation}
    \Phi_\text{dipole}=\frac1{4\pi\epsilon_0}\frac{\vb{p}\vdot\vb{x}}{r^3}
    =\frac1{4\pi\epsilon_0}\frac{p\cos\theta}{r^2}
\end{equation}
since in this instance, we know that $\vb{p}=p\zhat$ from geometry. Thus, we can
write from \eqref{p1:Phi} that
\begin{equation}
    \vb{p}=6\pi\epsilon_0R^2Ve^{-i\omega t}\zhat
\end{equation}
Then from (9.19, Jackson), we can calculate the magnetic field
\begin{align}
    \vb{B}
    =\mu_0\vb{H}
    =\frac{c\mu_0k^2}{4\pi}\qty(\vb{n}\times\vb{p})\frac{e^{ikr}}{r}
    =\frac32\frac{k^2R^2V}{c}\qty(\vb{n}\times\zhat)\frac{e^{i(kr-\omega
    t)}}{r}
\end{align}
Note that by definition, $\vb{n}=\vb{x}/r$. So
$\vb{n}\times\zhat=-\sin\theta\cos\phi\yhat+\sin\theta\sin\phi\xhat=-\sin\theta\phihat$.
The magnetic field is then
\begin{equation}
    \vb{B}=-\frac32\frac{k^2R^2V}{c}\frac{e^{i(kr-\omega
    t)}}{r}\sin\theta\phihat 
\end{equation}
The second equation in (9.19, Jackson) gives the electric field
\begin{align}
    \vb{E}&=\frac{Z_0}{\mu_0}\vb{B}\times\vb{n}\notag\\
    &=cB\qty(\sin\phi\xhat-\cos\phi\yhat)\times\qty(\sin\theta\cos\phi\xhat+\sin\theta\sin\phi\yhat+\cos\theta\zhat)\notag\\
    &=cB\qty(-\cos\theta\rhat+\sin\theta\zhat)\notag\\
    &=-cB\thetahat\notag\\
    &=-\frac32k^2R^2V\frac{e^{i(kr-\omega t)}}{r}\sin\theta\thetahat
\end{align}

Then, the angular distribution of radiated power is
\begin{align}
    \frac{dP}{d\Omega}
    &=\frac12\Re\qty{r^2\vb{n}\vdot\qty(\vb{E}\times\frac{\vb{B}^\ast}{\mu_0})}
        \notag\\
    &=\frac1{2\mu_0}\Re\qty{r^2\rhat\vdot\qty[\frac94\frac{k^4R^4V^2}{c}\sin^2\theta\thetahat\times\phihat]}\notag\\
    &=\frac98\frac{k^4R^4V^2}{c\mu_0}\sin^2\theta\Re\qty{\rhat\vdot\rhat}\notag\\
    &=\frac98\frac{k^4R^4V^2}{c\mu_0}\sin^2\theta
\end{align}
and the total radiated power is
\begin{align}
    P=\int\frac{dP}{d\Omega}d\Omega
    =\frac98\frac{k^4R^4V^2}{c\mu_0}\int_0^\pi\sin^3\theta
    d\theta\int_0^{2\pi}d\phi
    =\frac{3\pi k^4R^4V^2}{c\mu_0}
\end{align}
\end{solution}
\end{problem}
%%%%%%%%%%%%%%%%%%%%%%%%%%%%%%%%%%%%%%%%%%%%%%%%%%%%%%%%%%%%%%%%%%%%%%%%%%%%%%%    
%%%%%%%%%%%%%%%%%%%%%%%%%%%%%%%%%%%%%%%%%%%%%%%%%%%%%%%%%%%%%%%%%%%%%%%%%%%%%%%
\begin{problem}{1.2}[Array of dipoles]
Consider an array of $2N+1$ dipoles. Each has a time-dependent dipole moment
\begin{equation}
    \vb{p}(t)=p\zhat e^{-i\omega t}, 
\end{equation}
and they are spaced each a distance $D$ apart along the $z$-axis, centered on
the origin. Find the resulting antenna pattern $dP/d\Omega$ at $r\gg D$. You may
assume that the wavelength is large compared to each individual dipole, but do
not assume it is large compared to the spacing between dipoles (that is, do not
assume $\lambda\gg D$.) Explicitly work out what happens at $\theta\approx\pi/2$
where $\theta$ is the usual angle from the $z$-axis in spherical coordinates.

This is an iconic probem: many similar radiators distributed in space, so the
antenna pattern tells you things about the spatial distribution of radiators.
\begin{solution}
For $n\in\Z$ and $-N\leq n\leq N$, let $\vb{A}_n$ be the contribution of the
dipole located at $\vb{z}_n=nD\zhat$ to the vector potential. From (9.16,
Jackson),
\begin{equation}
    \vb{A}_n=-i\omega\frac{\mu_0}{4\pi}\vb{p}\frac{e^{ikr_n}}{r_n} 
\end{equation}
where $r_n=\abs{\vb{x}-\vb{z}_n}\approx r-nD\cos\theta$. In the denominator, we
can write $r_n\approx r$, but we need to keep the $kD$ term since it is not
necessarily that $kD\ll 1$. Thus,
\begin{equation}
    \vb{A}_n\approx-i\omega\frac{\mu_0}{4\pi}\vb{p}\frac{e^{ikr}}{r}e^{-inkD\cos\theta} 
\end{equation}
The total vector potential is then
\begin{equation}
    \vb{A}=-i\omega\frac{\mu_0}{4\pi}\vb{p}\frac{e^{ikr}}{r}\sum_{n=-N}^Ne^{-inkD\cos\theta}
    =-i\omega\frac{\mu_0}{4\pi}\vb{p}\frac{e^{ikr}}{r}\frac{\sin\qty[(2N+1)kD\cos\theta/2]}{\sin\qty[kD\cos\theta/2]}
\end{equation}
Let
\begin{equation}
    C=\frac{\sin\qty[(2N+1)kD\cos\theta/2]}{\sin\qty[kD\cos\theta/2]}
\end{equation}
Then the vector potential has a form similar to (9.16, Jackson),
\begin{equation}
    \vb{A}=-i\omega\frac{\mu_0}{4\pi}\vb{\overline{p}}\frac{e^{ikr}}{r} 
\end{equation}
with $\overline{\vb{p}}=C\vb{p}$. Thus, the angular distribution of radiated
power in the radiation zone is, from (9.23),
\begin{equation}
    \frac{dP}{d\Omega}=\frac{c^2Z_0}{32\pi^2}k^4\overline{p}^2\sin^2\theta
    =\frac{c^2Z_0}{32\pi^2}k^4p^2\sin^2\theta\frac{\sin^2\qty[(2N+1)kD\cos\theta/2]}{\sin^2\qty[kD\cos\theta/2]}
\end{equation}
At small angle, $\sin x\to x$. Thus, as $\theta\to\pi/2$, $\cos\theta\to0$ and
we can write
\begin{equation}
    \frac{dP}{d\Omega}(\theta\approx\pi/2)=(2N+1)^2\frac{c^2Z_0}{32\pi^2}k^4p^2\sin^2\theta 
\end{equation}
\end{solution}
\end{problem}
%%%%%%%%%%%%%%%%%%%%%%%%%%%%%%%%%%%%%%%%%%%%%%%%%%%%%%%%%%%%%%%%%%%%%%%%%%%%%%%
%%%%%%%%%%%%%%%%%%%%%%%%%%%%%%%%%%%%%%%%%%%%%%%%%%%%%%%%%%%%%%%%%%%%%%%%%%%%%%%
\begin{problem}{1.3}[Almost a sphere]
(a) The surface of a charge distribution of uniform density is almost a sphere:
the radius as a function of the polar angle is
\begin{equation}
    R(\theta)=R_0(1+\gamma\cos\theta). 
\end{equation}
The quantity $\gamma$ oscillates in time with frequency $\omega$, so we are
describing something like surface waves on a water balloon. Working to lowest
nontrivial order in the small parameter $\gamma$, and in the long wavelength
limit, first find an expression for the total charge within the surface $Q$, and
then find an expression for the antenna pattern of emitted radiation
$dP/d\Omega$ and the total power radiated $P$ with these answers written in
terms of $Q$.

(b) The previous part is a variation of Jackson 9.12, except in that problem
$R(\theta)=R_0(1+\beta P_2(\cos\theta))$. What is the leading multipole behavior
of the radiation in that case? Why? (You don't have to calculate it -- just say
which kind of multipole it is and why.)
\begin{solution}
(a) Assuming a uniform density $\rho$, the total charge is
\begin{align}\label{p3:Q}
    Q
    &=\int\rho d^3x' \notag\\
    &=2\pi\rho\int_0^\pi\sin\theta'd\theta'\int_0^{R(\theta)}r'^2dr'\notag\\
    &=\frac{2\pi R_0^3}3\rho
    \int_0^\pi\sin\theta'\qty(1+\gamma\cos\theta')^3d\theta'
    \notag\\
    &=\frac{4\pi R_0^3}{3}\rho\qty(1+\gamma^2)
\end{align}

Now, we calculate the dipole moments. The first two components are trivial
\begin{align}
    p_x&=\int x'\rho d^3x'\propto\int_0^{2\pi}\cos\phi'd\phi'=0\notag\\
    p_y&=\int y'\rho d^3x'\propto\int_0^{2\pi}\sin\phi'd\phi'=0
\end{align}
So the moment $\vb{p}=p_z\zhat$ with
\begin{align}
    p_z&=\int z'\rho d^3x'\notag\\
       &=2\pi\rho\int_0^\pi\sin\theta'\cos\theta'd\theta'
        \int_0^{R(\theta)}r'^3dr'\notag\\
       &=\frac{\pi
       R_0^4}{2}\rho\int_0^{\pi}\sin\theta'\cos\theta'(1+\gamma\cos\theta')^4d\theta'\notag\\
       &=\frac{4\pi R_0^4}{15}\rho\gamma(5+3\gamma^2)\notag\\
       &=\frac{QR_0}{5}\frac{\gamma(5+3\gamma^2)}{1+\gamma^2}\notag\\
       &\approx QR_0\gamma
\end{align}
Then the antenna pattern is, from (9.23, Jackson),
\begin{equation}
    \frac{dP}{d\Omega}=\frac{c^2Z_0}{32\pi^2}k^4\abs{\vb{p}}^2\sin^2\theta
    =\frac{c^2Z_0}{32\pi^2}k^4Q^2R_0^2\gamma^2\sin^2\theta
\end{equation}

The total power is
\begin{equation}
    P=\frac{c^2Z_0}{32\pi^2}k^4Q^2R_0^2\gamma^2\int\sin^2\theta d\Omega 
    =\frac{c^2Z_0}{12\pi}k^4Q^2R_0^2\gamma^2
\end{equation}

(b) Since $P_2(x)=(1/2)(-1+3x^2)$, the radius is modulated as
\begin{equation}
    R(\theta)=R_0\qty(1-\frac{\gamma}{2}+\frac{3\gamma}{2}\cos^2\theta) 
\end{equation}
So because there is a modulating term independent of the polar angle $\theta$,
the leading mulipole term must be the monopole.
\end{solution}
\end{problem}
%%%%%%%%%%%%%%%%%%%%%%%%%%%%%%%%%%%%%%%%%%%%%%%%%%%%%%%%%%%%%%%%%%%%%%%%%%%%%%%
\end{document}
