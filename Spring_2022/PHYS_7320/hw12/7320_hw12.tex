\documentclass[12pt]{article}

%%%%%%%%%%%%%%%%%%%%%%%%%%%%%%%%%%%%%%%%%%%%%%%%%%%%%%%%%%%%%%%%%%%%%%%%%%%%%%%%
%                           Package preset for homework
%%%%%%%%%%%%%%%%%%%%%%%%%%%%%%%%%%%%%%%%%%%%%%%%%%%%%%%%%%%%%%%%%%%%%%%%%%%%%%%%
% Miscellaneous
\usepackage[margin=1in]{geometry}
\usepackage[utf8]{inputenc}
\usepackage{indentfirst}
\usepackage{blindtext}
\usepackage{graphicx}
\usepackage{xr-hyper}
\usepackage{hyperref}
\usepackage{color}
\usepackage{float}
% Math
\usepackage{latexsym}
\usepackage{amsfonts}
\usepackage{amssymb}
\usepackage{amsmath}
\usepackage{commath}
\usepackage{amsthm}
\usepackage{bbold}
\usepackage{bm}
% Physics
\usepackage{physics}
\usepackage{siunitx}
% Code typesetting
\usepackage{listings}
% Citation
\usepackage[authoryear]{natbib}
\usepackage{appendix}
\usepackage[capitalize]{cleveref}
% Title & name
\title{Homework}
\author{Tien Vo}
\date{\today}


%%%%%%%%%%%%%%%%%%%%%%%%%%%%%%%%%%%%%%%%%%%%%%%%%%%%%%%%%%%%%%%%%%%%%%%%%%%%%%%%
%                   User-defined commands and environments
%%%%%%%%%%%%%%%%%%%%%%%%%%%%%%%%%%%%%%%%%%%%%%%%%%%%%%%%%%%%%%%%%%%%%%%%%%%%%%%%
%%% Misc
\sisetup{load-configurations=abbreviations}
\newcommand{\due}[1]{\date{Due: #1}}
\newcommand{\hint}{\textit{Hint}}
\let\oldt\t
\renewcommand{\t}[1]{\text{#1}}

%%% Bold sets & abbrv
\newcommand{\N}{\mathbb{N}}
\newcommand{\Z}{\mathbb{Z}}
\newcommand{\R}{\mathbb{R}}
\newcommand{\Q}{\mathbb{Q}}
\let\oldP\P
\renewcommand{\P}{\mathbb{P}}
\newcommand{\LL}{\mathcal{L}}
\newcommand{\FF}{\mathcal{F}}
\newcommand{\HH}{\mathcal{H}}
\newcommand{\NN}{\mathcal{N}}
\newcommand{\ZZ}{\mathcal{Z}}
\newcommand{\RN}[1]{\textup{\uppercase\expandafter{\romannumeral#1}}}
\newcommand{\ua}{\uparrow}
\newcommand{\da}{\downarrow}

%%% Unit vectors
\newcommand{\xhat}{\vb{\hat{x}}}
\newcommand{\yhat}{\vb{\hat{y}}}
\newcommand{\zhat}{\vb{\hat{z}}}
\newcommand{\nhat}{\vb{\hat{n}}}
\newcommand{\rhat}{\vb{\hat{r}}}
\newcommand{\phihat}{\bm{\hat{\phi}}}
\newcommand{\thetahat}{\bm{\hat{\theta}}}

%%% Other math stuff
\providecommand{\units}[1]{\,\ensuremath{\mathrm{#1}}\xspace}
% Set new style for problem
\newtheoremstyle{problemstyle}  % <name>
        {10pt}                   % <space above>
        {10pt}                   % <space below>
        {\normalfont}           % <body font>
        {}                      % <indent amount}
        {\bfseries\itshape}     % <theorem head font>
        {\normalfont\bfseries:} % <punctuation after theorem head>
        {.5em}                  % <space after theorem head>
        {}                      % <theorem head spec (can be left empty, 
                                % meaning `normal')>

% Set problem environment
\theoremstyle{problemstyle}
\newtheorem{problemenv}{Problem}[section]
\newenvironment{problem}[1]{%
  \renewcommand\theproblemenv{#1}%
  \problemenv
}{\endproblemenv}
% Set lemma environment
\newenvironment{lemma}[2][Lemma]{\begin{trivlist}
\item[\hskip \labelsep {\bfseries #1}\hskip \labelsep {\bfseries #2.}]}{\end{trivlist}}
% Set solution environment
\newenvironment{solution}{
    \begin{proof}[Solution]$ $\par\nobreak\ignorespaces
}{\end{proof}}
\numberwithin{equation}{problemenv}

%%% Page format
\setlength{\parindent}{0.5cm}
\setlength{\oddsidemargin}{0in}
\setlength{\textwidth}{6.5in}
\setlength{\textheight}{8.8in}
\setlength{\topmargin}{0in}
\setlength{\headheight}{18pt}

%%% Code environments
\definecolor{dkgreen}{rgb}{0,0.6,0}
\definecolor{gray}{rgb}{0.5,0.5,0.5}
\definecolor{mauve}{rgb}{0.58,0,0.82}
\lstset{frame=tb,
  language=Python,
  aboveskip=3mm,
  belowskip=3mm,
  showstringspaces=false,
  columns=flexible,
  basicstyle={\small\ttfamily},
  numbers=none,
  numberstyle=\tiny\color{gray},
  keywordstyle=\color{blue},
  commentstyle=\color{dkgreen},
  stringstyle=\color{mauve},
  breaklines=true,
  breakatwhitespace=true,
  tabsize=4
}
\lstset{
  language=Mathematica,
  numbers=left,
  numberstyle=\tiny\color{gray},
  numbersep=5pt,
  breaklines=true,
  captionpos={t},
  frame={lines},
  rulecolor=\color{black},
  framerule=0.5pt,
  columns=flexible,
  tabsize=2
}


\title{Homework 12: Phys 7320 (Spring 2022)}
\due{April 20, 2022}

\begin{document}
\maketitle
%%%%%%%%%%%%%%%%%%%%%%%%%%%%%%%%%%%%%%%%%%%%%%%%%%%%%%%%%%%%%%%%%%%%%%%%%%%%%%%
\begin{problem}{12.1}[The Lagrangian for a charged particle]
The Lagrangian for a charged particle of mass $m$ and charge $e$ with position
$\vb{r}$ and velocity $\vb{u}\equiv d\vb{r}/dt$ moving in scalar and vector
potentials $\Phi$ and $\vb{A}$ is
\begin{equation}
    \LL=-mc^2\sqrt{1-\frac{u^2}{c^2}}-e\Phi+\frac{e}{c}\vb{u}\vdot\vb{A}.
\end{equation}
(a) Show that the Euler-Lagrange equations for this Lagrangian indeed give rise
to the Lorentz force law. \textit{Hint}: I suggest using index notation, and
remember the potentials $\Phi,\vb{A}$ can depend on both $\vb{r}$ and $t$, which
as far as the particle is concerned means they depend on time in two ways:
$\Phi(\vb{r}(t),t)$ and $\vb{A}(\vb{r}(t),t)$.

(b) Go though the steps (12.13)-(12.17) in Jackson to derive the Hamiltonian
\begin{equation}
    \HH=\sqrt{(c\vb{P}-e\vb{A})^2+m^2c^4}+e\Phi. 
\end{equation}
Along the way derive the canonical/conjugate momentum $\vb{P}$ (which is
different from the familiar ``kinematic momentum'' $\vb{p}=\gamma m\vb{u}$), and
invert it to find an expression for $\vb{u}$ in terms of $\vb{P}$ and $\vb{A}$.

If you want extra practice, you can show that Hamilton's equations for this
Hamiltonian also give rise to the Lorentz force law, with some steps similar to
part (a).
\begin{solution}
\end{solution}
\end{problem}
\newpage
%%%%%%%%%%%%%%%%%%%%%%%%%%%%%%%%%%%%%%%%%%%%%%%%%%%%%%%%%%%%%%%%%%%%%%%%%%%%%%%    
%%%%%%%%%%%%%%%%%%%%%%%%%%%%%%%%%%%%%%%%%%%%%%%%%%%%%%%%%%%%%%%%%%%%%%%%%%%%%%%
\begin{problem}{12.2}[Equivalent Lagrangians]
(a) Use the Principle of Least Action (really the principle of extremal action)
to show that if the Lagrangian $\LL$ is changed by adding the time derivative of
some function of the coordinates and time, then the Euler-Lagrange equations are
unchanged. The new and old Lagrangians are said to be \textit{equivalent}.
Generalize this to a statement about what change to a Lagrangian
\textit{density} $\LL$ leaves the EL equations unchanged.

(b) Show that under a gauge transformation, the Lagrangian for a charged
particle given in the previous problem becomes an equivalent Lagrangian, thus
showing the equations of motion do not change.
\begin{solution}
\end{solution}
\end{problem}
\newpage
%%%%%%%%%%%%%%%%%%%%%%%%%%%%%%%%%%%%%%%%%%%%%%%%%%%%%%%%%%%%%%%%%%%%%%%%%%%%%%%
%%%%%%%%%%%%%%%%%%%%%%%%%%%%%%%%%%%%%%%%%%%%%%%%%%%%%%%%%%%%%%%%%%%%%%%%%%%%%%%
\begin{problem}{12.3}[$SO(2)$ symmetry of two real scalar fields.]
Consider the dynamics of two (real) scalar fields $\phi_1(\vb{x},t)$ and
$\phi_2(\vb{x},t)$ specified by the Lagrangian density
\begin{equation}
    \LL=\frac12\partial_\mu\phi_1\partial^\mu\phi_1+\frac12\partial_\mu\phi_2\partial^\mu\phi_2-V(\phi_1,\phi_2), 
\end{equation}
where the potential $V$ depends only on the combination $\phi_1^2+\phi_2^2$. In
class, we will study this case with the real scalars combined into a single
compex scalar; here we will leave them as two real scalars. Let's make a
definite choice for the potential:
\begin{equation}
    V(\phi_1,\phi_2)=\frac12m^2\phi_1^2+\frac12m^2\phi_2^2+\frac{\lambda}2\qty(\phi_1^2+\phi_2^2)^2. 
\end{equation}

(a) Calculate the equations of motion (Euler-Lagrange equations) for both
$\phi_1$ and $\phi_2$.

(b) Show that the $SO(2)$ transformation
\begin{equation}
    \mqty(\phi_1\\\phi_2)\mapsto\mqty(\phi_1'\\\phi_2')=\mqty(\cos\alpha&\sin\alpha\\-\sin\alpha&\cos\alpha)\mqty(\phi_1\\\phi_2), 
\end{equation}
where $\alpha$ is a constant, is a symmetry of the Lagrangian. Is this a
rotation in physical space? What space does this ``rotation'' act on?

(c) According to Noether's theorem, the existence of this symmetry means there
is a corresponding conserved current $J^\mu$. Find $J^\mu$ in terms of $\phi_1$
and $\phi_2$ (you may drop an overall constant $\alpha$) and show that it is
conserved, $\partial_\mu J^\mu=0$, when you use the equations of motion.
\begin{solution}
\end{solution}
\end{problem}
\newpage
%%%%%%%%%%%%%%%%%%%%%%%%%%%%%%%%%%%%%%%%%%%%%%%%%%%%%%%%%%%%%%%%%%%%%%%%%%%%%%%
\end{document}
