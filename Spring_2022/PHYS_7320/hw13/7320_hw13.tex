\documentclass[12pt]{article}

%%%%%%%%%%%%%%%%%%%%%%%%%%%%%%%%%%%%%%%%%%%%%%%%%%%%%%%%%%%%%%%%%%%%%%%%%%%%%%%%
%                           Package preset for homework
%%%%%%%%%%%%%%%%%%%%%%%%%%%%%%%%%%%%%%%%%%%%%%%%%%%%%%%%%%%%%%%%%%%%%%%%%%%%%%%%
% Miscellaneous
\usepackage[margin=1in]{geometry}
\usepackage[utf8]{inputenc}
\usepackage{indentfirst}
\usepackage{blindtext}
\usepackage{graphicx}
\usepackage{xr-hyper}
\usepackage{hyperref}
\usepackage{enumitem}
\usepackage{color}
\usepackage{float}
% Math
\usepackage{latexsym}
\usepackage{amsfonts}
\usepackage{amssymb}
\usepackage{amsmath}
\usepackage{commath}
\usepackage{amsthm}
\usepackage{bbold}
\usepackage{bm}
% Physics
\usepackage{physics}
\usepackage{siunitx}
% Code typesetting
\usepackage{listings}
% Citation
\usepackage[authoryear]{natbib}
\usepackage{appendix}
\usepackage[capitalize]{cleveref}
% Title & name
\title{Homework}
\author{Tien Vo}
\date{\today}


%%%%%%%%%%%%%%%%%%%%%%%%%%%%%%%%%%%%%%%%%%%%%%%%%%%%%%%%%%%%%%%%%%%%%%%%%%%%%%%%
%                   User-defined commands and environments
%%%%%%%%%%%%%%%%%%%%%%%%%%%%%%%%%%%%%%%%%%%%%%%%%%%%%%%%%%%%%%%%%%%%%%%%%%%%%%%%
%%% Misc
\sisetup{load-configurations=abbreviations}
\newcommand{\due}[1]{\date{Due: #1}}
\newcommand{\hint}{\textit{Hint}}
\let\oldt\t
\renewcommand{\t}[1]{\text{#1}}

%%% Bold sets & abbrv
\newcommand{\N}{\mathbb{N}}
\newcommand{\Z}{\mathbb{Z}}
\newcommand{\R}{\mathbb{R}}
\newcommand{\Q}{\mathbb{Q}}
\let\oldP\P
\renewcommand{\P}{\mathbb{P}}
\newcommand{\LL}{\mathcal{L}}
\newcommand{\FF}{\mathcal{F}}
\newcommand{\HH}{\mathcal{H}}
\newcommand{\NN}{\mathcal{N}}
\newcommand{\ZZ}{\mathcal{Z}}
\newcommand{\RN}[1]{\textup{\uppercase\expandafter{\romannumeral#1}}}
\newcommand{\ua}{\uparrow}
\newcommand{\da}{\downarrow}

%%% Unit vectors
\newcommand{\xhat}{\vb{\hat{x}}}
\newcommand{\yhat}{\vb{\hat{y}}}
\newcommand{\zhat}{\vb{\hat{z}}}
\newcommand{\nhat}{\vb{\hat{n}}}
\newcommand{\rhat}{\vb{\hat{r}}}
\newcommand{\phihat}{\bm{\hat{\phi}}}
\newcommand{\thetahat}{\bm{\hat{\theta}}}

%%% Other math stuff
\providecommand{\units}[1]{\,\ensuremath{\mathrm{#1}}\xspace}
% Set new style for problem
\newtheoremstyle{problemstyle}  % <name>
        {10pt}                   % <space above>
        {10pt}                   % <space below>
        {\normalfont}           % <body font>
        {}                      % <indent amount}
        {\bfseries\itshape}     % <theorem head font>
        {\normalfont\bfseries:} % <punctuation after theorem head>
        {.5em}                  % <space after theorem head>
        {}                      % <theorem head spec (can be left empty, 
                                % meaning `normal')>

% Set problem environment
\theoremstyle{problemstyle}
\newtheorem{problemenv}{Problem}[section]
\newenvironment{problem}[1]{%
  \renewcommand\theproblemenv{#1}%
  \problemenv
}{\endproblemenv}
% Set lemma environment
\newenvironment{lemma}[2][Lemma]{\begin{trivlist}
\item[\hskip \labelsep {\bfseries #1}\hskip \labelsep {\bfseries #2.}]}{\end{trivlist}}
% Set solution environment
\newenvironment{solution}{
    \begin{proof}[Solution]$ $\par\nobreak\ignorespaces
}{\end{proof}}
\numberwithin{equation}{problemenv}

%%% Page format
\setlength{\parindent}{0.5cm}
\setlength{\oddsidemargin}{0in}
\setlength{\textwidth}{6.5in}
\setlength{\textheight}{8.8in}
\setlength{\topmargin}{0in}
\setlength{\headheight}{18pt}

%%% Code environments
\definecolor{dkgreen}{rgb}{0,0.6,0}
\definecolor{gray}{rgb}{0.5,0.5,0.5}
\definecolor{mauve}{rgb}{0.58,0,0.82}
\lstset{frame=tb,
  language=Python,
  aboveskip=3mm,
  belowskip=3mm,
  showstringspaces=false,
  columns=flexible,
  basicstyle={\small\ttfamily},
  numbers=none,
  numberstyle=\tiny\color{gray},
  keywordstyle=\color{blue},
  commentstyle=\color{dkgreen},
  stringstyle=\color{mauve},
  breaklines=true,
  breakatwhitespace=true,
  tabsize=4
}
\lstset{
  language=Mathematica,
  numbers=left,
  numberstyle=\tiny\color{gray},
  numbersep=5pt,
  breaklines=true,
  captionpos={t},
  frame={lines},
  rulecolor=\color{black},
  framerule=0.5pt,
  columns=flexible,
  tabsize=2
}


\title{Homework 13: Phys 7320 (Spring 2022)}
\due{April 27, 2022}

\begin{document}
\maketitle
%%%%%%%%%%%%%%%%%%%%%%%%%%%%%%%%%%%%%%%%%%%%%%%%%%%%%%%%%%%%%%%%%%%%%%%%%%%%%%%
\begin{problem}{13.1}[Noether currents for rotations and boosts]

(0) An infinitesimal rotation or Lorentz boost can be written in the form
\begin{equation}\label{p10:xp}
    x'^\mu=x^\mu+L^\mu_\nu x^\nu, 
\end{equation}
where $L^\mu_\nu$ is constant, small and obeys $L_{\mu\nu}=-L_{\nu\mu}$ (this is
Jackson's (11.89) in tensor notation; $L$ is proportional to either the
rotation angle $\theta$ or the boost parameter $\zeta$). Take a general
Lagrangian density $\LL$ with no particular form depending on some fields
$\phi_i$, and show that the Lagrangian density $\LL(x)$ changes by a total
derivative under these transformations. (Compare to what we did in class to
derive the Noether currents for translations.) Then show the associated Noether
current takes the form (up to a possible constant factor)
\begin{equation}
    J^\mu=L_{\nu\rho}M^{\mu\nu\rho}, 
\end{equation}
with $M^{\mu\nu\rho}$ as in Jackson (12.109),
\begin{equation}
    M^{\mu\nu\rho}=T^{\mu\nu}x^\rho-T^{\mu\rho} x^\nu. 
\end{equation}
We have now motivated the form of Jackson's (12.109) for a general Lagrangian.
Now when you do Jackson's part (a) and (b), use the improved form of the
electromagnetic energy-momentum tensor (12.113) in the expression for
$M^{\mu\nu\rho}$ as in (12.117).

Consider the various conservation laws that are contained in the integral of
$\partial_\alpha M^{\alpha\beta\gamma}=0$ over all space, where
$M^{\alpha\beta\gamma}$ is defined in (12.117).

(a) Show that when $\beta$ and $\gamma$ are both space indices conservation of
the total field angular momentum follows.

(b) Show that when $\beta=0$ the conservation law is
\begin{equation}
    \frac{d\vb{X}}{dt}=\frac{c^2\vb{P}_\t{em}}{E_\t{em}}, 
\end{equation}
where $\vb{X}$ is the coordinate of the center of mass of the electromagnetic
fields, defined by
\begin{equation}
    \vb{X}\int ud^3x=\int \vb{x}ud^3x ,
\end{equation}
where $u$ is the electromagnetic energy density and $E_\t{em}$ and
$\vb{P}_\t{em}$ re the total energy and momentum of the fields.
\begin{solution}
(0) First, the Jacobian of the transformation \eqref{p10:xp} is
\begin{equation}
    \frac{\partial x'^\mu}{\partial x^\nu}
    =\delta^\mu_\nu+L_\nu^\mu.
\end{equation}
Expanding $\LL(x')$ to first order in $\epsilon^\mu=L_\nu^\mu x^\nu$, we get
\begin{equation}
    \LL(x')=\LL(x)+\epsilon^\mu\partial_\mu \LL(x)+\order{\epsilon^2}. 
\end{equation}
Writing $K^\mu=L_\nu^\mu x^\nu\LL$ and using the product rule, we 
get
\begin{equation}
    \partial_\mu K^\mu=L_\nu^\mu \delta^\nu_\mu\LL+L_\nu^\mu
    x^\nu\partial_\mu\LL
    =\delta\LL,
\end{equation}
where $L_\nu^\mu \delta_\mu^\nu=L_\mu^\mu=0$, since $L$ is antisymmetric. So the
Lagrangian density $\LL$ changes by a total derivative of $K$. Also, the scalar
fields $\phi_i(x)$ transform as
\begin{equation}
    \phi_i(x')=\phi_i(x)+\epsilon^\mu\partial_\mu\phi_i(x)+\order{\epsilon^2}.
\end{equation}
So $\delta\phi_i=L_\nu^\mu x^\nu\partial_\mu\phi_i$. The current associated with
these changes is defined as
\begin{align}
    J^\mu
    &=\frac{\partial\LL}{\partial(\partial_\mu\phi_i)}\delta\phi_i-K^\mu\notag\\
    &=\frac{\partial\LL}{\partial(\partial_\mu\phi_i)}L_\rho^\nu
    x^\rho\partial_\nu \phi_i-L_\rho^\mu x^\rho\LL\notag\\
    &=\frac{\partial\LL}{\partial(\partial_\mu\phi_i)}L_{\nu\rho}x^\rho\partial^\nu\phi_i-g^{\mu\nu}L_{\nu\rho}x^\rho\LL\notag\\
    &=L_{\nu\rho}\qty[\frac{\partial\LL}{\partial(\partial_\mu\phi_i)}\partial^\nu\phi_i-g^{\mu\nu}\LL]x^\rho\notag\\
    &=L_{\nu\rho}T^{\mu\nu}x^\rho.
\end{align}
By a change of indices ($\nu\mapsto\rho$ and $\rho\mapsto\nu$), we can also
write
\begin{equation}
    L_{\nu\rho}T^{\mu\nu}x^\rho
    =L_{\rho\nu}T^{\mu\rho}x^\nu
    =-L_{\nu\rho}T^{\mu\rho}x^\nu.
\end{equation}
Thus, the current can be written as
\begin{equation}
    J^{\mu}=\frac12L_{\nu\rho}\qty(T^{\mu\nu}x^\rho-T^{\mu\rho}x^\nu)
    \sim L_{\nu\rho}M^{\mu\nu\rho}, 
\end{equation}
up to a factor of 1/2.

(a) First, by definition, we note that
\begin{align}
    M^{0ij}&=\Theta^{0i}x^j-\Theta^{0j}x^i\notag\\
    &=\frac1{4\pi}\qty[(\vb{E}\times\vb{B})^ix^j-(\vb{E}\times\vb{B})^jx^i]
    \tag{from (12.114, Jackson)}\\
    &=-\frac1{4\pi}\epsilon^{ijk}\qty[\vb{x}\times(\vb{E}\times\vb{B})]^k.
\end{align}
This is just proportional to the field angular momentum density. Thus,
\begin{equation}
    \int d^3xM^{0ij}=-\frac1{4\pi}\epsilon^{ijk}\int d^3x
\qty[\vb{x}\times\qty(\vb{E}\times\vb{B})]^k
=-c\epsilon^{ijk}L_\t{field}^k.
\end{equation}
Then, since $\partial_\mu M^{\mu ij}=0$, we can write
\begin{equation}
    \int d^3x\partial_0 M^{0ij}
    =-c\epsilon^{ijk}\partial_0L_\t{field}^k=\int d^3x\partial_k M^{kij}
    \Rightarrow\epsilon^{ijk}\frac{\partial L_\t{field}^k}{\partial t}
    =-\int d^3x\partial_kM^{kij}.
\end{equation}
This is a sort of continuity equation for the field angular momentum, where the
RHS (by Gauss divergence law) is a surface integral of the Maxwell stress 
tensor, according to (12.115, Jackson). Thus, this conservation law states that
the change of the field angular momentum is balanced by the \textit{flux} of 
electromagnetic force on a given surface.

(b) Setting $\beta=0$, we consider
\begin{equation}
    M^{00i}=\Theta^{00}x^i-\Theta^{0i}x^0
    =\frac{E^2+B^2}{8\pi}x^i-\frac1{4\pi}\qty(\vb{E}\times\vb{B})^ix^0
    =2ux^i-2cg^ix^0.
\end{equation}
Integrating, we get
\begin{equation}
    \int d^3xM^{00i}=2\int d^3xx^iu-2cx^0d^3x g^i
    =2X^iE_\t{em}-2cx^0P_\t{em}^i,
\end{equation}
where $E_\t{em}=\int d^3x u$ is the total field energy and $\vb{P}_\t{em}=\int
d^3x\vb{g}$ is the total field momentum. If $M^{00i}$ is a conserved charge,
then
\begin{equation}
    \frac{dX^i}{dt}E_\t{em}-c^2P_\t{em}^i=0
    \Rightarrow\frac{d\vb{X}}{dt}=\frac{c^2\vb{P}_\t{em}}{E_\t{em}},
\end{equation}
where we have also assumed energy and momentum conservation $dE_\t{em}=0$ and
$d\vb{P}_\t{em}/dt=0$.
\end{solution}
\end{problem}
\newpage
%%%%%%%%%%%%%%%%%%%%%%%%%%%%%%%%%%%%%%%%%%%%%%%%%%%%%%%%%%%%%%%%%%%%%%%%%%%%%%%    
%%%%%%%%%%%%%%%%%%%%%%%%%%%%%%%%%%%%%%%%%%%%%%%%%%%%%%%%%%%%%%%%%%%%%%%%%%%%%%%
\begin{problem}{13.2}[Proca equation.]
The Proca equation for a ``massive'' vector field $A_\mu(x)$ coupled to a
current $J^\nu$ is
\begin{equation}
    \partial_\mu F^{\mu\nu}+\mu^2A^\nu=\frac{4\pi}{c}J^\nu. 
\end{equation}
Due to the ``mass'' term $\mu^2A^\nu$, this equation has no gauge invariance.
The Proca equation is obeyed by massive spin-1 fields like the fields of the
W-boson and Z-boson, and also describes the photon when it is superconducting.

(a) Assuming the parameter $\mu\neq0$ and the current is conserved, show that
the Proca equation implies the vanishing 4-divergence
\begin{equation}
    \partial_\mu A^\mu=0. 
\end{equation}
Note this is not a gauge condition (there is no gauge invariance for massive
$A_\mu$) but a consequence of the Proca equation. Then show the Proca equation
becomes
\begin{equation}
    (\partial_\mu\partial^\mu+\mu^2)A^\nu=\frac{4\pi}{c}J^\nu, 
\end{equation}
which is a Klein-Gordon equation for each component $A^\nu$ with mass parameter
$\mu$, sourced by the current $J^\nu$.

(b) Consider Proca waves in a region with no current, $J^\nu=0$. Use a plane
wave ansatz of the form
\begin{equation}
    A_\mu=\epsilon_\mu e^{-ik\vdot x}, 
\end{equation}
where $k\vdot x$ is a 4-vector dot product involving a 4-wavevector
$k^\mu=(\omega/c,\vb{k})$, and $\epsilon_\mu$ is a constant polarization vector.
Using the results of part (a), first find the relationship between $\omega$ and
$\vb{k}$, showing that $k^\mu$ is a timelike wavevector, not null, and so the
waves travel at less than the speed of light.

Then find a constraint relating $\epsilon_\mu$ and $k^\mu$. Since there is no
gauge invariance, this is the only constraint obeyed by $\epsilon_\mu$. Since
$k^\mu$ is timelike, you may choose a frame where $k^\mu=(\omega/c,\vb{0})$;
what directions may $\epsilon_\mu$ point in this frame? We see there are three
possible polarizations for the (massive) Proca wave, unlike the ordinary
(massless) Maxwell electromagnetic wave which has two.

(c) Consider a time-independent delta-function charge at the origin
$\rho(x)=q\delta^3(\vb{x})$, and show the Proca equation is solved by the Yukawa
potential,
\begin{equation}
    \Phi(x)=q\frac{e^{-\mu r}}{r}, 
\end{equation}
with $\vb{A}=\vb{0}$. Thus the force fied for a massive vector field is
exponentially suppressed, explaining why we don't see long-range forces from the
W-boson and Z-boson.

\textit{Hint}: Recall that the 3D Laplacian acting on $1/r$ gives a delta
function, as is needed for the case of the usual Coulomb potential (which is the
$\mu\to0$ limit of the Yukawa potential). The other pieces can be treated by
looking at the Laplacian in spherical coordinates.
\begin{solution}
(a) By definition, $F^{\mu\nu}=\partial^\mu A^\nu-\partial^\nu A^\mu$, so
\begin{align}
    \partial_{\mu\nu} F^{\mu\nu} 
    =\partial_{\mu\nu}\partial^\mu A^\nu-\partial_{\mu\nu}\partial^\nu A^\mu
    =\partial_\mu\partial^\mu(\partial_\nu A^\nu)
    -\partial_\nu\partial^\nu(\partial_\mu A^\mu)
    =\square(\partial_\nu A^\nu-\partial_\mu A^\mu)
    =0,
\end{align}
where $\square=\partial_\mu\partial^\mu$ is the 4-Laplacian. Thus, the
derivative of the Proca equation becomes
\begin{equation}
    \partial_{\mu\nu}F^{\mu\nu}+\mu^2\partial_\nu A^\nu
    =\mu^2\partial_\nu A^\nu=\frac{4\pi}{c}\partial_\nu J^\nu=0,
\end{equation}
since the current $J^\mu$ is conserved. This implies that $\partial_\nu A^\nu=0$
if $\mu\neq0$. Then, expanding the Proca equation, we can write
\begin{equation}
    \partial_\mu F^{\mu\nu}+\mu^2A^\nu=\partial_\mu\partial^\mu A^\nu
    -\partial^\nu\partial_\mu A^\mu+\mu^2A^\nu
    =\qty(\partial_\mu\partial^\mu+\mu^2)A^\nu=\frac{4\pi}{c}J^\nu.
\end{equation}

(b) From the previous result, assuming $J^\mu=0$,
\begin{equation}
    \qty(\partial_\mu\partial^\mu+\mu^2)A^\nu
    =\epsilon^\nu\qty(-k_\mu k^\mu+\mu^2)e^{-ik\vdot x}=0.
\end{equation}
Since $\epsilon^\nu\neq0$, it follows that
\begin{equation}
    k_\mu k^\mu=\qty(\frac{\omega}{c})^2-k^2=\mu^2>0. 
\end{equation}
So $k^\mu$ is timelike. Also, the dispersion relation is
\begin{equation}
    \omega^2=c^2k^2+c^2\mu^2. 
\end{equation}
The phase velocity is
\begin{equation}
    \frac{\omega}{k}=c\sqrt{1+\frac{\mu^2}{k^2}}, 
\end{equation}
and the group velocity is
\begin{equation}
    \frac{d\omega}{dk}=\frac{c^2}{\omega/k} 
    =\frac{c}{\sqrt{1+\mu^2/k^2}}.
\end{equation}
Since the denominator is always larger than unity because $\mu>0$, $d\omega/dk$
is less than the speed of light. Also, since $\partial_\mu A^\mu=0$, it follows
that
\begin{equation}
    \partial_\mu A^\mu=-ik_\mu\epsilon^\mu e^{-ik_\nu x^\nu}=0
    \Rightarrow k_\mu \epsilon^\mu=0.
\end{equation}

Without loss of generality, assume $\vb{k}=k\xhat$ and let the Lorentz boost
be along $\xhat$
\begin{equation}
    \Lambda=\mqty(\frac{\omega^2}{c^2\mu^2}&-\frac{k\omega}{c\mu^2}&0&0\\
    -\frac{k\omega}{c\mu^2}&\frac{\omega^2}{c^2\mu^2}&0&0\\
    0&0&1&0\\
    0&0&0&1).
\end{equation}
Then it follows that
\begin{equation}
    k'^\mu
    =\Lambda_\nu^\mu k^\nu=\mqty(\omega/c\\\vb{0}).
\end{equation}
From the above constraint that $k_\mu\epsilon^\mu=0$. The transformed
$\epsilon'^\mu$ must also be perpendicular with $k'^\mu$. So it can point in any
of the three space-like directions, resulting in 3 possible polarizations.

(c) We expand the LHS of the Proca equation for $A^0=\Phi$, since 
$A^i=0$.
\begin{align}
    \t{LHS}&=(-\grad^2+\mu^2)\Phi \notag\\
           &=\mu^2\Phi-q\qty[\frac1r\grad^2\qty(e^{-\mu r})+e^{-\mu
           r}\grad^2\qty(\frac1r)+2\grad\qty(e^{-\mu r})\vdot\grad\qty(\frac1r)]
           \notag\\
           &=\mu^2\Phi-q\qty[\frac{\mu^2}{r}e^{-\mu r}-\frac{2\mu}{r}e^{-\mu
           r}-4\pi\delta(\vb{x})e^{-\mu r}+2\frac{\mu}{r^2}e^{-\mu r}]\notag\\
           &=4\pi q\delta^3(\vb{x})\notag\\
           &=\t{RHS},
\end{align}
where we have used the familiar results
\begin{equation}
    \grad^2\qty(\frac1r)=-4\pi\delta^3(\vb{x}),\qquad\t{and}\qquad
    \grad\qty(\frac1r)=-\frac{\rhat}{r^2},
\end{equation}
and also, we have recognized that $J^0=c\rho(\vb{x})=cq\delta^3(\vb{x})$. Thus,
the Yukawa potential solves the Proca equation.
\end{solution}
\end{problem}
\newpage
%%%%%%%%%%%%%%%%%%%%%%%%%%%%%%%%%%%%%%%%%%%%%%%%%%%%%%%%%%%%%%%%%%%%%%%%%%%%%%%
\end{document}

