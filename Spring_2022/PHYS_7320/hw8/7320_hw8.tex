\documentclass[12pt]{article}

%%%%%%%%%%%%%%%%%%%%%%%%%%%%%%%%%%%%%%%%%%%%%%%%%%%%%%%%%%%%%%%%%%%%%%%%%%%%%%%%
%                           Package preset for homework
%%%%%%%%%%%%%%%%%%%%%%%%%%%%%%%%%%%%%%%%%%%%%%%%%%%%%%%%%%%%%%%%%%%%%%%%%%%%%%%%
% Miscellaneous
\usepackage[margin=1in]{geometry}
\usepackage[utf8]{inputenc}
\usepackage{indentfirst}
\usepackage{blindtext}
\usepackage{graphicx}
\usepackage{xr-hyper}
\usepackage{hyperref}
\usepackage{enumitem}
\usepackage{color}
\usepackage{float}
% Math
\usepackage{latexsym}
\usepackage{amsfonts}
\usepackage{amssymb}
\usepackage{amsmath}
\usepackage{commath}
\usepackage{amsthm}
\usepackage{bbold}
\usepackage{bm}
% Physics
\usepackage{physics}
\usepackage{siunitx}
% Code typesetting
\usepackage{listings}
% Citation
\usepackage[authoryear]{natbib}
\usepackage{appendix}
\usepackage[capitalize]{cleveref}
% Title & name
\title{Homework}
\author{Tien Vo}
\date{\today}


%%%%%%%%%%%%%%%%%%%%%%%%%%%%%%%%%%%%%%%%%%%%%%%%%%%%%%%%%%%%%%%%%%%%%%%%%%%%%%%%
%                   User-defined commands and environments
%%%%%%%%%%%%%%%%%%%%%%%%%%%%%%%%%%%%%%%%%%%%%%%%%%%%%%%%%%%%%%%%%%%%%%%%%%%%%%%%
%%% Misc
\sisetup{load-configurations=abbreviations}
\newcommand{\due}[1]{\date{Due: #1}}
\newcommand{\hint}{\textit{Hint}}
\let\oldt\t
\renewcommand{\t}[1]{\text{#1}}

%%% Bold sets & abbrv
\newcommand{\N}{\mathbb{N}}
\newcommand{\Z}{\mathbb{Z}}
\newcommand{\R}{\mathbb{R}}
\newcommand{\Q}{\mathbb{Q}}
\let\oldP\P
\renewcommand{\P}{\mathbb{P}}
\newcommand{\LL}{\mathcal{L}}
\newcommand{\FF}{\mathcal{F}}
\newcommand{\HH}{\mathcal{H}}
\newcommand{\NN}{\mathcal{N}}
\newcommand{\ZZ}{\mathcal{Z}}
\newcommand{\RN}[1]{\textup{\uppercase\expandafter{\romannumeral#1}}}
\newcommand{\ua}{\uparrow}
\newcommand{\da}{\downarrow}

%%% Unit vectors
\newcommand{\xhat}{\vb{\hat{x}}}
\newcommand{\yhat}{\vb{\hat{y}}}
\newcommand{\zhat}{\vb{\hat{z}}}
\newcommand{\nhat}{\vb{\hat{n}}}
\newcommand{\rhat}{\vb{\hat{r}}}
\newcommand{\phihat}{\bm{\hat{\phi}}}
\newcommand{\thetahat}{\bm{\hat{\theta}}}

%%% Other math stuff
\providecommand{\units}[1]{\,\ensuremath{\mathrm{#1}}\xspace}
% Set new style for problem
\newtheoremstyle{problemstyle}  % <name>
        {10pt}                   % <space above>
        {10pt}                   % <space below>
        {\normalfont}           % <body font>
        {}                      % <indent amount}
        {\bfseries\itshape}     % <theorem head font>
        {\normalfont\bfseries:} % <punctuation after theorem head>
        {.5em}                  % <space after theorem head>
        {}                      % <theorem head spec (can be left empty, 
                                % meaning `normal')>

% Set problem environment
\theoremstyle{problemstyle}
\newtheorem{problemenv}{Problem}[section]
\newenvironment{problem}[1]{%
  \renewcommand\theproblemenv{#1}%
  \problemenv
}{\endproblemenv}
% Set lemma environment
\newenvironment{lemma}[2][Lemma]{\begin{trivlist}
\item[\hskip \labelsep {\bfseries #1}\hskip \labelsep {\bfseries #2.}]}{\end{trivlist}}
% Set solution environment
\newenvironment{solution}{
    \begin{proof}[Solution]$ $\par\nobreak\ignorespaces
}{\end{proof}}
\numberwithin{equation}{problemenv}

%%% Page format
\setlength{\parindent}{0.5cm}
\setlength{\oddsidemargin}{0in}
\setlength{\textwidth}{6.5in}
\setlength{\textheight}{8.8in}
\setlength{\topmargin}{0in}
\setlength{\headheight}{18pt}

%%% Code environments
\definecolor{dkgreen}{rgb}{0,0.6,0}
\definecolor{gray}{rgb}{0.5,0.5,0.5}
\definecolor{mauve}{rgb}{0.58,0,0.82}
\lstset{frame=tb,
  language=Python,
  aboveskip=3mm,
  belowskip=3mm,
  showstringspaces=false,
  columns=flexible,
  basicstyle={\small\ttfamily},
  numbers=none,
  numberstyle=\tiny\color{gray},
  keywordstyle=\color{blue},
  commentstyle=\color{dkgreen},
  stringstyle=\color{mauve},
  breaklines=true,
  breakatwhitespace=true,
  tabsize=4
}
\lstset{
  language=Mathematica,
  numbers=left,
  numberstyle=\tiny\color{gray},
  numbersep=5pt,
  breaklines=true,
  captionpos={t},
  frame={lines},
  rulecolor=\color{black},
  framerule=0.5pt,
  columns=flexible,
  tabsize=2
}


\title{Homework 8: Phys 7320 (Spring 2022)}
\due{March 16, 2022}

\begin{document}
\maketitle
%%%%%%%%%%%%%%%%%%%%%%%%%%%%%%%%%%%%%%%%%%%%%%%%%%%%%%%%%%%%%%%%%%%%%%%%%%%%%%%
\begin{problem}{8.1}[Source-free Maxwell equations]
Maxwell's source-free equations take the form
\begin{equation}\label{p1:source_free_maxwell}
    \partial_\mu F_{\nu\rho}+\partial_\nu F_{\rho\mu}
    +\partial_\rho F_{\mu\nu}=0.
\end{equation}

(a) Show that the LHS of equation \eqref{p1:source_free_maxwell} is
\textit{totally} antisymmetric, meaning it flips sign under the exchange of any
pair of indices (note you have to do the exchange simultaneously in all terms).
To do this you'll need the antisymmetry of the Faraday tensor
$F_{\mu\nu}=-F_{\nu\mu}$. Thus, the LHS vanishes automaticaly unless all three
indices take different values.

(b) Show that the case where the three indices are the three different spatial
directions leads to the Maxwell equation $\div{\vb{B}}=0$.

(c) Show that the case where one index is in the time direction and the other
two indices are in two different spatial directions (say $x$ and $y$) implies
the component of Faraday's law in the third spatial direction.

(d) Show that if the Faraday tensor is defined in terms of a 4-vector potential
$F_{\mu\nu}=\partial_\mu A_\nu-\partial_\nu A_\mu$, then
\eqref{p1:source_free_maxwell} is automatically satisfied.
\begin{solution}
(a) First, by the $\rho\longleftrightarrow\nu$ exchange, the LHS becomes
\begin{equation}
    \partial_\mu F_{\rho\nu}
    +\partial_\rho F_{\nu\mu}
    +\partial_\nu F_{\mu\rho}
    =-\partial_\mu F_{\nu\rho}
    -\partial_\rho F_{\mu\nu}
    -\partial_\nu F_{\rho\mu}
    =-\t{LHS}.
\end{equation}
Second, by the $\mu\longleftrightarrow\nu$ exchange, it becomes
\begin{equation}
    \partial_\nu F_{\mu\rho}
    +\partial_\mu F_{\rho\nu}
    +\partial_\rho F_{\nu\mu}
    =-\partial_\nu F_{\rho\mu}
    -\partial_\mu F_{\nu\rho}
    -\partial_\rho F_{\mu\nu}
    =-\t{LHS}.
\end{equation}
Thus, due to commutativity, the LHS flips sign under exchange of any pair of
indices.

(b) Given from (11.138, Jackson) that
\begin{equation}
    F_{\alpha\beta}=\mqty(
        0&E_x&E_y&E_z\\-E_x&0&-B_z&B_y\\-E_y&B_z&0&-B_x\\
        -E_z&-B_y&B_x&0),
\end{equation}
if $\mu,\nu,\rho\in\qty{1,2,3}$, \eqref{p1:source_free_maxwell} becomes
\begin{align}
    \partial_1F_{23}+\partial_2F_{31}+\partial_3F_{12}
    =\partial_x(-B_x)+\partial_2(-B_y)+\partial_3(-B_z)
    =-\div{\vb{B}}=0.
\end{align}
Thus, we retrieve one of the Maxwell equations.

(c) Now, let $\mu=0,\nu=1,\rho=2$, then \eqref{p1:source_free_maxwell} becomes
\begin{align}
    \partial_0 F_{12}
    +\partial_1 F_{20}
    +\partial_2F_{01}
    =\partial_t(-B_z)
    +\partial_x(-E_y)
    +\partial_y(E_x)
    =0
    \Rightarrow\qty(\curl{\vb{E}})_z=-\qty(\partial\vb{B}/\partial t)_z.
\end{align}
This is the $z$ component of Faraday's law.

(d) Assuming the potential $A_\rho$ is smooth, then
$\partial_{\mu\nu}A_\rho=\partial_{\nu\mu}A_\rho$. It then follows that
\begin{align}
    \partial_\mu F_{\nu\rho}+\partial_\nu F_{\rho\mu}+\partial_\rho F_{\mu\nu}
    &=\partial_{\mu\nu}A_\rho-\partial_{\mu\rho}A_\nu
    +\partial_{\nu\rho}A_\mu-\partial_{\nu\mu}A_\rho
    +\partial_{\rho\mu}A_\nu-\partial_{\rho\nu}A_\mu\notag\\
    &=0.
\end{align}
\end{solution}
\end{problem}
\newpage
%%%%%%%%%%%%%%%%%%%%%%%%%%%%%%%%%%%%%%%%%%%%%%%%%%%%%%%%%%%%%%%%%%%%%%%%%%%%%%%    
%%%%%%%%%%%%%%%%%%%%%%%%%%%%%%%%%%%%%%%%%%%%%%%%%%%%%%%%%%%%%%%%%%%%%%%%%%%%%%%
\begin{problem}{8.2}[Transformation of the electromagnetic field]

(a) Write down the Lorentz transformation matrix $\Lambda$ for a Lorentz boost
in the $x$ direction with velocity $v$, and act this appropriately on the matrix
$F$ for the Faraday tensor to derive the transformation rules for the components
of $\vb{E}$ and $\vb{B}$ as given in class or (11.148, Jackson). Then repeat
this, but for a rotation around the $z$-axis with angle $\theta$, and thus show
that both $\vb{E}$ and $\vb{B}$ transform as you would expect separate 3-vectors
to transform.

(b) Evaluate the Lorentz scalars $F_{\mu\nu}F^{\mu\nu}$ and
$F_{\mu\nu}\FF^{\mu\nu}$ (where $\FF^{\mu\nu}$) is the dual Faraday tensor given
in (11.140, Jackson)) in terms of $\vb{E}$ and $\vb{B}$, and thus show that
$I_1\equiv B^2-E^2$ and $I_2\equiv \vb{E}\vdot\vb{B}$ are Lorentz invariants.
Argue that $\FF_{\mu\nu}\FF^{\mu\nu}$ does not give you anything new (you can
calculate it explicitly, but some thought about the properties of $\FF^{\mu\nu}$
will also reveal the answer).

(c) Verify by explicit calculation that $I_1$ and $I_2$ are indeed invariant
under a Lorentz boost in the $x$-direction by velocity $v$. Show that $E^2+B^2$
is not invariant under such a transformation. (It is invariant under rotations,
but not boosts).

(d) Is it possible to have an electromagnetic field that appears as a purely
electric field in one inertial frame, and a purely magnetic field in some other
inertial frame? Explain. If there is some inertial frame where the electric
field is zero, what conditions must hold on $\vb{E}$ and $\vb{B}$ in any frame?
\begin{solution}
(a) The Lorentz transformation is
\begin{equation}
    \Lambda=\mqty(\gamma&-\gamma\beta&0&0\\-\gamma\beta&\gamma&0&0\\
            0&0&1&0\\0&0&0&1).
\end{equation}
Thus, by matrix multiplication,
\begin{align}
    F'
    &=\Lambda F\Lambda^T\notag\\
    &=\Lambda\mqty(
        \gamma\beta E_x&-\gamma E_x&-E_y&E_z\\
        \gamma E_x&-\gamma\beta E_x&-B_z&B_y\\
        \gamma\qty(E_y-\beta B_z)&\gamma\qty(B_z-\gamma E_y)&0&-B_x\\
        \gamma\qty(E_z+\beta B_y)&-\gamma\qty(B_y+\beta E_z)&B_x&0
    )\notag\\
    &=\mqty(0&-E_x&-\gamma\qty(E_y-\beta B_z)&-\gamma\qty(E_z+\beta B_y)\\
            E_x&0&-\gamma\qty(B_z-\beta E_y)&\gamma\qty(B_y+\beta E_z)\\
            \gamma\qty(E_y-\beta B_z)&\gamma\qty(B_z-\beta E_y)&0&-B_x\\
            \gamma\qty(E_z+\beta B_y)&-\gamma\qty(B_y+\beta E_z)&B_x&0).
\end{align}
Thus, reading off of this, the fields transform as
\begin{subequations}\label{p2a:EBp}
    \begin{align}
        E_x'&=E_x & B_x'&=B_x\\
        E_y'&=\gamma\qty(E_y-\beta B_z)&B_y'&=\gamma\qty(B_y+\beta E_z)\\
        E_z'&=\gamma\qty(E_z+\beta B_y)&B_z'&=\gamma\qty(B_z-\beta E_y)
    \end{align} 
\end{subequations}
This is the same as (11.148, Jackson). Similarly, a rotation about $z$ is
\begin{equation}
    R_z=\mqty(1&0&0&0\\0&\cos\theta&\sin\theta&0\\0&-\sin\theta&\cos\theta&0\\
    0&0&0&1). 
\end{equation}
By matrix multiplication,
\begin{align}
    F'&=R_zFR_z^T \notag\\
      &=R_z\mqty(
    0&-E_x\cos\theta-E_y\sin\theta&-E_y\cos\theta+E_x\sin\theta&-E_z\\
    E_x&-B_z\sin\theta&0&B_y\\
    E_y&B_z\cos\theta&-B_z\sin\theta&-B_x\\
    E_z&-B_y\cos\theta+B_x\sin\theta&B_x\cos\theta+B_y\sin\theta&0
        )\notag\\
       &=\mqty(
    0&-\qty(E_x\cos\theta+E_y\sin\theta)&-\qty(E_y\cos\theta-E_x\sin\theta)
     &-E_z\\
    E_x\cos\theta+E_y\sin\theta&0&-B_z&B_y\cos\theta-B_x\sin\theta\\
    E_y\cos\theta-E_x\sin\theta&B_z&0&-\qty(B_x\cos\theta+B_y\sin\theta)\\
    E_z&-\qty(B_y\cos\theta-B_x\sin\theta)&B_x\cos\theta+B_y\sin\theta&0
       ).
\end{align}
Thus,
\begin{subequations}
    \begin{align}
        E_x'&=E_x\cos\theta+E_y\sin\theta &
        B_x'&=B_x\cos\theta+B_y\sin\theta\\
        E_y'&=E_y\cos\theta-E_x\sin\theta &
        B_y'&=B_y\cos\theta-B_x\sin\theta\\
        E_z'&=E_z & B_z'&=B_z
    \end{align} 
\end{subequations}
$\vb{E}$ and $\vb{B}$ in the $z$-direction remain the same, while the components
in the perpendicular component ($xy$) are rotated counterclockwise by $\theta$,
as expected.

(b) Because $F_{\mu\nu}$ is antisymmetric, $F_{\mu\mu}=0$, while
$F_{\mu\nu}=-F_{\nu\mu}$ for $\nu\neq\mu$. So,
\begin{align}
    F_{\mu\nu}F^{\mu\nu}
    &=2\qty(F_{01}F^{01}+F_{02}F^{02}+F_{03}F^{03}+F_{12}F^{12}
    +F_{13}F^{13}+F_{23}F^{23})\notag\\
    &=2\qty(-E_x^2-E_y^2-E_z^2+B_z^2+B_y^2+B_x^2)\notag\\
    &=2\qty(B^2-E^2).
\end{align}
Thus, $B^2-E^2$ is a Lorentz invariant since $F_{\mu\nu}F^{\mu\nu}$ is a Lorentz
scalar. Similarly, $\FF_{\mu\nu}$ is also antisymmetric. So,
\begin{align}
    F_{\mu\nu}\FF^{\mu\nu}
    &=2\qty(F_{01}\FF^{01}+F_{02}\FF^{02}+F_{03}\FF^{03} +F_{12}\FF^{12}
        +F_{13}\FF^{13}+F_{23}\FF^{23})\notag\\
    &=2\qty(-E_xB_x-E_yB_y-E_zB_z-B_zE_z-B_yE_y-B_xE_x)\notag\\
    &=-4\vb{E}\vdot\vb{B}.
\end{align}
Thus, $\vb{E}\vdot\vb{B}$ is Lorentz invariant for the same reason. By
construction, $\FF^{\alpha\beta}$ differs from $F^{\alpha\beta}$ by an exchange
$\vb{B}\to\vb{E}$ and $\vb{E}\to\vb{B}$. Thus,
$\FF_{\mu\nu}\FF^{\mu\nu}=2(E^2-B^2)$, which doesn't reveal any additional
information.

(c) From \eqref{p2a:EBp}, we can calculate
\begin{align}
    B'^2-E'^2
    &=B_x^2+\gamma^2\qty(B_y^2+B_z^2+\beta^2(E_y^2+E_z^2)+2\beta\qty(E_zB_y-E_yB_z))\notag\\
    &\qquad-E_x^2
    -\gamma^2\qty(E_y^2+E_z^2+\beta^2(B_y^2+B_z^2)+2\beta\qty(E_zB_y-E_yB_z))
    \notag\\
    &=B_x^2-E_x^2+\gamma^2\qty(\frac{B_y^2+B_z^2}{\gamma^2}-\frac{E_y^2+E_z^2}{\gamma^2})\notag\\
    &=B^2-E^2.
\end{align}
So $B^2-E^2$ is invariant. In the first equality, we can see clearly
how $E^2+B^2$ wouldn't be invariant, since if there is a positive
instead of a negative sign in the last term, they would factor as
$1+\beta^2$ instead of $1-\beta^2$, and would not cancel the $\gamma^2$
in the front. Now, similarly,
\begin{align}
    \vb{E}'\vdot\vb{B}' 
    &=E_xB_x+\gamma^2\qty(E_yB_y+\beta E_yE_z-\beta B_yB_z-\beta^2E_zB_z)
    \notag\\
    &\qquad+\gamma^2\qty(E_zB_z-\beta E_yE_z+\beta B_yB_z -\beta^2 E_yB_y)
    \notag\\
    &=E_xB_x+\gamma^2\qty(\frac{E_yB_y}{\gamma^2}+\frac{E_zB_z}{\gamma^2})\notag\\
    &=\vb{E}\vdot\vb{B}.
\end{align}
So it is also invariant.

(d) Suppose that there exists a velocity $\bm\beta$ such that the fields
transform as in (11.149, Jackson). Now, suppose the un-primed frame has zero
magnetic field and the primed frame has zero electric field. Then we can solve
\begin{equation}
    \vb{E}'=\gamma\vb{E}-\frac{\gamma^2}{\gamma+1}\bm\beta(\bm\beta\vdot\vb{E})
    =\bm0\Rightarrow\vb{E}=\frac{\gamma}{\gamma+1}(\bm\beta\vdot\vb{E})\bm\beta.
\end{equation}
And the magnetic field in the primed frame must be
\begin{equation}
    \vb{B}' =\gamma\vb{E}\times\bm\beta.
\end{equation}
However, since $\vb{E}$ is parallel to $\bm\beta$, $\vb{E}\times\bm\beta=\bm0$,
which is a contradiction. So there does not exist such a $\bm\beta$. If the
electric field is zero in the un-primed frame, then the following must be true.
\begin{subequations}
    \begin{align}
        \vb{E}'&=\gamma\bm\beta\times\vb{B}\label{p2d:1}\\
        \vb{B}'&=\gamma\vb{B}-\frac{\gamma^2}{\gamma+1}\bm\beta(\bm\beta\vdot\vb{B})\label{p2d:2}
    \end{align} 
\end{subequations}
Substituing \eqref{p2d:2} into \eqref{p2d:1}, we get
\begin{equation}
    \vb{E}'=\bm\beta\times\vb{B}'.
\end{equation}
So the electric field and magnetic field in any reference frame must follow this
relationship, which depends on the relative velocity of that frame with the
un-primed frame.
\end{solution}
\end{problem}
\newpage
%%%%%%%%%%%%%%%%%%%%%%%%%%%%%%%%%%%%%%%%%%%%%%%%%%%%%%%%%%%%%%%%%%%%%%%%%%%%%%%
%%%%%%%%%%%%%%%%%%%%%%%%%%%%%%%%%%%%%%%%%%%%%%%%%%%%%%%%%%%%%%%%%%%%%%%%%%%%%%%
\begin{problem}{8.3}[A charged wire]
An infinitely long straight wire of negligible cross-sectional area is at rest
and has a uniform linear charge density $q_0$ in the inertial frame $K'$. The
frame $K'$ (and the wire) move with a velocity $\vb{v}$ parallel to the
direction of the wire with respect to the laboratory frame $K$.

(a) Write down the electric and magnetic fields in cylindrical coordinates in
the rest frame of the wire. Using the Lorentz transformation properties of the
fields, find the components of the electric and magnetic fields in the
laboratory.

(b) What are the charge and current densities associated with the wire in its
rest frame? In the laboratory?

(c) From the laboratory charge and current densities, calculate directly the
electric and magnetic fields in the laboratory. Compare with the results of part
(a).
\begin{solution}
(a) First, \textit{let us call $q_0'$ the charge density in the inertial frame
$K'$ for consistency}, since $K$ is the lab frame. Then, draw a cylindrical 
Gaussian surface with length $L'$ along the wire in $K'$. By Gauss law, 
$E'A=E'2\pi rL'=4\pi Q=4\pi q_0'L'$. Thus, by symmetry,
\begin{equation}\label{p3a:E}
    \vb{E}'=\frac{2q_0'}{r}\rhat.
\end{equation}
The system is electrostatic in $K'$. Thus, $\vb{B}'=\bm0$. Then, since $K$ moves
with a relative velocity $\bm\beta=-(v/c)\zhat$ with respect to $K'$, the
electric field transform as
\begin{equation}
    \vb{E}=\gamma\vb{E}'-\frac{\gamma^2}{\gamma+1}\bm\beta(\bm\beta\vdot\vb{E}') 
    =\gamma\vb{E}'=\gamma\frac{2q_0'}{r}\rhat,
\end{equation}
since $\bm\beta\perp\vb{E}'$. Note that $\rhat$ is perpendicular to
the Lorentz boost (in the $z$ direction), so $r$ is not contracted. Similarly,
\begin{equation}
    \vb{B}=\gamma\vb{E}' \times\bm\beta
    =-\gamma\frac{2q_0'}{r}\beta\rhat\times\zhat
    =\gamma\beta\frac{2q_0'}{r}\phihat.
\end{equation}
$\phihat$ is also perpendicular to $\zhat$, so there is no relativistic effect
in the $\phi$ direction.

(b) In $K'$, the charge density is given as $q_0'$ and there is no current
density since it is electrostatic. In $K$, the length is contracted
($L=L'/\gamma$). Thus, the charge density is $q_0=Q/L=\gamma Q/L'=\gamma q_0'$.
The current density is, by definition, $\vb{J}=nQv\zhat$ where $n$ is the
number density. Then the current is $I=\int\vb{J}\vdot d\vb{a}=q_0v$.

(c) Similar to \eqref{p3a:E}, the electric field in the lab frame is
\begin{equation}
    \vb{E}=\frac{2q_0}{r}\rhat=\gamma\frac{2q_0}{r}\rhat=\gamma\vb{E}'.
\end{equation}
Now, draw a circular Amperian loop with the wire perpendicular and is centered 
to its plane. Then by Ampere law,
\begin{equation}
    B2\pi r=\frac{4\pi I}{c}\Rightarrow
    \vb{B}=\frac{2I}{cr}\phihat=\frac{2q_0v}{cr}\phihat
    =\gamma\beta\frac{2q_0'}{r}\phihat.
\end{equation}
These results are the same as those in part (a).
\end{solution}
\end{problem}
\newpage
%%%%%%%%%%%%%%%%%%%%%%%%%%%%%%%%%%%%%%%%%%%%%%%%%%%%%%%%%%%%%%%%%%%%%%%%%%%%%%%
\end{document}
