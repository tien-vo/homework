\documentclass[12pt]{article}

%%%%%%%%%%%%%%%%%%%%%%%%%%%%%%%%%%%%%%%%%%%%%%%%%%%%%%%%%%%%%%%%%%%%%%%%%%%%%%%%
%                           Package preset for homework
%%%%%%%%%%%%%%%%%%%%%%%%%%%%%%%%%%%%%%%%%%%%%%%%%%%%%%%%%%%%%%%%%%%%%%%%%%%%%%%%
% Miscellaneous
\usepackage[margin=1in]{geometry}
\usepackage[utf8]{inputenc}
\usepackage{indentfirst}
\usepackage{blindtext}
\usepackage{graphicx}
\usepackage{xr-hyper}
\usepackage{hyperref}
\usepackage{enumitem}
\usepackage{color}
\usepackage{float}
% Math
\usepackage{latexsym}
\usepackage{amsfonts}
\usepackage{amssymb}
\usepackage{amsmath}
\usepackage{commath}
\usepackage{amsthm}
\usepackage{bbold}
\usepackage{bm}
% Physics
\usepackage{physics}
\usepackage{siunitx}
% Code typesetting
\usepackage{listings}
% Citation
\usepackage[authoryear]{natbib}
\usepackage{appendix}
\usepackage[capitalize]{cleveref}
% Title & name
\title{Homework}
\author{Tien Vo}
\date{\today}


%%%%%%%%%%%%%%%%%%%%%%%%%%%%%%%%%%%%%%%%%%%%%%%%%%%%%%%%%%%%%%%%%%%%%%%%%%%%%%%%
%                   User-defined commands and environments
%%%%%%%%%%%%%%%%%%%%%%%%%%%%%%%%%%%%%%%%%%%%%%%%%%%%%%%%%%%%%%%%%%%%%%%%%%%%%%%%
%%% Misc
\sisetup{load-configurations=abbreviations}
\newcommand{\due}[1]{\date{Due: #1}}
\newcommand{\hint}{\textit{Hint}}
\let\oldt\t
\renewcommand{\t}[1]{\text{#1}}

%%% Bold sets & abbrv
\newcommand{\N}{\mathbb{N}}
\newcommand{\Z}{\mathbb{Z}}
\newcommand{\R}{\mathbb{R}}
\newcommand{\Q}{\mathbb{Q}}
\let\oldP\P
\renewcommand{\P}{\mathbb{P}}
\newcommand{\LL}{\mathcal{L}}
\newcommand{\FF}{\mathcal{F}}
\newcommand{\HH}{\mathcal{H}}
\newcommand{\NN}{\mathcal{N}}
\newcommand{\ZZ}{\mathcal{Z}}
\newcommand{\RN}[1]{\textup{\uppercase\expandafter{\romannumeral#1}}}
\newcommand{\ua}{\uparrow}
\newcommand{\da}{\downarrow}

%%% Unit vectors
\newcommand{\xhat}{\vb{\hat{x}}}
\newcommand{\yhat}{\vb{\hat{y}}}
\newcommand{\zhat}{\vb{\hat{z}}}
\newcommand{\nhat}{\vb{\hat{n}}}
\newcommand{\rhat}{\vb{\hat{r}}}
\newcommand{\phihat}{\bm{\hat{\phi}}}
\newcommand{\thetahat}{\bm{\hat{\theta}}}

%%% Other math stuff
\providecommand{\units}[1]{\,\ensuremath{\mathrm{#1}}\xspace}
% Set new style for problem
\newtheoremstyle{problemstyle}  % <name>
        {10pt}                   % <space above>
        {10pt}                   % <space below>
        {\normalfont}           % <body font>
        {}                      % <indent amount}
        {\bfseries\itshape}     % <theorem head font>
        {\normalfont\bfseries:} % <punctuation after theorem head>
        {.5em}                  % <space after theorem head>
        {}                      % <theorem head spec (can be left empty, 
                                % meaning `normal')>

% Set problem environment
\theoremstyle{problemstyle}
\newtheorem{problemenv}{Problem}[section]
\newenvironment{problem}[1]{%
  \renewcommand\theproblemenv{#1}%
  \problemenv
}{\endproblemenv}
% Set lemma environment
\newenvironment{lemma}[2][Lemma]{\begin{trivlist}
\item[\hskip \labelsep {\bfseries #1}\hskip \labelsep {\bfseries #2.}]}{\end{trivlist}}
% Set solution environment
\newenvironment{solution}{
    \begin{proof}[Solution]$ $\par\nobreak\ignorespaces
}{\end{proof}}
\numberwithin{equation}{problemenv}

%%% Page format
\setlength{\parindent}{0.5cm}
\setlength{\oddsidemargin}{0in}
\setlength{\textwidth}{6.5in}
\setlength{\textheight}{8.8in}
\setlength{\topmargin}{0in}
\setlength{\headheight}{18pt}

%%% Code environments
\definecolor{dkgreen}{rgb}{0,0.6,0}
\definecolor{gray}{rgb}{0.5,0.5,0.5}
\definecolor{mauve}{rgb}{0.58,0,0.82}
\lstset{frame=tb,
  language=Python,
  aboveskip=3mm,
  belowskip=3mm,
  showstringspaces=false,
  columns=flexible,
  basicstyle={\small\ttfamily},
  numbers=none,
  numberstyle=\tiny\color{gray},
  keywordstyle=\color{blue},
  commentstyle=\color{dkgreen},
  stringstyle=\color{mauve},
  breaklines=true,
  breakatwhitespace=true,
  tabsize=4
}
\lstset{
  language=Mathematica,
  numbers=left,
  numberstyle=\tiny\color{gray},
  numbersep=5pt,
  breaklines=true,
  captionpos={t},
  frame={lines},
  rulecolor=\color{black},
  framerule=0.5pt,
  columns=flexible,
  tabsize=2
}


\title{Midterm: Phys 7320 (Spring 2022)}
\due{March 9, 2022}

\begin{document}
\maketitle
%%%%%%%%%%%%%%%%%%%%%%%%%%%%%%%%%%%%%%%%%%%%%%%%%%%%%%%%%%%%%%%%%%%%%%%%%%%%%%%
\begin{problem}{M.1}[Radiation from two dipoles]
Consider a pair of identical dipoles, with dipole moments $p=qd$ where $d$ is
the separation of the charges in the dipole. The dipoles sit on the $z$-axis
centered at locations $z=\pm L$. Assume that $d\ll L$, so that we can treat each
dipole as essentially point-like. The dipoles are made to oscillate at frequency
$e^{-i\omega t}$.

(a) First, assume both dipoles are pointing in the positive $z$-direction. Go to
the radiation zone $r\gg L$ and $r\gg \lambda$, but do not assume anything about
$\lambda$ vs. $L$, and find an expression for the antenna pattern $dP/d\Omega$
for the whole system.

(b) Find the limit of the result of part (a) for large wavelengths $\lambda\gg
L$. Show this matches the leading multipole moment of the system, and give the
total power $P$ in this limit.

(c) Now assume the dipole at $z=L$ is pointing in the positive $z$-direction,
but the dipole at $z=-L$ is pointing in the negative $z$-direction (all times
the common factor $e^{-i\omega t}$). Again without assuming anything about
$\lambda$ vs. $L$, find the antenna pattern $dP/d\Omega$ for the whole system in
the radiation zone.

(d) Again take the limit of the result of part (c) for large wavelengths
$\lambda\gg L$ and find the leading nonzero contribution. Independently
calculate the leading multipole moment of the system and show this matches what
you just found. You don't have to calculate the total power.
\begin{solution}
\end{solution}
\end{problem}
\newpage
%%%%%%%%%%%%%%%%%%%%%%%%%%%%%%%%%%%%%%%%%%%%%%%%%%%%%%%%%%%%%%%%%%%%%%%%%%%%%%%    
%%%%%%%%%%%%%%%%%%%%%%%%%%%%%%%%%%%%%%%%%%%%%%%%%%%%%%%%%%%%%%%%%%%%%%%%%%%%%%%
\begin{problem}{M.2}[Diffraction from rectangular slits]
A perfectly conducting screen sits in the $xy$-plane at $z=0$. Planar
electromagnetic radiation is incident on the screen, moving in the positive
$z$-direction. We will approximate the EM wave with a scalar field $\psi$.

(a) First, let the screen have one aperture in it, a rectangle centered on the
origin from $x=-a$ to $x=a$ and from $y=-b$ to $y=b$. In the Fraunhofer limit,
use the scalar Dirichlet Kirchhoff formula to calculate the diffracted field
$\psi$ and the intensity $\abs{\psi}^2$. Describe what will appear on a screen
far away for $ka\sim kb\sim20$.

(b) Reduce your result for $\abs{\psi}^2$ from part (a) to the case of a slit: 
the limit $b\to0$ with $k$ and $a$ fixed, and $E_0$ scaled large so that the
quantity $\hat{E}_0\equiv 2bE_0$ stays finite in the limit; write the result in
terms of $\hat{E}_0$. Describe what happens to the $y$-dependence, and why this
is now effectively a 1-dimensional diffraction pattern. Sketch the intensity in
the $x$ direction for $ka\sim20$.

(c) Now change the screen so that it consists of two rectangular slits, each of
width $2a$ in the $x$ direction, centered at $x=\pm d$, with $d>a$. Again using
the Dirichlet scalar Kirchhoff formula in the Fraunhofer limit, calculate the
intensity for this case. Sketch the intensity as a function of $x$, assuming
that $d$ is bigger than $a$ by a factor of approximately 3.
\begin{solution}
\end{solution}
\end{problem}
\newpage
%%%%%%%%%%%%%%%%%%%%%%%%%%%%%%%%%%%%%%%%%%%%%%%%%%%%%%%%%%%%%%%%%%%%%%%%%%%%%%%
%%%%%%%%%%%%%%%%%%%%%%%%%%%%%%%%%%%%%%%%%%%%%%%%%%%%%%%%%%%%%%%%%%%%%%%%%%%%%%%
\begin{problem}{M.3}[A moving car]
Consider a car with length $l$, and choose the origin in its rest frame such
that the front of the car is at $x_F=0$, and the rear of the car is at $x_R=l$.

(a) Move to an inertial frame with coordinates $x',t'$ where the car is moving
to the left with constant velocity $v$. Find an equation relating $x_F'$ and
$t_F'$ (the worldline of the front of the car in this frame), and another
equation relating $x_R'$ and $t_R'$ (the world line of the back of the car in
this frame).

(b) Use your results from the previous part to obtain the length of the car in
the primed frame in terms of $l$ (that is, rederive length contraction from your
results). Also, what is the slope of the two worldlines in the spacetime diagram
with $ct'$ the vertical axis and $x'$ the horizontal axis?

(c) The car has a resting length of $l=5$\,\si{m}, a rest mass of 2000\,\si{kg},
and in the primed frame is moving at $3c/5$. Find the energy and momentum of the
car in its rest frame, and the energy, momentum and length of the car in the
moving frame. (The speed of light is $3\times10^8$\,\si{m/s}.)
\begin{solution}
\end{solution}
\end{problem}
\newpage
%%%%%%%%%%%%%%%%%%%%%%%%%%%%%%%%%%%%%%%%%%%%%%%%%%%%%%%%%%%%%%%%%%%%%%%%%%%%%%%
\end{document}
